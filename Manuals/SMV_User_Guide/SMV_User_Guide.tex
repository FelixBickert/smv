\documentclass[11pt,twoside]{book}

\newcommand{\feda}{\input smv_ug_fed1.tex}
\newcommand{\fedb}{\input smv_ug_fed2.tex}
%\newcommand{\feda}{}
%\newcommand{\fedb}{}

\newcommand{\utilchap}{\section}
\newcommand{\utilsect}{\subsection}
\newcommand{\textchap}{section}

%% include commands for bios, titles etc used in multiple documents
%%
\input{../Bibliography/commoncommands}
\IfFileExists{../Bibliography/gitrevision.tex}
{\input{../Bibliography/gitrevision}}
{\newcommand{\gitrevision}{unknown} }
\usepackage{picins}

%% commands only used by this guide

\newcommand{\svini}{{\tt smokeview.ini}\ }
\newcommand{\infigheight}{0.85in}
\newcommand{\figheightAbar}{2.2in}
\newcommand{\figheightC}{2.5in}
\newcommand{\infigr}[2]{
\parpic[r]{
\begin{tabular}{c}
\includegraphics[height=\infigheight]{SCRIPT_FIGURES/#1}\\
{\small\tt #2}
\end{tabular}
}
}
\newcommand{\infigl}[2]{
\parpic[l]{
\begin{tabular}{c}
\includegraphics[height=\infigheight]{SCRIPT_FIGURES/#1}\\
{\small\tt #2}
\end{tabular}
}
}
\newcommand{\frameit}[1]{\fbox{\tt #1}}
\newcommand{\kitem}[1]{\item[{\bf {\tt #1 \  }} \hfill]}
\newcommand{\figheight}{1.5in}
\newcommand{\figheightA}{2.5in}
\newcommand{\figwidth}{3.333333in}
\newcommand{\figwidthb}{2.0in}
\newcommand{\parma}{.75}
\newcommand{\parmb}{.5}
\newcommand{\parmc}{0.25}
\newcommand{\blist}{
\begin{list}
{}{
\setlength{\leftmargin}{\parma in}
\setlength{\labelwidth}{\parmb in}
\setlength{\labelsep}{\parmc in}
\setlength{\listparindent}{0.3in}
\setlength{\topsep}{.3in}
\setlength{\parsep}{.0in}
}}
\newcommand{\elist}{\end{list}}
\newcommand{\loadmenu}{\fbox{\ct Load/Unload}}
\newcommand{\hitem}[1]{\item[{\bf #1} \hfill]}
\newcommand{\hitemNULL}[1]{}
\newcommand{\hhitem}[2]{\item[{\bf #1}\ ({\em #2}) \hfill]}

% command to double space
%\linespread{2.0}
\begin{document}

\bibliographystyle{unsrt}
\pagestyle{empty}

%
% ----------------------  first cover/title page --------------------------
%
\begin{minipage}[t][9in][s]{6.5in}

\headerA{1017-1\\Sixth Edition\\}


\vspace{1in}

\headerB{
Smokeview, A Tool for Visualizing\\
Fire Dynamics Simulation Data\\
Volume I: User's Guide\\
}

\vspace{.5in}
\headerC{Glenn P. Forney}

\vfill

\begin{flushright}
\includegraphics[width=2.in]{\SMVfigdir/nistident_flright_vec}
\end{flushright}
\end{minipage}

\newpage

\hspace{5in}
\newpage

%
% ----------------------  second cover/title page --------------------------
%
\begin{minipage}[t][9in][s]{6.5in}

\headerA{1017-1\\Sixth Edition}

\vspace{1.in}

\headerB{
Smokeview, A Tool for Visualizing\\
Fire Dynamics Simulation Data\\
Volume I: User's Guide\\
}

\vspace{.5in}

\headerC{Glenn P. Forney\\
{\em Fire Research Division} \\
{\em Engineering Laboratory}  \\
}

\vspace{.25in}


%\flushright{\today \\
\begin{flushright}
\today \\
Revision:~\gitrevision
\end{flushright}
%
\vfill

\begin{flushright}
\includegraphics[width=1in]{\SMVfigdir/doc}
\end{flushright}

\titlesigs

\end{minipage}


\date{}

\setlength{\parindent}{0.25in}

\newpage

\begin{minipage}[t][9in][s]{6.5in}


\begin{flushright}
Certain commercial entities, equipment, or materials may be identified in this \\
document in order to describe an experimental procedure or concept adequately. Such \\
identification is not intended to imply recommendation or endorsement by the \\
National Institute of Standards and Technology, nor is it intended to imply that the \\
entities, materials, or equipment are necessarily the best available for the purpose.
\end{flushright}

\vspace{3in}


\vspace{3in}

\large
\begin{flushright}
\bf National Institute of Standards and Technology Special Publication 1017-1 \\
Natl.~Inst.~Stand.~Technol.~Spec.~Publ.~1017-1, \pageref{LastPage} pages (June 2016) \\
CODEN: NSPUE2
\end{flushright}

\vfill

\end{minipage}


\frontmatter

\pagestyle{plain}

%---------------------------------------------------------------------------------
%------------------------ Preface ------------------------------------------------
%---------------------------------------------------------------------------------

\chapter{Preface}
\smvoverview
This guide is Volume I the  Smokeview User's guide.

Smokeview is a software tool designed to visualize numerical
calculations generated by fire models such as the Fire Dynamics
Simulator (FDS), a computational fluid dynamics (CFD) model of
fire-driven fluid flow or CFAST, a zone fire model. Smokeview
visualizes smoke and other attributes of the fire using
traditional scientific methods such as displaying tracer particle
flow, 2D or 3D shaded contours of gas flow data such as
temperature and flow vectors showing flow direction and magnitude.
Smokeview also visualizes fire attributes realistically so that
one can {\em experience}\ the fire. This is done by displaying a
series of partially transparent planes where the transparencies in
each plane (at each grid node) are determined from soot densities
computed by FDS.  Smokeview also visualizes static data at
particular times again using 2D or 3D contours of data such as
temperature and flow vectors showing flow direction and magnitude.

Smokeview and associated documentation for Windows, Linux and Mac
OS X may be downloaded from the web site {\bf
\hhref{http://pages.nist.gov/fds}}\ at no cost.

%---------------------------------------------------------------------------------
%------------------------ About the Author ---------------------------------------
%---------------------------------------------------------------------------------

\chapter{About the Author}

\begin{description}
\gforneybio
\end{description}

%---------------------------------------------------------------------------------
%------------------------ Disclaimer ---------------------------------------------
%---------------------------------------------------------------------------------

\chapter{Disclaimer}

The US Department of Commerce makes no warranty,
expressed or implied, to users of Smokeview, and accepts no
responsibility for its use. Users of Smokeview assume sole
responsibility under Federal law for determining the
appropriateness of its use in any particular application; for any
conclusions drawn from the results of its use; and for any actions
taken or not taken as a result of analysis performed using this
tools.

Smokeview and the companion program FDS is intended for use only
by those competent in the fields of fluid dynamics,
thermodynamics, combustion, and heat transfer, and is intended
only to supplement the informed judgment of the qualified user.
These software packages may or may not have predictive capability
when applied to a specific set of factual circumstances. Lack of
accurate predictions could lead to erroneous conclusions with
regard to fire safety. All results should be evaluated by an
informed user.

Throughout this document, the mention of computer hardware or
commercial software does not constitute endorsement by NIST,
nor does
it indicate that the products are necessarily those
best suited for the
intended purpose.

%---------------------------------------------------------------------------------
%------------------------ Acknowledgements ---------------------------------
%---------------------------------------------------------------------------------

\chapter*{Acknowledgements}
A number of people have made significant contributions to the
development of Smokeview. In trying to acknowledge those that have
contributed, we are inevitably going to miss a few people.  Let us
know and we will include those missed in the next version of this
guide.

The original version of Smokeview was inspired by Frames, a
visualization program written by James Sims for the Silicon
Graphics workstation.  This software was based on visualization
software written by Stuart Cramer for an Evans and Sutherland
computer. Frames used tracer particles to visualize smoke flow
computed by a pre-cursor to FDS. Judy Devaney made the
multi-screen eight foot Rave facility available allowing a stereo
version of Smokeview to be built that can display scenes in
3D.  Both Steve Satterfield and Tere Griffin on many occasions
helped me demonstrate Smokeview cases on the Rave inspiring many
people to the possibility of using Smokeview as a {\em virtual
reality-like}\ fire fighter training facility.

Many conversations with Nelson Bryner, Dave Evans, Anthony Hamins
and Doug Walton were most helpful in determining how Smokeview
could be adapted for use in fire fighter training applications.

Smokeview would not be possible without the use of a number of
software libraries developed by others.  Mark Kilgard while at
Silicon Graphics developed GLUT, the basic tool kit for
interfacing OpenGL with the underlying operating system on
multiple computer platforms. Paul Rademacher while a graduate
student at the University of North Carolina developed GLUI, the
software library for implementing the user friendly dialog boxes.

Significant contributions have been made by those that have used
Smokeview to visualize complex cases; cases that are used to
perform both applied and basic research.  The resulting feedback
has improved Smokeview as a result of their interaction with me,
pushing the envelope and not accepting the status quo.

For applied research, Daniel Madrzykowski, Doug Walton and Robert
Vettori of NIST have used Smokeview to analyze fire incidents.
Steve Kerber has used Smokeview to visualize flows resulting from
positive pressure ventilation (PPV) fans. David Stroup has used
Smokeview to analyze cases for use in fire fighter training
scenarios.  Conversations with Doug Walton have been particularly
helpful in identifying needed features and clarifying how best to
make their implementation user friendly.  David Evans, William
(Ruddy) Mell and Ronald Rehm used Smokeview to visualize {\em
wildland-urban interface}\ fires.   For basic research, Greg
Linteris has used Smokeview to visualize fire simulations
involving the cone calorimeter. Anthony Hamins has used Smokeview
to visualize the structure of CH$_4$/air flames undergoing the
transition from normal to microgravity conditions and fire
suppression in a compartment. Jiann Yang has used Smokeview to
visualize smoke or particle number density and saturation ratio of
condensable vapor.

This user's guide has improved through the many constructive
comments of the reviewers Anthony Hamins, Doug Walton, Ronald
Rehm, and David Sheppard. Chuck Bouldin helped port Smokeview to
the Macintosh.

Many people have sent in multiple comments and feedback by email,
in particular Adrian Brown, Scot Deal, Charlie Fleischmann, Jason
Floyd, Simo Hostikka, Bryan Klein, Davy Leroy, Dave McGill, Brian
McLaughlin, Derek Nolan, Steven Olenick, Stephen Priddy, Boris
Stock, Jason Sutula, Javier Trelles, and Christopher Wood.

Feedback is encouraged and may be sent to glenn.forney@nist.gov .

\cleardoublepage
\tableofcontents

\cleardoublepage
\listoffigures

\cleardoublepage
\listoftables

\mainmatter

\pagenumbering{arabic}

%---------------------------------------------------------------------------------
%------------------------ Introduction ----------------------------------------
%---------------------------------------------------------------------------------

\part{Using Smokeview}
\chapter{Introduction}
%---------------------------------------------------------------------------------
%---------------------------------------------------------------------------------
\section{Overview}
Smokeview is a scientific software tool designed for visualizing
numerical predictions generated by fire models such as the Fire
Dynamics Simulator (FDS), a computational fluid dynamics (CFD)
model of fire-driven fluid flow~\cite{FDS_Tech_Guide}\ and CFAST, a
zone model of compartment fire phenomena~\cite{CFAST_Tech_Guide_7}. This
report documents version 6 of Smokeview. For details on setting up
and running FDS cases read the FDS User's
guide~\cite{FDS_Users_Guide}.

FDS and Smokeview are primarily used to model and visualize
time-varying fire phenomena. FDS and Smokeview are not limited to
fire simulation, however. For example, one may use these
applications to model other phenomena such as contaminant flow in
a building or evacuation flow. Smokeview performs visualizations
by displaying time dependent tracer particle flow, animated
contour slices of computed gas variables and surface data.
Smokeview also presents contours and vector plots of static data
anywhere within a simulation scene at a fixed time. Several
examples using these techniques to investigate fire incidents are
documented in Refs.~\cite{CHERRYROAD,Iowa,HOUSTON,WTC}.

Smokeview is used before, during and after model runs. Smokeview
is used in a post-processing step to visualize FDS data after a
calculation has been completed. Smokeview  may also be used during
a calculation to monitor a simulation's progress and before a
calculation to setup FDS input files more quickly.  Figure
\ref{figfdsoverview}\ gives an overview of how data files used by
FDS,  Smokeview and Smokezip, a program used to compress FDS
generated data files, are related.

Smokeview is written in C~\cite{C:book}\ and C++\cite{Cpp17}. It consists
of about \smvlines\ lines of code. The C portion visualizes the
data, while the C++ portoin is used to implement dialog boxes.
Smokeview uses the 3D graphics
library OpenGL~\cite{OpenGLRed}\ for generating the visualizations
and the Graphics Library Utility Toolkit (GLUT)~\cite{OpenGLGlut}
for interacting with the underlying OS. Smokeview uses the GLUT
software library so that most of the development effort can be
spent implementing the visualizations rather than creating an
elaborate user interface. Smokeview uses a number of auxiliary
libraries to implement image capture (GD~\cite{BOUTELL,GDLIB},
PNG~\cite{PNGLIB}, JPEG~\cite{JPEGLIB}), image and general file
compression (ZLIB~\cite{ZLIB}) and dialog creation
(GLUI~\cite{GLUILIB}). Each of these libraries is portable running
under LINUX, OS~X and Windows allowing
Smokeview to run on these platforms as well.
\begin{figure}[bph]
\centerline{
\includegraphics[width=6.5in]{\SMVfigdir/SMV_Overview_Diagram}}
 \caption[FDS file overview]{Diagram illustrating files used and created by the Fire Dynamics
 Simulator (FDS), Smokezip and Smokeview.}
\label{figfdsoverview}%
\end{figure}

%---------------------------------------------------------------------------------
%---------------------------------------------------------------------------------
\section{Features}

Smokeview is a program designed to visualize numerical
calculations generated by the Fire Dynamics Simulator.
The version of FDS used to run the cases illustrated  in this report
is given by:
\lstinputlisting{SCRIPT_FIGURES/fds.version}
The version of Smokeview described here and used
to generate most figures in this report is given by:
\lstinputlisting{SCRIPT_FIGURES/smokeview.version}

%%-----------------------------------------------------------------------
\subsection{Visualizing Data}

Smokeview visualizes data primarily generated by FDS.
Smokeview visualizes data that is both dynamic and static.  Dynamic
data is visualized by animating particle flow (showing
location and {\em values}\ of tracer particles), 2D contour
slices (both within the domain and on solid surfaces) and
3D iso surfaces.  2D contour slices can also be drawn
with colored vectors that use velocity data to show flow
direction, speed and value. Static data is visualized
similarly by drawing 2D contours, vector plots and 3D level
surfaces.

%%-----------------------------------------------------------------------
\subsubsection{Particles}\ Lagrangian or moving particles
(Section \ref{section:particles}) may be
used to visualize the flow field. Often these particles represent
smoke or water droplets.  Particles may also be used to represent
people when modeling evacuation flow.

Particle data may also be visualized as streak lines (a particle
drawn where it has been for a short period of time in the past).
Streak lines are a good method for displaying motion using static
images.

%%-----------------------------------------------------------------------
\subsubsection{Volumetric - Realistic Smoke}
Smoke and fire (heat release rate per unit volume) are displayed
realistically using a series of partially transparent planes
(Section \ref{section:volsmoke}). The smoke transparencies are
determined by using smoke densities computed by FDS.  The fire and
sprinkler spray transparencies are determined by using a heuristic
based on heat release rate and water density data, again computed
by FDS. Various settings for the 3D smoke option may be set using
the 3D Smoke dialog box found in the {\em Dialogs$>$Data bounds}\ menu.
The windows version of Smokeview uses the graphical processing
unit (GPU) on the video card to perform some of the calculations
required to visualize smoke.

%%-----------------------------------------------------------------------
\subsubsection{Slices - 2D contours}
Animated 2D shaded color contour plots (Section
\ref{section:slices}) are used to visualize gas phase information,
such as temperature or density. The contour plots are drawn in
horizontal or vertical planes along any coordinate direction.
Contours can also be drawn in shades of gray.
Shaded contours may also be used to visualize information
computed on solid objects (Section \ref{section:bf}).  These contours are known as boundary files.

Animated 2D shaded color contour plots are also used to
visualize solid phase quantities such as radiative flux or
heat release rate per unit area.

Vector slice files may be visualized if U, V and W velocity slice files are recorded.
Though
similar to solidly shaded contour animations (the vector colors are
the same as the corresponding contour colors), vector animations are better
than solid contour animations for highlighting flow
features since vectors accentuate the direction that flow is occurring.

A 3D region of
data may be visualized using slice files.  Slices may be moved from one plane to
the next just as with Plot3D files (using up/down cursor keys or
page up/page down keys).
3D slices may also be rotated and/or translated by double clicking and
moving the mouse. If the {\tt CTRL}\ or {\tt d}\ key
is also pressed
(press and release the {\tt d}\ key do no hold it down),
the slice moves up and down.
If the {\tt ALT}\ or {\tt f}\ key is pressed,
(press and release the {\tt f}\ key do no hold it down),
the slice moves side to side.
Otherwise, the slice rotates.
Data for 3D slice files are generated by specifying a 3D rather than a
2D region with the {\tt \&SLCF}\ keyword.

Data computed at cell centers rather than interpolated at cell nodes may be visualized.
This is useful for investigating numerical algorithms as the data visualized
has not been interpolated before being seen.

%%-----------------------------------------------------------------------
\subsubsection{2D Plots}
Spreadsheet data or csv files may be used to generate 2D plots of data (Chapter \ref{chap:2DPLOT}).  2D plots may
also be generated from data at specified locations or regions in a slice file.

\subsubsection{Surfaces - 3D contours}
Isosurface or 3D level surface animations (Section
\ref{section:isosurface}) may be used to represent flame
boundaries, layer interfaces and various other gas phase
variables. Multiple isocontours may be stored in one file,
allowing one to view several isosurface levels simultaneously.


%%-----------------------------------------------------------------------
\subsection{Exploring Data}

\subsubsection{Data Mining}\ The user can analyze and examine the simulated
data by altering its appearance to more easily identify features
and behaviors found in the simulation data. One may flip or
reverse the order of colors in the colorbar and also click in the
colorbar and slide the mouse to highlight data values in the
scene. These options may be found under the {\em Options/Shades}
menu.

The user may click in the timebar and slide the mouse to
change the simulation time displayed. One use for the timebar and colorbar selection modes might be to determine
when smoke of a particular temperature enters a room.

\subsubsection{Data Filtering}\ The File/Bounds Settings...
dialog box allows one to set bounds, to chop or hide data and in the case
of slice file data to time average. (Chapter \ref{chapter:settingoptions})
The data chopping
feature is useful for highlighting data.  A ceiling jet, for example,
may be visualized by hiding ambient temperature
data,  data below a prescribed temperature.
Using time averaging allows one to smooth noisy data over a user selectable time
interval.

\subsubsection{Data coloring}\ Multiple colorbars are available for displaying simulation data.
New colorbars may be created using the colorbar editor (Section \ref{section:colorbar}).
Colorbars may then be adapted to best highlight the simulation data visualized.
Regions in the simulation with certain data values may be highlighted by clicking
on the colorbar.

\subsubsection{Data Compression}\ An option has been added to the
{\em LOAD/UNLOAD}\ menu to compress 3D smoke and boundary
files (Section \ref{ch:smokezip}). The option shells out to the program Smokezip which runs in
the background enabling one to continue to use Smokeview while
files are compressing.

\subsubsection{Data Comparison}\ A stand alone utility program named Smokediff
may be used to compare two FDS cases (Section \ref{ch:smokediff}).
Smokediff generates the difference between corresponding slice and boundary
files for two cases with the same geometry.  Smokediff creates a
{\tt .smv}\ of the differenced data which may then be viewed with Smokeview.


%%-----------------------------------------------------------------------
\subsection{Exploring the Scene}

\subsubsection{Motion/View/Render}\ The Motion/View/Render dialog box may be used to
allow more precise control of scene movement and orientation.
Cursor keys have been mapped to scene translation/rotation to
allow easy navigation within the scene.  Viewpoints may be saved for later access.

The first person or eye view mode for moving
allows one to move through a scene more
realistically (Section \ref{section:eyeview}).  Using the cursor keys and the
mouse, one can move through a scene {\em virtually}.

\subsubsection{Stereo views}A method for displaying stereo/3D
images has been implemented that does
not require any specialized equipment such as shuttered glasses or
quad buffered enabled video cards (Section \ref{section:stereo}) .
Stereo pair images are displayed side by side after invoking the option with
the Stereo dialog box or pressing the ``S'' key (upper case).
A 3D view appears by relaxing the eyes, allowing the two images to merge into one.
Pressing the ``S'' key again results in stereo views generated by
displaying red and blue versions of the scene.
Glasses with a red left lens and a blue right lens are required to view the image.

\subsubsection{Scene Clipping}\ It is often difficult to visualize data
in complicated geometries due to the number of obstructed
surfaces. Interior portions of the scene may be seen more easily
by clipping part of the scene away. (Section
\ref{section:clipping})

Clipping discussed above occurs in 3D within the scene.  A
screenshot converted to a PNG or JPEG file may also be clipped or
cropped using the Render portion of the Motion/View/Render
dialog box.

%%-----------------------------------------------------------------------
\subsection{Automating the Visualization}
\subsubsection{Virtual Tour}\   A series of checkpoints or keyframes
specifying position and view direction may be specified. (Chapter
\ref{chapter:touring}) A smooth path is computed using
Kochanek-Bartels splines~\cite{Moller:02}\ to go through these key
frames so that one may control the position and view direction of
an observer as they move through the simulation. One can then see
the simulation as the observer would. This option is available
under the {\em Tour}\ menu item. Existing tours may be edited and
new tours may be created using the Tour dialog box found in
the {\em Dialogs$>$View}\ menu. Tour settings are stored in the local
configuration file (casename.ini).

\subsubsection{Scripting}\ Smokeview may be run in an unattended mode using
instructions found in a script file. (Chapter
\ref{chapter:scripting}) These instructions direct Smokeview to
load data files, load configuration files, set view points and
time values in order to document a case by rendering the Smokeview
scene into one or more image files. The script file may be created
by Smokeview as a user performs various actions or may be created
by editing a text file.

%%-----------------------------------------------------------------------
\subsection{Customizing the Scene}

\subsubsection{Objects}A method for drawing realistic appearing objects such as a heat detector,
smoke detector, sprinkler sensor, etc. has been implemented.
(Chapter \ref{chap:devices}) . Objects are specified in a data
file rather than in Smokeview as C code. This allows one to
customize the look and feel of the objects (to match the types of
detectors/sprinklers that are being used) without requiring code
changes in Smokeview.

\subsubsection{Texture Mapping}\ Image files may be drawn over top
of a blockage, vent or enclosure boundary (Chapter
\ref{chapter:texturemaps}). This is called texture mapping.  This
allows Smokeview scenes to appear more realistic. These image
files may be obtained from the internet, a digital camera, a
scanner or from any other source that generates these file
formats. Image files used for texture mapping should be seamless.
A seamless texture as the name suggests is periodic in both
horizontal and vertical directions. This is an especially
important requirement when textures are tiled or repeated across a
blockage surface.

\subsubsection{Annotating Cases}Text may be added to a scene in order to help document Smokeview
output. (Chapter \ref{section:annotate}) It allows one to place
colored labels at specified locations at specified times. A second
keyword, {\tt TICK}\ keyword places equally spaced tick marks
between specified bounds. These marks along with {\tt LABEL}\ text
may be used to specify length scales in the scene.

The {\em User Tick Settings}\ tab of the Display dialog box
provides an easier way to place ticks with length annotations
along coordinate axes.


%---------------------------------------------------------------------------------
%------------------------ Getting Started -------------------------------------
%---------------------------------------------------------------------------------

\section{Getting Started}

%%-----------------------------------------------------------------------
\subsection{Obtaining Smokeview}

Smokeview is available at \hhref{http://pages.nist.gov/fds/downloads.html}.
This web page contains links to FDS and Smokeview installers for
Windows, Linux and Mac~OS~X computer platforms. It also contains documentation for
Smokeview and FDS, sample FDS input files, software updates and
links for requesting feedback about the software.

After obtaining the setup program, install Smokeview (and FDS)  on a PC by
double-clicking the downloaded
setup program. The setup program then steps through the
installation. It copies the FDS and Smokeview executables,
sample input files and documentation to the selected directory.  The setup program also
defines PATH variables and associates the {\tt .smv}\ file
extension to the Smokeview program so that one may either type
Smokeview at any command line prompt or double click on any {\tt
.smv}\ file. Smokeview uses the OpenGL graphics library which is a
part of all Windows distributions.

Most computers purchased today are perfectly adequate for running
Smokeview. For Smokeview it is more important to obtain a fast
graphics card than a fast CPU. If the computer will run both FDS
and Smokeview, then a fast CPU is important as well.
For example, the townhouse case used for many examples in this
report consists of about 23000 grid cells.  This case requires
about 13 CPU minutes on a 3.4~GHZ Intel Windows 10
system. Cases with more grid cells and longer simulation times
(the townhouse case simulated 60~s of smoke flow) would clearly
benefit from a faster CPU and more memory which are now relatively
inexpensive.

%---------------------------------------------------------------------------------
%------------------------ Basics  ------------------------------------------------
%---------------------------------------------------------------------------------

%%-----------------------------------------------------------------------
\subsection{Running Smokeview}

See the FDS User's guide for running FDS\cite{FDS_Users_Guide}.  Briefly, a typical procedure for using FDS and Smokeview is to:
\begin{enumerate}

\item Create a file named {\tt casename.fds}\ describing the fire
scenario.

\item Type {\tt fds\_local~casename.fds}\ in a command shell on a Windows PC (or fds on other platforms) to run the
case.

\item Double click on the file named {\tt casename.smv}\ (if on the
PC) or type {\tt smokeview~casename}\ in a command shell (on other
platforms) to start Smokeview.

\item Right clicking within the scene and select a file to load
within the {\em Load/Unload}\ menu.
\end{enumerate}

\noindent This report documents step 3 and 4. Steps 1 and 2 are
documented in the FDS User's Guide~\cite{FDS_Users_Guide}.

Smokeview menus are selected by clicking the right mouse
button anywhere within the Smokeview window.  Data files may be
visualized by selecting the desired {\em Load/Unload}\ menu item.
Other menu options are discussed in Appendix
\ref{sectionmenu}. Many menu commands have equivalent keyboard
shortcuts. These shortcuts are listed in Smokeview's {\em Help}\
menu and are described in Appendix \ref{sectionkeyboard}.
Visualization features not controllable through the menus may be
customized by using the Smokeview preference file, \svini,
discussed in Appendix \ref{appendixini}.

Smokeview  is started on a Windows PC by double-clicking the file
named {\tt casename.smv}\ where casename is the name specified by
the {\tt CHID}\ keyword defined in the FDS input data file. Menus
are accessed by clicking with the right mouse button.  The {\em
Load/Unload}\ menu may be used to read in the data files to be
visualized. The {\em Show/Hide}\ menu may be used to change how
the visualizations are presented. For the most part, the menu
choices are self explanatory. Menu items exist for showing and
hiding various simulation elements, creating screen dumps,
obtaining help, etc. Menu items are described in Appendix
\ref{sectionmenu}.

To use Smokeview from a command line, open a command shell. Then
change to the directory containing the FDS case to be viewed and
type:
\begin{lstlisting}
smokeview casename
\end{lstlisting}
where again casename is the name specified by the {\tt CHID}
keyword defined in the FDS input data file. Data files may be
loaded and options may be selected by clicking the right mouse
button and picking the appropriate menu item.


Smokeview opens two windows, one displays the scene and the other
displays status information. Closing either window will end the
Smokeview session.  Multiple copies of Smokeview may be run
simultaneously if the computer has adequate resources.

Normally Smokeview is run during an FDS run, after the run has
completed and as an aid in setting up FDS cases by visualizing
geometric components such as blockages, vents, sensors, etc. One
can then verify that these modeling elements have been defined and
located as intended. One may select the color of these elements
using color parameters in the \svini\ to help distinguish one
element from another. \svini\ file entries are described in
section \ref{appendixini}.

Although specific video card brands cannot be recommended, they
should be high-end due to Smokeview's intensive graphics
requirements. These requirements will only increase in the future
as more features are added.  A video card designed to perform well
for {\em fancy}\ computer games should do well for Smokeview. Some
apparent bugs in Smokeview have been found to be the result of
problems found in video cards on older computers.

%---------------------------------------------------------------------------------
%---------------------------------------------------------------------------------
\section{Manipulating the Scene}

A Smokeview scene may be rotated or moved by using either the mouse or the Motion/View/Render dialog box.
The scene may be rotated about a point within the scene, usually the scene center, or rotated about the point where the user is located.
In either case, to rotate, click the scene with the left mouse button and drag either horizontally or vertically.
Motion about the
scene center which is the default is called world or global view while motion about the
user location is called eye or first person view. These viewing modes
 may be swapped by pressing the ``e'' key or by selecting
the appropriate radio button in the Motion/View/Render
dialog box.

Similarly, the scene may be translated by clicking the scene with the left mouse button while either the {\tt ALT}\ or {\tt CTRL}\ keys are pressed.
The {\tt ALT} key results in vertical motion of the scene while the {\tt CTRL} key results in motion in and out.

%%-----------------------------------------------------------------------
\subsection{World View}

The scene may be rotated or translated using the mouse or
by using controls in the Motion/View/Render dialog box.
This dialog box, illustrated in Fig. \ref{figMOTIONmotion}, is opened
using the {\em Dialogs$>$Motion}\ menu item.

Clicking the left mouse button and dragging horizontally,
vertically or a combination of both results in scene rotation or
translation depending upon whether the CTRL or ALT modifier keys
are pressed or not.

\blist

\hitem{no modifier keys}\ Horizontal mouse movement results in
scene rotation about the Z axis. Vertical mouse movement results
in scene rotation about the X axis.  If the 3-axis rotation option
is selected then mouse movement around the periphery of the scene
results in clockwise or counter clockwise movement about the Y
axis.

\hitem{CTRL key depressed}\ Horizontal mouse movement results in
scene translation from side to side along the X axis. Vertical
mouse movement results in scene translation in and out of the
along the Y axis.

\hitem{ALT key depressed}\ Vertical mouse movement results in scene
translation up and down along the Z axis. Horizontal mouse
movement has no effect.

\elist

%%-----------------------------------------------------------------------
\subsection{First Person View}
\label{section:eyeview}
First person view is entered by either pressing the appropriate
radio buttons in the Motion/View/Render dialog box (button
labeled eye centered) or by pressing the ``e'' key until first
person view is obtained. When in {\em eye center}\ mode, several
key mappings have been added, inspired by popular computer games,
to allow for easier movement within the scene. For example, the up
and down cursor keys allow one to move forward or backwards.  The
left and right cursor keys allow one to rotate left or right.
Other keyboard mappings are described in Table \ref{tabKEYS}.

\begin{table}[bph]
\begin{center}
\caption{Keyboard mappings for {\em eye centered}\ or first person scene movement.}
\vspace{0.1in}
\begin{tabular}{|l|l|}
\hline Key &   Description  \\

\hline\hline
up/down cursor & \multirow{2}{*}{move forward/backward}\  \\
w/s &   \\\hline
{\tt ALT}\ + left/right cursor  & \multirow{2}{*}{slide left/right}\ \\
{\tt a/d}\  &  \\ \hline
{\tt ALT}\ + up/down cursor  & move up/down  \\ \hline\hline
left/right cursor  & rotate left/right \\ \hline
{\tt Page Up/Down}\  & look up/down \\ \hline
{\tt Home}\  & look level \\ \hline\hline
\multicolumn{2}{|p{3.5in}|}{Pressing the {\tt SHIFT}\ key while moving, sliding or rotating
results in a  4x speedup of these actions. }\ \\ \hline

\end{tabular}
\label{tabKEYS}
\end{center}
\end{table}

%%-----------------------------------------------------------------------
\subsection{Motion Dialog Box}
The Motion dialog box illustrated in Fig \ref{figMOTIONmotion} may also be used to manipulate the
scene. Buttons in the Motion
region of the dialog box allow one to translate or rotate the scene. The
\frameit{Horizontal}\ button is used to translate the scene
horizontally within a horizontal plane (left/right or in/out within the scene).  The
\frameit{Vertical}\ button allows one to translate the scene vertically.
Scene translation and rotations may also be specified by using controls to set x, y, z
coordinates and azimuth, elevation rotation angles within the {\em Specify Orientation}\ panel.
Within this same panel one may also specify whether the gravity vector (usually pointed down) is visualized and/or used to draw the scene.

The \frameit{Rotate about}\ selection list allows one to change the center of rotation.
Rotation center choices are: the scene center, denoted as {\em world center}\ in this list, the center of each mesh and a center specified by the user.
Changing the rotation center is useful for cases where the portion of the scene being viewed is far away from the currently used rotation center.
To allow for easier changes, the rotation center may be made visible (drawn as a small black sphere) by selecting the {\em Show}\ checkbox within the {\em rotation center panel}.  The rotation center may be changed by using the
{\em x, y, z}\ spinners below this checkbox.

\begin{figure}[bph]
\centerline{
\begin{tabular}{cc}
\includegraphics[width=2.722222in]{\SMVfigdir/figMOTION}
\end{tabular}
}\ \caption[Dialog box for controlling scene motion.]{Dialog box for controlling scene motion. To rotate the scene, select the rotation type
then select and move the mouse in the  rotation control.  One may also select where to rotate about (scene center, any mesh center or user specified center). To translate the scene, select and move the mouse in the horizontal or vertical control.}\ \label{figMOTIONmotion}
\end{figure}

%---------------------------------------------------------------------------------
%---------------Realistic or Qualitative Visualization - 3D Smoke--------------
%---------------------------------------------------------------------------------

\chapter{Visualizing Smoke}

\section{Tracers and Streaklines}

\renewcommand{\figheight}{1.4in}

\label{section:particles}\ Particle files contain the locations of
tracer particles used to visualize the flow field. Figure
\ref{figparticle}\ shows several snapshots of a developing kitchen
fire visualized by using particles where particles are colored
black. If present, sprinkler water droplets would be colored blue.
Particles are stored in files ending with the extension {\tt
.prt5}\ and are displayed by selecting the particle file entry from
the {\em Load/Unload}\ menu.

Streaklines are a technique for showing motion in a still image.
Figure \ref{figstreak}\ shows a snapshot of the same kitchen fire
using streak lines instead of particles.  The streaks begin at 9~s
and end at 10~s.


\begin{figure}[bph]
\begin{center}
\begin{tabular}{cc}
 \includegraphics[height=\figheightA]{SCRIPT_FIGURES/thouse5_part_005}&
 \includegraphics[height=\figheightA]{SCRIPT_FIGURES/thouse5_part_010}\\
 5.0 s&10.0 s\\
\includegraphics[height=\figheightA]{SCRIPT_FIGURES/thouse5_part_030}&
\includegraphics[height=\figheightA]{SCRIPT_FIGURES/thouse5_part_060}\\
30.0 s&60.0 s\\
\end{tabular}
\end{center}

\caption{Townhouse kitchen fire visualized using tracer
particles.}
\label{figparticle}%
\end{figure}

\begin{figure}[bph]
\begin{center}
\includegraphics[height=4.0in]{SCRIPT_FIGURES/thouse5_streak_010}
\end{center}

\caption{Townhouse kitchen fire visualized using streak lines. The
{\em pin heads}\ shows flow conditions at 10~s, the corresponding
{\em tails}\ shows conditions 1.0~s earlier.}
\label{figstreak}%
\end{figure}

Particle file data may be converted to an isosurface using
Smokezip.  The isosurface location is defined in terms of particle
density and the isosurface color is defined in terms of averaged
particle values. See  Chapter \ref{ch:smokezip}\ for more details
on using Smokezip for generating isosurface files from particle
files and Section \ref{section:isosurface}\ for some examples.

%---------------------------------------------------------------------------------
%---------------------------------------------------------------------------------
\section{Realistic}
\label{section:volsmoke}\ FDS generates several data files
visualized by Smokeview. Each file type may be loaded or unloaded
using the {\em Load/Unload}\ menu described in Appendix
\ref{sectload}. Visualizations produced by these data files are
described in this and the following sections. The format used to
store each of the data files is given in the FDS User's
Guide~\cite{FDS_Users_Guide}.

Visualizing smoke realistically is a daunting challenge for at
least three reasons. First, the storage requirements for
describing smoke can easily exceed the disk capacities of present
32 bit operating systems such as Linux, i.e., file sizes can easily
exceed 2 gigabytes. Second, the computation required both by the
CPU and the video card to display each frame can easily exceed
0.1~s, the time corresponding to a 10~frame/s display rate. Third,
the physics required to describe smoke and its interactions with
itself and surrounding light sources is complex and
computationally intensive. Therefore, approximations and
simplifications are required to display smoke rapidly.

Smoke visualization techniques such as tracer particles or shaded
2D contours are useful for quantitative analysis but not suitable
for virtual reality applications, where displays need to be
realistic and fast as well as accurate. The approach taken by
Smokeview is to display a series of parallel planes.  Each plane
is colored black (for smoke) with transparency values pre-computed
by FDS using time dependent soot densities also computed by FDS
corresponding to the grid spacings of the simulation. The
transparencies are adjusted in real time by Smokeview to account
for differing path lengths through the smoke as the view direction
changes. The graphics hardware then combines the planes together
to form one image.

Fire by default is colored a dark shade of orange wherever the
computed heat release rate per unit volume exceeds a user-defined
cutoff value.  The visual characteristics of fire are not
automatically accounted for.  The user though may use the 3D
Smoke dialog box to change both the color and transparency of
the fire for fires that have non-standard colors and opacities.

The windows version of Smokeview has the option of using the GPU
or graphics programming unit to perform some of the calculations
required to visualize realistic smoke.  These calculations consist
of adjusting the smoke opaqueness as pre-computed in FDS to
account for off-axis viewing directions. The GPU performs the
computations in parallel while the former method using the CPU
performs them sequentially.  For many (but not all) cases, the use
of the GPU results in a smoke drawing speed up of 50 \% or more.
This option is turned on or off by pressing the {\tt G}\ key.


Figure \ref{figsmoke3d}\ illustrates a visualization of realistic
smoke.

\begin{figure}[bph]
\begin{center}
\begin{tabular}{cc}
 \includegraphics[height=\figheightA]{SCRIPT_FIGURES/thouse5_smoke_005}&
 \includegraphics[height=\figheightA]{SCRIPT_FIGURES/thouse5_smoke_010}\\
 5.0 s&10.0 s\\
\includegraphics[height=\figheightA]{SCRIPT_FIGURES/thouse5_smoke_030}&
\includegraphics[height=\figheightA]{SCRIPT_FIGURES/thouse5_smoke_060}\\
30.0 s&60.0 s\\
\end{tabular}
\end{center}
\caption{Smoke3d file snapshots at various times in a simulation
of a townhouse kitchen fire.
  }
\label{figsmoke3d}%
\end{figure}

%---------------------------------------------------------------------------------
%------------------------ Visualizing Data Quantitatively ------------------------
%---------------------------------------------------------------------------------

\chapter{Visualizing Data Quantitatively}

%---------------------------------------------------------------------------------
%---------------------------------------------------------------------------------
\section{Coloring data}Smokeview uses a 1D texture map for coloring data occurring in
slice, boundary and Plot3D files. 1D texture maps or colorbars may
be selected using the Data coloring dialog box illustrated in Fig. \ref{figDatacoloring}.
This dialog box is used to control how data is colored.
One may select the mapping used to associate data with color (a colorbar).
One may also select the colors used to color extreme data, data with values greater
the maximum or smaller than the minimum (smallest and greatest colorbar data labels).
When the colorbar is selected with the mouse a portion of it changes color to black.
Data in the scene with the same values are also colored black.
The width of the selection region (default 5 pixels) may be selected with this dialog box.
The number of digits used in colorbar labels may be specified.
Other data coloring properties such as transparency, order of the colorbar may also be selected.


\begin{figure}[bph]
\centerline{\includegraphics[width=6.5in]{\SMVfigdir/figDatacoloring}
}\ \caption [Dialog box for selecting colorbars.] {Dialog
box for selecting colorbars.  The selected colorbar may be modified by
shading it continuously, stepped or as a series of discrete lines.  The
colorbar may also be converted to shades of gray.  Colors for
extreme data (data outside of specified bounds) may be specified. Two colorbars may be toggled
to compare how data appears.
A split colorbar may be specified using one range of colors between the minimum
value and the split and another range of colors between the split and the maximum value.}
\label{figDatacoloring}
\end{figure}

\begin{figure}[bph]
\begin{center}
\begin{tabular}{ccc}
\includegraphics[height=\figheightAbar]{SCRIPT_FIGURES/thouse5_slicesplit_005}&
\includegraphics[height=\figheightAbar]{SCRIPT_FIGURES/thouse5_slicesplit_010}\\
5.0 s&10.0 s\\
\includegraphics[height=\figheightAbar]{SCRIPT_FIGURES/thouse5_slicesplit_030}&
\includegraphics[height=\figheightAbar]{SCRIPT_FIGURES/thouse5_slicesplit_060}&\\
30.0 s&60.0 s
&\raisebox{0.0ex}[0pt]{\includegraphics[height=5.0in]{\SMVfigdir/colorbar_uvel}}\\
\end{tabular}
\caption [Slice file snapshots of shaded U velocity contours using a colorbar with a split at 0.0~m/s.]
{Slice file snapshots of shaded U velocity contours at various
times in a simulation using a colorbar with a split at 0.0~m/s. Velocities greater than 0.0~m/s are colored with shades of red.
Velocities less than 0.0~m/s are colored with shades of blue. }
\label{figslicesplit}%
\end{center}
\end{figure}


A colorbar defined with a split may also be specified with this dialog box.  One range of colors are specified between the minimum
data value and the split value and another range of colors are specified between the split value and the maximum value. This is useful when highlighting
data that has a special property, for example a tenability temperature criteria or where flow velocity reverses.
Figure \ref{figslicesplit} illustrates a scene using a colorbar with a split at 0.0~m/s. Velocities greater than 0.0~m/s are colored with shades of red.
Velocities less than 0.0~m/s are colored with shades of blue.

Note, due to the way that transparent objects are drawn (from back to front),
3D Smoke/Fire and transparent slices may not display
properly when shown at the same time.

%---------------------------------------------------------------------------------
%---------------------------------------------------------------------------------
\section{2D Shaded Contours and Vector Slices - Slice Files}
\label{section:slices}
%%-----------------------------------------------------------------------
\subsection{Axis aligned slices}

Slice files contain results recorded within a rectangular array of
grid points at each recorded time step. Continuously shaded
contours are drawn for simulation quantities such as temperature,
gas velocity and heat release rate. Figure \ref{figslice}\ shows
several snapshots of a vertical animated slice where the slice is
colored according to gas temperature. Slice files have file names
with extension {\tt .sf}\ and are displayed by selecting the
desired entry from the {\em Load/Unload}\ menu.

All slice files oriented along the same plane (x, y and/or z directions) may be loaded
with one mouse click by choosing the desired orientation from the {\em Slice}
portion of the {\em Load/Unload}\ menu.  These menu entries do not exist in the Multi-Slice menu.
However, when selected from the {\em Slice}\ menu, Smokeview will load all x/y/z oriented multi-slices.


\begin{figure}[bph]
\begin{center}
\begin{tabular}{ccc}
\includegraphics[height=\figheightAbar]{SCRIPT_FIGURES/thouse5_slice_005}&
\includegraphics[height=\figheightAbar]{SCRIPT_FIGURES/thouse5_slice_010}\\
5.0 s&10.0 s\\
\includegraphics[height=\figheightAbar]{SCRIPT_FIGURES/thouse5_slice_030}&
\includegraphics[height=\figheightAbar]{SCRIPT_FIGURES/thouse5_slice_060}&\\
30.0 s&60.0 s
&\raisebox{0.0ex}[0pt]{\includegraphics[height=5.0in]{\SMVfigdir/colorbar_20_620}}\\
\end{tabular}
\caption [Slice file snapshots of shaded temperature contours.]
{Slice file snapshots of shaded temperature contours at various
times in a simulation. These contours were generated by adding
``{\tt \&SLCF PBY=1.5, QUANTITY='TEMPERATURE' /}'' to the FDS
input file. }
\label{figslice}%
\end{center}
\end{figure}

\indent To specify in FDS a vertical slice 1.5~m from the $y=0$
boundary colored by temperature, use the line:
\begin{lstlisting}[basicstyle=\ttfamily]
&SLCF PBY=1.5 QUANTITY='TEMPERATURE' /
\end{lstlisting}
A more complete list of output quantities may be found in
Ref.~\cite{FDS_Users_Guide}.

\paragraph{Vector slices}Animated vectors are displayed using data contained in two or more
slice files.  The direction and length of the vectors are
determined from the $U$, $V$ and/or $W$ velocity slice files. The
vector colors are determined from the file (such as temperature)
selected from the {\em Load/Unload}\ menu. The length of the
vectors can be adjusted by pressing the {\tt `a'}\ key. For cases
with a fine grid, the number of vectors may be overwhelming.
Vectors may be skipped by pressing the {\tt `s'}\ key.  Figure
\ref{figvslice}\ shows a sequence of vector slices corresponding to
the shaded temperature contours found in Fig.~\ref{figslice}.

\begin{figure}[bph]
\begin{center}
\begin{tabular}{ccc}
\includegraphics[height=\figheightAbar]{SCRIPT_FIGURES/thouse5_vslice_005}&
\includegraphics[height=\figheightAbar]{SCRIPT_FIGURES/thouse5_vslice_010}\\
5.0 s&10.0 s\\
\includegraphics[height=\figheightAbar]{SCRIPT_FIGURES/thouse5_vslice_030}&
\includegraphics[height=\figheightAbar]{SCRIPT_FIGURES/thouse5_vslice_060}\\
30.0 s&60.0 s
&\raisebox{0.0ex}[0pt]{\includegraphics[height=5.0in]{\SMVfigdir/colorbar_20_620}}\\
\end{tabular}
\end{center}
\caption [Vector slice file snapshots of shaded vector plots.]
{Vector slice file snapshots of shaded vector plots. These vector
plots were generated by using ``{\tt \&SLCF
PBY=1.5,QUANTITY='TEMPERATURE',VECTOR=.TRUE. /}''.}
\label{figvslice}%
\end{figure}

Similar to slice files, all vector slice files oriented along the same plane (x, y and/or z directions) may be loaded
with one mouse click by choosing the desired orientation from the {\em Vector Slice}
portion of the {\em Load/Unload}\ menu.  These menu entries do not exist in the Vector Multi-Slice menu.
However, when selected from the {\em Vector Slice}\ menu, Smokeview will load all x/y/z oriented multi-slices.

To generate the extra velocity files needed to view vector
animations, add {\tt VECTOR=.TRUE.}\ to the above {\tt \&SLCF}\ line
to obtain:
\begin{lstlisting}
&SLCF PBY=1.50,QUANTITY='TEMPERATURE',VECTOR=.TRUE. /
\end{lstlisting}

When adjacent grids have different resolutions, vector slices may be displayed uniformly
by selecting the {\em uniform spacing}\ checkbox in the Slice file settings dialog box.
This is illustrated in Figure \ref{figvsliceuniformx}.

\begin{figure}[bph]
\begin{center}
\begin{tabular}{c}
 \includegraphics[width=6.5in]{../SMV_Verification_Guide/SCRIPT_FIGURES/vectorskipx_grid}\\
 grid\\
 \includegraphics[width=6.5in]{../SMV_Verification_Guide/SCRIPT_FIGURES/vectorskipx}\\
 original spacing\\
 \includegraphics[width=6.5in]{../SMV_Verification_Guide/SCRIPT_FIGURES/vectorskipx_uniform}\\
 uniform spacing
 \end{tabular}
\end{center}
 \caption[A test showing uniform vector slices where grids are non-uniform in the X direction.]
{A test showing uniform vector slices where grids are non-uniform in the X direction.}
\label{figvsliceuniformx}%
\end{figure}


%%-----------------------------------------------------------------------
\subsection{3D slices}
The user may visualize a 3D region of data using slice files.
To specify a cube of data from 1.0 to 2.0 in each
of the X, Y and Z directions in FDS, use the line:
\begin{lstlisting}
&SLCF XB=1.0,2.0,1.0,2.0,1.0,2.0 QUANTITY='TEMPERATURE' /
\end{lstlisting}

A slice from the resulting slice file
may be moved from one plane to the next just as with Plot3D
files (using left/right, up/down cursor keys or page up/page down
keys).  3D slices and 3D vector slices may also be oriented arbitrarily.
To make these slices visible press the {\tt w}\ key.  Examples of
these slices are illustrated in Figs. \ref{figgslice}\ and \ref{figvgslice}.
These slice may also be oriented in arbitrary positions and directions by
double clicking within the scene.  While holding down the mouse after double clicking,
move the mouse from
side to side or up and down to rotate the general slice.  Double clicking and moving
the mouse vertically while holding down the {\tt ALT}\
key causes the center of rotation for the general slice to move up and down.
Double clicking and moving the mouse horizontally and vertically while holding down
the {\tt SHIFT}\ key causes the center of rotation for the general slice to move along
the {\tt X}\ and {\tt Y}\ axis respectively.

The position and orientation of 3D slices may be manipulated using the Slice motion
portion of the Motion/View/Render dialog box as illustrated
in Fig. \ref{figGSLICE}.

\begin{figure}[bph]
\begin{center}
\begin{tabular}{ccc}
\includegraphics[height=\figheightAbar]{SCRIPT_FIGURES/thouse5_gslice1}&
\includegraphics[height=\figheightAbar]{SCRIPT_FIGURES/thouse5_gslice2}\\
\includegraphics[height=\figheightAbar]{SCRIPT_FIGURES/thouse5_gslice3}&
\includegraphics[height=\figheightAbar]{SCRIPT_FIGURES/thouse5_gslice4}&\\
&&\raisebox{0.0ex}[0pt]{\includegraphics[height=5.0in]{\SMVfigdir/colorbar_20_620}}\\
\end{tabular}
\caption [General oriented temperature slices.]
{
Slices from a 3D temperature slice file at 60~s displayed using four orientations.
3D slices may be re-oriented by double clicking and dragging the mouse
or by changing settings in the Motion/View/Render dialog box.
These images were generated using
``{\tt \&SLCF XB=0.0,6.4,0.0,8.0,0.0,4.8, QUANTITY='TEMPERATURE' /}'' in an FDS
input file. }
\label{figgslice}%
\end{center}
\end{figure}

\begin{figure}[bph]
\begin{center}
\begin{tabular}{ccc}
\includegraphics[height=\figheightAbar]{SCRIPT_FIGURES/thouse5_vgslice1}&
\includegraphics[height=\figheightAbar]{SCRIPT_FIGURES/thouse5_vgslice2}\\
\includegraphics[height=\figheightAbar]{SCRIPT_FIGURES/thouse5_vgslice3}&
\includegraphics[height=\figheightAbar]{SCRIPT_FIGURES/thouse5_vgslice4}&\\
&&\raisebox{0.0ex}[0pt]{\includegraphics[height=5.0in]{\SMVfigdir/colorbar_20_620}}\\
\end{tabular}
\caption [General oriented vector temperature slices.]
{
Vector 3D temperature slices at 60~s displayed using four orientations.
3D vector slices may be re-oriented by double clicking and dragging the mouse
or by changing settings in the Motion/View/Render dialog box.
These images were generated using
``{\tt \&SLCF XB=0.0,6.4,0.0,8.0,0.0,4.8, QUANTITY='TEMPERATURE', VECTOR=.TRUE./}'' in an FDS
input file. }
\label{figvgslice}%
\end{center}
\end{figure}

\begin{figure}[bph]
\centerline{
\includegraphics[width=2.7152777in]{\SMVfigdir/figGSLICE}
}
\caption[Dialog box for controlling the orientation of a 3D slice file.]{Dialog box for controlling the orientation of a 3D slice file.}
\label{figGSLICE}
\end{figure}

%%-----------------------------------------------------------------------
\subsection{Wind Roses}
A wind rose displays a 2D summary of how flow velocities are distributed at a point over some period of time. Data for a wind rose is generated by specifying devices for U, V and W components of velocity using {\tt \&DEVC}\ keywords such as

\begin{verbatim}
&DEVC XYZ=1.2,0.8,1.0 QUANTITY='U-VELOCITY' /
&DEVC XYZ=1.2,0.8,1.0 QUANTITY='V-VELOCITY' /
&DEVC XYZ=1.2,0.8,1.0 QUANTITY='W-VELOCITY' /
\end{verbatim}

\noindent in an FDS input file.  Smokeview creates a two dimensional histogram recording the distribution of wind speeds and direction.
The user can then set various viewing options using the wind rose dialog box illustrated in Figure \ref{figWINDROSE}. Figure \ref{figWINDROSEplots}
illustrates wind roses at several locations drawn in horizontal (xy) and vertical (xz) planes.

\begin{figure}[bph]
\centerline{
\includegraphics[width=2.381944in]{\SMVfigdir/figWINDROSE}
}
\caption[Dialog box for setting wind rose options]
{Dialog box for setting wind rose options}
\label{figWINDROSE}
\end{figure}

\begin{figure}[bph]
\begin{center}
\begin{tabular}{cc}
\includegraphics[width=3.0in]{SCRIPT_FIGURES/windrose_xz}&
\includegraphics[width=3.0in]{SCRIPT_FIGURES/windrose_xy}\\
XZ&XY
\end{tabular}
\end{center}
\caption[Wind roses visualizing velocity flow distributions in XZ and XY planes at several locations]
{Wind roses visualizing velocity flow distributions in XZ and XY planes at several locations}
\label{figWINDROSEplots}
\end{figure}

%%-----------------------------------------------------------------------
%*** fed section
\feda

%%-----------------------------------------------------------------------
\subsection{Duplicate Slices}
FDS outputs duplicate slices whenever a {\tt \&SLCF}\ entry is specified where two meshes coincide.
One set of slices is output for each mesh.
Using the Slice/Duplicates panel of the File/Bounds dialog box, illustrated in Figure \ref{fig:sliceduplicate},
one may specify whether to keep all duplicate slices, keep the finely gridded slices or keep the coarsely gridded slices.
One may similarly specify preferences for vector slice files. By default Smokeview keeps only finely
gridded slices and keeps all vector slices (so one may diagnose possible flow problems).

FDS also outputs duplicate slices when two or more identical {\tt \&SLCF}\ entries are specified in the input ({\tt .fds}) file.  Smokeview ignores these duplicate slices when processing the {\tt .smv} file.

\begin{figure}[bph]
\centerline{
\includegraphics[width=1.8055in]{\SMVfigdir/figsliceduplicate}
}
\caption[Dialog box for specifying duplicate slice visibility.]{Dialog box for specifying duplicate slice visibility.}
\label{fig:sliceduplicate}
\end{figure}


%---------------------------------------------------------------------------------
%---------------------------------------------------------------------------------
\section{2D Shaded Contours on Solid Surfaces - Boundary Files}
\label{section:bf}
Boundary files contain simulation data recorded at blockage or
wall surfaces. Continuously shaded contours are drawn for
quantities such as wall surface temperature, radiative flux, etc.
Figure \ref{figboundary}\ shows several snapshots of a boundary
file animation where the surfaces are colored according to their
temperature. Boundary files have file names with extension {\tt
.bf}\ and are displayed by selecting the desired entry from the
{\em Load/Unload}\  menu. Figure \ref{figtruncboundary}\ shows the
same snapshots as in Fig. \ref{figboundary}\ except that data
below 200~\degC\ is chopped.
\begin{figure}[bph]
\begin{center}
\begin{tabular}{ccc}
\includegraphics[height=\figheightAbar]{SCRIPT_FIGURES/thouse5_bound_005}&
\includegraphics[height=\figheightAbar]{SCRIPT_FIGURES/thouse5_bound_010}\\
5.0 s&10.0 s\\
\includegraphics[height=\figheightAbar]{SCRIPT_FIGURES/thouse5_bound_030}&
\includegraphics[height=\figheightAbar]{SCRIPT_FIGURES/thouse5_bound_060}\\
30.0 s&60.0 s
&\raisebox{0.0ex}[0pt]{\includegraphics[height=5.0in]{\SMVfigdir/colorbar_20_620}}\\
\end{tabular}
\end{center}
\caption [Boundary file snapshots of shaded wall temperatures
contours (cell averaged data).] {Boundary file snapshots of shaded
wall temperatures (cell averaged data). These snapshots were
generated by using ``{\tt\&BNDF QUANTITY='WALL\_TEMPERATURE'/}''.
}
\label{figboundary}%
\end{figure}

\begin{figure}[bph]
\begin{center}
\begin{tabular}{ccc}
\includegraphics[height=\figheightAbar]{SCRIPT_FIGURES/thouse5_bound_trunc_005}&
\includegraphics[height=\figheightAbar]{SCRIPT_FIGURES/thouse5_bound_trunc_010}\\
5.0 s&10.0 s\\
\includegraphics[height=\figheightAbar]{SCRIPT_FIGURES/thouse5_bound_trunc_030}&
\includegraphics[height=\figheightAbar]{SCRIPT_FIGURES/thouse5_bound_trunc_060}\\
30.0 s&60.0 s
&\raisebox{0.0ex}[0pt]{\includegraphics[height=5.0in]{\SMVfigdir/colorbar_20_620}}\\
\end{tabular}
\end{center}
\caption [Boundary file snapshots of truncated shaded wall
temperatures contours (cell averaged data).] {Boundary file
snapshots of truncated shaded wall temperatures (cell averaged
data).  Data values are truncated or chopped below 200~\degC.
These snapshots were generated by using ``{\tt\&BNDF
QUANTITY='WALL\_TEMPERATURE'/}''. }
\label{figtruncboundary}%
\end{figure}

\begin{figure}[bph]
\begin{center}
\begin{tabular}{ccc}
\includegraphics[height=\figheightAbar]{SCRIPT_FIGURES/thouse5_bound_cell_005}&
\includegraphics[height=\figheightAbar]{SCRIPT_FIGURES/thouse5_bound_cell_010}\\
5.0 s&10.0 s\\
\includegraphics[height=\figheightAbar]{SCRIPT_FIGURES/thouse5_bound_cell_030}&
\includegraphics[height=\figheightAbar]{SCRIPT_FIGURES/thouse5_bound_cell_060}\\
30.0 s&60.0 s
&\raisebox{0.0ex}[0pt]{\includegraphics[height=5.0in]{\SMVfigdir/colorbar_20_620}}\\
\end{tabular}
\end{center}
\caption [Boundary file snapshots of shaded wall temperatures
contours (cell centered data).] {Boundary file snapshots of shaded
wall temperatures (cell centered data) These snapshots were
generated by using ``{\tt\&BNDF QUANTITY='WALL\_TEMPERATURE'
CELL\_CENTERED=.TRUE. /}''. }
\label{figboundary_cell_centered}%
\end{figure}
A boundary file containing wall temperature data may be generated
by using:
\begin{lstlisting}
&BNDF QUANTITY='WALL TEMPERATURE' /
\end{lstlisting}
Loading a boundary file is a memory intensive operation.  The
entire boundary file is read in to determine the minimum and
maximum data values.  These bounds are then used to convert four
byte floats to one byte color indices.  To drastically reduce the
memory requirements, simply specify the minimum and maximum data
bounds using the Set Bounds dialog box.  This should be
done before loading the boundary file data.  When this is done,
memory for the boundary file data is allocated for only one time
step rather than for all time steps.

%---------------------------------------------------------------------------------
%---------------------------------------------------------------------------------
\section{3D Contours - Isosurface Files}
\label{section:isosurface}
\begin{figure}[bph]
\begin{center}
\begin{tabular}{cc}
\includegraphics[height=\figheightA]{SCRIPT_FIGURES/thouse5_iso_005}&
\includegraphics[height=\figheightA]{SCRIPT_FIGURES/thouse5_iso_010}\\
5.0 s&10.0 s\\
\includegraphics[height=\figheightA]{SCRIPT_FIGURES/thouse5_iso_030}&
\includegraphics[height=\figheightA]{SCRIPT_FIGURES/thouse5_iso_060}\\
30.0 s&60.0 s
\end{tabular}
\end{center}
\caption [Isosurface file snapshots of temperature levels. ]

{ Isosurface file snapshots of temperature levels. The orange
surface is drawn where the air/smoke temperature is 30~\degC\ and
the white surface is drawn where the air/smoke temperature is
100~\degC. These snapshots were generated by adding ``{\tt\&ISOF
QUANTITY='TEMPERATURE',VALUE(1)=30.0,VALUE(2)=100.0 /}'' to the
FDS input file.}
\label{figiso}%
\end{figure}

\begin{figure}[bph]
\begin{center}
\begin{tabular}{cc}
\includegraphics[height=\figheightA]{SCRIPT_FIGURES/thouse5_tiso_005}&
\includegraphics[height=\figheightA]{SCRIPT_FIGURES/thouse5_tiso_010}\\
5.0 s&10.0 s\\
\includegraphics[height=\figheightA]{SCRIPT_FIGURES/thouse5_tiso_030}&
\includegraphics[height=\figheightA]{SCRIPT_FIGURES/thouse5_tiso_060}\\
30.0 s&60.0 s
\end{tabular}
\end{center}
\caption [Isosurface file snapshots of HRRPUV levels colored by temperature. ]

{ Isosurface file snapshots of HRRPUV levels colored by temperature.}
\label{figtiso}%
\end{figure}

An isosurface is a surface where a quantity such as temperature has the same
value. This surface may also be called a
level surface or 3D contour. Isosurface files contain data
specifying locations for a given quantity at one or
more levels. These surfaces are represented using triangles.
An isosurface file has the extension {\tt .iso}\ and is
displayed by selecting an entry from the {\em Load/Unload>Isosurface}\ menu.

Isosurfaces are specified in an FDS input file using the {\tt
\&ISOF}\ keyword.  To specify an isosurface with constant
30\degC\ and 100\degC\ temperature, as illustrated in Fig. \ref{figiso}, add
the line:
\begin{lstlisting}
&ISOF QUANTITY='TEMPERATURE', VALUE(1:2)=30.0, 100.0 /
\end{lstlisting}
to the FDS input file.
A second quantity may be specified for coloring the isosurface.
For example, to color a constant 600 kw/m2 HRRPUV isosurface by temperature,
illustrated in Figure \ref{figtiso},
use:
\begin{lstlisting}
&ISOF QUANTITY='HRRPUV' , VALUE(1)=600, QUANTITY2='TEMPERATURE' /
\end{lstlisting}
A complete list of isosurface quantities
may be found in Ref.~\cite{FDS_Users_Guide}

%%-----------------------------------------------------------------------
\subsection{Isosurfaces from particle files}
The Smokezip -part2iso option may be used to generate isosurfaces from particle data.
Isosurface locations indicate a boundary separating particle and
no-particle regions, i.e., wherever particle density is 0.5
particles per grid cell.  Isosurface  coloring is determined using
averaged particle data.  Representing particle data with an
isosurface is useful when particles are used to model objects such
as trees especially when the objects are viewed up close.  See
Chapter \ref{ch:smokezip}\ for more details on generating
isosurface files from particle files.  Figure \ref{figisoparticle}
shows a snapshot of a fire plume generated using particles and the command
\begin{lstlisting}
smokezip -part2iso plumeiso
\end{lstlisting}
The plume is visualized using both particles and an isosurface
generated from these same particles.


\begin{figure}[bph]
\begin{center}
\begin{tabular}{cc}
\includegraphics[height=3.75in]{../SMV_Verification_Guide/SCRIPT_FIGURES/plumeiso_prt5_10}&
\includegraphics[height=3.75in]{../SMV_Verification_Guide/SCRIPT_FIGURES/plumeiso_prt5_iso_10}\\
particles at 10.0~s&particle isosurface at 10.0~s\\
\includegraphics[height=3.75in]{../SMV_Verification_Guide/SCRIPT_FIGURES/plumeiso_prt5_30}&
\includegraphics[height=3.75in]{../SMV_Verification_Guide/SCRIPT_FIGURES/plumeiso_prt5_iso_30}\\
particles at 30.0~s&particle isosurface at 30.0~s\\
\end{tabular}
\end{center}
\caption{Fire plume visualized using particles and isosurfaces
generated from  particles.}
\label{figisoparticle}%
\end{figure}

%%-----------------------------------------------------------------------
%*** fed section
\fedb

%---------------------------------------------------------------------------------
%---------------------------------------------------------------------------------
\section{HVAC Networks}

An HVAC network defined in an FDS input file may be visualized  by
selecting the Show/Hide>HVAC menu item.  This feature is available when cases are run with FDS~6.8.0 or later.
The view may be customized using the
HVAC settings dialog box opened by selecting the Dialogs>View>HVAC~settings... menu item.
Portions of the network may be shown or hidden by either selecting a network ID or
by selecting a connection number.  All parts of an HVAC network that are connected
are given the same connection number.  Two portions of an HVAC network
that are isolated are given different connection numbers.  AN HVAC setup may
be debugged using this feature.

Figure \ref{figHVACsettings}\ illustrates the dialog box used to specify HVAC settings.
One my use this dialog box to show or hide various duct components such as fans, dampers
or air coils and show or hide node
components such as filters. This dialog box may also be used to color duct lines or nodes and change the
size of duct line widths or the size of nodes.

Metro mode may be specified to make the HVAC layout easier to view.  It forces duct line directions to be either horizontal or vertical, not diagonal. The nodes may also be
offset to reduce overlap of duct lines.

Figure \ref{figHVACexample} shows a simple example of an HVAC network with a fan, damper , aircoil
and filter.  At $t=0.0$~s these components are inactive.  The fan is not moving and
the damper is open .
At $t=5.0$~s these components are active.  The fan is moving (when viewed with smokeview)
 and the damper is closed, drawn vertically.

\begin{figure}[bph]
\begin{center}
\includegraphics[width=3.7291in]{\SMVfigdir/figHVACsettings}
\end{center}
\caption[HVAC settings dialog box.]{HVAC settings dialog box.
This dialog box allows the user to show and hide various components of an HVAC
network being visualized.}\ \label{figHVACsettings}
\end{figure}


\begin{figure}[bph]
\begin{center}
\begin{tabular}{cc}
\includegraphics[width=3.729166in]{SCRIPT_FIGURES/hvac_comp_00}&
\includegraphics[width=3.729166in]{SCRIPT_FIGURES/hvac_comp_50}\\
0.0~s&5.0~s
\end{tabular}
\end{center}
\caption[HVAC network example.]{HVAC network example.
This example illustrates an HVAC network showing a air coil, damper, fan and filter . The central sphere
changes from green to color when activated.
In adddition, The damper is drawn vertically and the fan rotates when activated.}\ \label{figHVACexample}
\end{figure}


%---------------------------------------------------------------------------------
%---------------------------------------------------------------------------------
\section{Device data - .csv files}
Spreadsheet data, generated by FDS or CFAST or imported from some
other source, may be visualized by Smokeview.
Version 6 of both FDS and CFAST
generate spreadsheet files using a file format given by
\begin{lstlisting}
unit1,unit2, ..., unitN
label1,label2, ..., labelN
data11,data12, ..., data1N
data21,data22, ..., data2N
....
datam1,datam2, ..., dataMN
\end{lstlisting}
where the {\tt unit}\ and {\tt label}\ entries are character strings and the
{\tt data}\ entries are floating point numbers.

FDS uses spreadsheet files to store device and heat release data.
CFAST uses spreadsheet files to store the results of the
simulation (room pressures, layer heights, layer temperatures,
etc.). To view spreadsheet data generated by FDS, open the
Devices/Objects dialog box illustrated in Fig.
\ref{figDEVICES}\ and select the {\em Show values}\ checkbox. If U,
V and/or W velocity data is contained in the spreadsheet file then
velocity vectors may also be displayed. Figures \ref{figdevicevectors} and
\ref{figwindprofile} illustrates velocity visualization using arrows and continuous profiles.


Flow vectors may be visualized as lines, arrows, Smokeview objects or continuous profiles.  The length
and diameter of vector lines may be specified.  In addition, if arrows are selected, the length and diameter of the arrow head may be specified.
The following {\tt \&DEVC}\ lines give an example of defining device flow vectors.

\begin{verbatim}
&DEVC XYZ=3.7,2.0,0.2 QUANTITY='U-VELOCITY' /
&DEVC XYZ=3.7,2.0,0.2 QUANTITY='W-VELOCITY' /
&DEVC XYZ=3.7,2.0,0.2 QUANTITY='TEMPERATURE' /
\end{verbatim}


\begin{figure}[bph]
\begin{center}
\includegraphics[width=2.381944in]{\SMVfigdir/figDEVICES}
\end{center}
\caption{A dialog box for displaying device data values stored in
FDS formatted spreadsheet files.}\ \label{figDEVICES}
\end{figure}


\begin{figure}[bph]
\begin{center}
\begin{tabular}{ccc}
\includegraphics[height=\figheightAbar]{SCRIPT_FIGURES/thouse5_device_005}&
\includegraphics[height=\figheightAbar]{SCRIPT_FIGURES/thouse5_device_010}\\
5.0 s&10.0 s\\
\includegraphics[height=\figheightAbar]{SCRIPT_FIGURES/thouse5_device_030}&
\includegraphics[height=\figheightAbar]{SCRIPT_FIGURES/thouse5_device_060}&\\
30.0 s&60.0 s
&\raisebox{0.0ex}[0pt]{\includegraphics[height=5.0in]{\SMVfigdir/colorbar_20_620}}\\
\end{tabular}
\caption [Visualization of device flow vectors.]
{Visualization of device flow vectors.
Device data may be visualized as colored flow vectors by defining {\tt QUANTITY='U-VELOCITY'},
{\tt 'V-VELOCITY'}\ and/or
{\tt 'W-VELOCITY'}\ keywords
on {\tt \&DEVC}\ namelists each placed at the same {\tt XYZ}\ location.
The flow vectors may be colored by
using {\tt QUANTITY='TEMPERATURE'}\ (or some other quantity).
}
\label{figdevicevectors}%
\end{center}
\end{figure}

\begin{figure}[bph]
\begin{center}
\begin{tabular}{cc}
\includegraphics[height=2.5in]{SCRIPT_FIGURES/wind_test2_arrow}&
\includegraphics[height=2.5in]{SCRIPT_FIGURES/wind_test2_profile}\\
arrows&profile\\

\end{tabular}
\caption [Velocity visualization using arrows and continuous profiles.]
{Velocity visualization using arrows and continuous profiles.}
\label{figwindprofile}%
\end{center}
\end{figure}

%---------------------------------------------------------------------------------
%---------------------------------------------------------------------------------
\section{Static Data - Plot3D Files}\ Data stored in Plot3D files
use a format developed by NASA~\cite{PLOT3D}\ and are used by many
CFD programs for representing simulation results. Plot3D files
store five data values at each grid cell. FDS uses Plot3D files to
store temperature, three components of velocity (U, V, W) and heat
release rate. Other quantities may be stored if desired.

An FDS simulation will automatically  create Plot3D files at
several specified times throughout the simulation. Plot3D data is
visualized in three ways: as 2D contours, vector plots and
isosurfaces. Figure \ref{fig2dcontour}a shows an example of a 2D
Plot3D contour. Vector plots may be viewed if one or more of the
U,V and W velocity components are stored in the Plot3D file. The
vector length and direction show the direction and relative speed
of the fluid flow. The vector colors show a scalar fluid quantity
such as temperature. Figure \ref{figvector2}b shows vectors. The
vector lengths may be adjusted by depressing the ``a'' key. Figure
\ref{fig3dcontour}\ gives an example of isosurfaces. Plot3D data
are stored in files with extension {\tt .q}\ .

\begin{figure}[bph]
\begin{center}
\begin{tabular}{cc}
\includegraphics[height=\figheightA]{SCRIPT_FIGURES/thouse5_plot3d_val}
&\includegraphics[height=\figheightA]{SCRIPT_FIGURES/thouse5_plot3d_vec}\\
a)

\parbox[t]{2.5in}{shaded 2D temperature contour plots in a vertical plane through the fire}
& b)
\parbox[t]{2.5in}{shaded temperature vector plot in a vertical plane through the fire.
The ``a'' key may be depressed to alter the vector sizes. The
``s'' key may be depressed to alter the number of vectors
displayed. }
\end{tabular}
\end{center}
\caption{Plot3D contour and vector plot examples.  }
\label{fig2dcontour}%
\label{figvector2}
\end{figure}

\begin{figure}[bph]
\begin{center}
\begin{tabular}{cc}
\includegraphics[height=\figheightA]{SCRIPT_FIGURES/thouse5_plot3d_iso1}
&\includegraphics[height=\figheightA]{SCRIPT_FIGURES/thouse5_plot3d_iso2}\\
a) temperature isosurface at 350 \degC&b) temperature isosurface
at 530 \degC
\end{tabular}
\end{center}
\caption{Plot3D isocontour example.}
\label{fig3dcontour}%
\end{figure}

%---------------------------------------------------------------------------------
%------------------------ Visualizing Zone Fire Data -----------------------------
%---------------------------------------------------------------------------------

\chapter{Visualizing Zone Fire Data}
Smokeview may be used to visualize data simulated by a zone fire
model. The zone fire model, CFAST~\cite{CFAST_Tech_Guide_7}, creates data
files containing geometric information such as room dimensions and
orientation, vent locations, etc.  It also outputs modeling
quantities such as pressure, layer interface heights, and lower
and upper layer temperatures. Smokeview visualizes the geometric
layout of the scenario.  It also visualizes the layer interface
heights, upper layer temperature and vent flow. Vent flow is
computed internally in Smokeview using the same equations and data
as used by CFAST.   For a given room, pressures , $P_i$, are
computed at a number of elevations, $h_i$ using
\begin{eqnarray}
P_i=P_f - \rho_L g \min(h_i,y_L) - \rho_U g \max(h_i-y_L,0)
\end{eqnarray}
where $P_f$ is the pressure at the floor (relative to ambient),
$\rho_L$ and $\rho_U$ are the lower and upper layer densities
computed from layer temperatures using the ideal gas law and $g$
is the acceleration of gravity.  When densities vary continuously
with height, this becomes $P_i=P_f-\int_0^h \rho(z)g\,\mbox{d}z$. A
pressure difference profile is then determined using pressures
computed on both sides of the given vent.

In the visualization, colors represent the gas temperature of the
vent flow.  The colors change because the flow may come from
either the lower (cooler) or upper (hotter) layer.   The length
and direction of the colored vent flow region represents a vent
flow speed and direction.  Plumes are represented as inverted
cones with heights calculated in Smokeview using the same
correlation as CFAST and heat release rate data computed by CFAST.
A Smokeview view of the one room sample case that comes with the
CFAST installation is illustrated in Figs. \ref{figcfast}, \ref{figcfastsmoke}
and \ref{figcfastboundary}.

\begin{figure}[bph]
\begin{center}
\begin{tabular}{ccc}
\includegraphics[width=3.00in]{SCRIPT_FIGURES/cfast_test_c1_100}&
\includegraphics[width=3.00in]{SCRIPT_FIGURES/cfast_test_c1_200}\\
100.0 s&200.0 s\\
\includegraphics[width=3.00in]{SCRIPT_FIGURES/cfast_test_c1_300}&
\includegraphics[width=3.00in]{SCRIPT_FIGURES/cfast_test_c1_400}\\
300.0 s&400.0 s
&\raisebox{0.0ex}[0pt]{\includegraphics[height=5.0in]{\SMVfigdir/colorbar_20_620}}\\
\\
\end{tabular}
\end{center}
\caption{CFAST test showing upper/lower layer temperatures and vent flow.}
\label{figcfast}%
\end{figure}

\begin{figure}[bph]
\begin{center}
\begin{tabular}{cc}
\includegraphics[width=3.25in]{SCRIPT_FIGURES/cfast_test_smoke_100}&
\includegraphics[width=3.25in]{SCRIPT_FIGURES/cfast_test_smoke_200}\\
100.0 s&200.0 s\\
\includegraphics[width=3.25in]{SCRIPT_FIGURES/cfast_test_smoke_300}&
\includegraphics[width=3.25in]{SCRIPT_FIGURES/cfast_test_smoke_400}\\
300.0 s&400.0 s\\
\end{tabular}
\end{center}
\caption{CFAST test showing upper/lower layer temperatures and vent flow.
Layers are visualized realistically and vent flow
is visualized using color.}
\label{figcfastsmoke}%
\end{figure}

\begin{figure}[bph]
\begin{center}
\begin{tabular}{ccc}
\includegraphics[width=3.00in]{SCRIPT_FIGURES/cfast_test_boundary_100}&
\includegraphics[width=3.00in]{SCRIPT_FIGURES/cfast_test_boundary_200}\\
100.0 s&200.0 s\\
\includegraphics[width=3.00in]{SCRIPT_FIGURES/cfast_test_boundary_300}&
\includegraphics[width=3.00in]{SCRIPT_FIGURES/cfast_test_boundary_400}\\
300.0 s&400.0 s
&\raisebox{0.0ex}[0pt]{\includegraphics[height=5.0in]{\SMVfigdir/colorbar_20_620}}\\
\\
\end{tabular}
\end{center}
\caption{CFAST test showing wall temperatures.}
\label{figcfastboundary}%
\end{figure}

%---------------------------------------------------------------------------------
%------------------------ Visualizing Data Using 2D Plots ------------------------
%---------------------------------------------------------------------------------

\chapter{Visualizing Data Using 2D Plots}
\label{chap:2DPLOT}

Smokeview may be used to create 2D plots using data found in spreadsheet (csv) files
or in slice files.
Spreadsheet files may be generated by FDS, by CFAST or  obtained from an experiment as long as the
format is the same as generated by FDS or CFAST.  Smokeview expects the first row to
contain quantities, the second row to contain units and the third row through last row to
contain data.
%---------------------------------------------------------------------------------
%---------------------------------------------------------------------------------
\section{Spreadsheet file 2D Plots}
Two examples of spreadsheet files generated by FDS are
{\tt casename\_hrr.csv}\ and {\tt casename\_devc.csv}\ where casename is the
{\tt CHID}\ of the case being modeled.
The file {\tt casename\_hrr.csv}\ consists of data related
to heat output such as heat release rate (HRR),
radiative output (Q\_RAD) or convective output
(Q\_CONV) .  The file {\tt casename\_devc.csv}\ consists of data quantities specified on \&DEVC lines
in the FDS input file
such as U-VEL, V-VEL, W-VEL ($U$, $V$, $W$ components of velocity) or TEMP (temperature).
Using the 2D Plot dialog box, illustrated in Figure \ref{fig:plot2ddialog},
one can select data columns from either of these two or other csv files
generated by FDS. Plots may contain up to two different curve types
where all curves of a given type have the same unit.
Labels for the first curve type picked are drawn on the left axis
and labels for the second curve type picked are drawn on the right axis.

Figure \ref{fig:plot2ddialog}
shows the dialog box used for creating 2D plots.  This dialog box may be opened using the `"` keyboard
shortcut or by selecting the {\em Dialogs>Data>2D~Plots}\ menu item.
The general procedure for creating a plot is to
click on the New Plot button followed by selecting one or more curves from one of the csv files listed.
The plot may
then be customized by showing or hiding various labels, setting line widths or color {\em etc.}..  In more detail,
the steps for generating a plot are:

\begin{figure}[htbp]
\begin{center}
\begin{tabular}{c}
\includegraphics[width=4.3403in]{../../../fig/smv/figures/plot2d_dialog}
\end{tabular}
\end{center}
\caption[2D Plotting Dialog Box.]{2D Plotting Dialog Box. To create a plot, click on the New plot
button, then select one or more curves, then customize how the plot appears
by setting various plot and curve property options.}
\label{fig:plot2ddialog}%
\end{figure}

\begin{enumerate}

\item Click on the \frameit{New plot} button in the \frameit{add/remove/select} panel.
You may also delete a plot or select a plot if more than one plot has been created.

\item Select the csv file type in the \frameit{add curves} panel. For example select devc
for device data, hrr for heat release data data or step for cpu and time step data.

\item Select one or more curves to add to the plot using the \frameit{select curve} list box.
The number of curves in this list box may be reduced by selecting a unit (m, m/s, kW {\em etc.}).
Only curves then with that unit will be listed.
A plot may consist of curves with at most two units.

\end{enumerate}

Plots for all devices may be added at once by selecting a quantity in the \frameit{Multiple devc plots}\ panel.
The plots that are generated are placed at the corresponding device locations. These plots may be removed by selecting
a quantity from the Remove listbox in the same panel.

Plot properties are applied in the \frameit{plot properties}\ panel.
The property plot position and plot title only apply to the currently selected plot.
All other plot properties such as x and y axis label visibility, size factor or smoothing interval
apply to all plots and curves in the case of smoothing interval.  Curve properties such as color or line width may be specified in the \frameit{curve properties} panel.  Plot positions may be distributed between two locations (x0,y0,z0) and (x1,y1,z1).
Curves may also be scaled.  For example one could scale a mass loss curve by a heat of combustion value
to compare it with a heat release rate curve.  One may also remove curves using buttons in this panel.

Smokeview displays data curves by computing bounds for all csv data columns and uses these bounds to scale curves.  These bounds may be overridden by specifying plot bounds in the
\frameit{plot bounds} panel.

Figure \ref{fig:plot2d_hrrplot} gives an example of a 2D plot.  The plot shows
heat release rate, Q\_RADII and
a propane mass loss curve.  The HRR curve is colored black, the Q\_RADI curve is colored red and the propane mass loss rate is colored blue.
The curves are smoothed by selecting a smoothing parameter of 2~s.

\begin{figure}[htbp]
\begin{center}
\begin{tabular}{c}
\includegraphics[width=6.50in]{../../../fig/smv/figures/thouse5_plot2d_030}
\end{tabular}
\end{center}
\caption{2D HRR plot.}
\label{fig:plot2d_hrrplot}%
\end{figure}

%---------------------------------------------------------------------------------
%---------------------------------------------------------------------------------
\section{Slice file 2D Plots}
2D plots may also be created using slice file data by selecting the
{\em Dialogs>File/Data/Coloring}\ menu entry then selecting the
{\em Slice/2D Plots}\ tab. This dialog box is illustrated in
Figure \ref{fig:plot2dslicedialog}. To create a plot,
load the desired slice file then set the position within the slice where
slice file data is retrieved.  One may also
time average the data before plotting and/or select a region
over which slice file data is averaged.
Figure \ref{fig:plot2d_sliceplot} shows an example of a 2D plot
generated from slice file data.


\begin{figure}[htbp]
\begin{center}
\begin{tabular}{c}
\includegraphics[width=4.00in]{../../../fig/smv/figures/plot2d_slice_dialog}
\end{tabular}
\end{center}
\caption[Slice file 2D Plot Dialog Box.]{Slice file 2D Plot Dialog Box.
To create a plot, load a slice file, select an x, y, z position with the slice,
then check the show plot checkbox.}
\label{fig:plot2dslicedialog}%
\end{figure}

\begin{figure}[htbp]
\begin{center}
\begin{tabular}{c}
\includegraphics[width=6.50in]{../../../fig/smv/figures/thouse5_sliceplot2d_030}
\end{tabular}
\end{center}
\caption{2D Slice File plot. A 2D plot of slice file data is was created
by loading a temperature slice and setting the position to $x=0.5$, $z=1.5$ .}
\label{fig:plot2d_sliceplot}%
\end{figure}

%---------------------------------------------------------------------------------
%------------------------ Controlling and Customizing Smokeview --------------------
%---------------------------------------------------------------------------------

\part{Controlling and Customizing Smokeview}

\chapter{Setting Options}
\label{chapter:settingoptions}
%---------------------------------------------------------------------------------
%---------------------------------------------------------------------------------
\section{Loading Data}

Data is loaded in smokeview by clicking the right mouse button and selecting one of the data types
under the Load/Unload menu. Smokeview loads data for all available times and meshes by default.
For large cases, it is useful to decrease the amount of data loaded in order to reduce load times .
To do this, use the Loading options tab of the Files/Data/Coloring dialog box
illustrated in Figure \ref{figDataloading}.
To open this dialog box select the Dialogs>File/Data/Coloring menu entry then select the Loading options tab.
To reduce the loading time interval, specify a min and/or max time.
To reduce the number of meshes over which
data is loaded specify an intersection box.
The intersection box may be viewed by selecting the show intersection box checkbox.
The intersection box is colored red. The meshes that are contained within this box
are drawn as outlines and colored black. Mesh indices (1 to number of meshes) may be displayed and used
to set the intersection box to that mesh.
Only data in meshes within the intersection box will
then be loaded. Meshes can also be selected directly, allowing one to load data on a disjoint set of meshes.
If the `Load a file only if unloaded` checkbox is selected and if the intersection box
is expanded to include more meshes then only data in these extra meshes where data was not loaded before will be loaded
(data will not be loaded twice).

\begin{figure}[bph]
\centerline{
\includegraphics[width=4.0in]{\SMVfigdir/figDataloading}}
\caption[Dialog box for limiting data loaded by time or space.]
{Dialog box for limiting data loaded by time or space.
Data may be loaded within specified time bounds.
Data may also be loaded from meshes within a specified box or from specified meshes.
}
\label{figDataloading}
\end{figure}

%---------------------------------------------------------------------------------
%---------------------------------------------------------------------------------
\section{Data Bounds}

Smokeview visualizes data by mapping data values to color indices
ranging from 0 to 255 using a mapping of the form

\begin{eqnarray*}
c=255 \frac{v-v_{\rm min}}{v_{\rm max}-v_{\rm min}}
\end{eqnarray*}

\noindent where $v$ is a data value and c is a colorbar index.
Value less than $v_{\rm min}$ are mapped to 0 and
values greater than $v_{\rm max}$ are mapped to 255.
The parameters $v_{\rm min}$ and $v_{\rm max}$ are set to be either
\begin{itemize}
\item a min/max values specified by the user,
\item a global min/max of loaded data values,
\item a global min/max of all data values or
\item percentile min/max values (1st and 99th percentile values).
\end{itemize}
This choice is set in the Data bounds dialog box as illustrated in Figure \ref{figBOUNDSset}.
Creating and modifying colorbars are discussed in Chapter \ref{chap:colorbar}.
User specified min/max values may be used to ensure consistent color shading when
displaying several data files simultaneously.
To quickly set bounds for actual data, press {\tt ALT r}\ and reload or update data files.
This puts Smokeview into {\em research mode}\ which uses actual global min/max bounds
when mapping data to color indices.

\begin{figure}[bph]
\centerline{
\includegraphics[width=4.5in]{\SMVfigdir/figBOUNDset}}
\caption[Dialog box for setting Slice file data bounds.]
{Dialog box for setting Slice file data bounds.
Select a variable and bound type (set, global  or percentile).  Enter a lower
and/or upper bound if set bound type was selected. Data may be excluded from the plot by
selecting a {\em Truncate data}\ bound.
}\ \label{figBOUNDSset}
\end{figure}


The Data bounds dialog box is opened from the {\em
Dialogs>Data bounds}\ menu. Each file type in Fig. \ref{figBOUNDSset}\
(slice, particle, Plot3D, etc.) has a set of {\em radio buttons}\
for selecting the variable type and radio buttons for bounding and truncating
data when converting data to color. Variable types are determined from the files generated
by FDS and are automatically recorded in the {\tt .smv}\ file. The
data bounds are set in a pair of edit boxes. Radio buttons
adjacent to the edit boxes determine what type of bounds should be
applied.
The \frameit{Update Colors}\ and \frameit{Reload Data} buttons
as the names suggest allow one to update colors (without re-reading data files) or reload data.

\begin{figure}[bph]
\begin{center}
\begin{tabular}{ccc}
\includegraphics[height=\figheightAbar]{SCRIPT_FIGURES/thouse5_cjet_005}&
\includegraphics[height=\figheightAbar]{SCRIPT_FIGURES/thouse5_cjet_010}\\
5.0 s&10.0 s\\
\includegraphics[height=\figheightAbar]{SCRIPT_FIGURES/thouse5_cjet_030}&
\includegraphics[height=\figheightAbar]{SCRIPT_FIGURES/thouse5_cjet_060}&\\
30.0 s&60.0 s
&\raisebox{0.0ex}[0pt]{\includegraphics[height=5.0in]{\SMVfigdir/colorbar_20_620}}\\
\end{tabular}
\caption [Ceiling Jet Visualization.] {   Ceiling jet
visualization created by {\em chopping data}\ below 360~$^\circ$C
using the Bounds dialog box as illustrated in
Fig.~\ref{figBOUNDSset}. }
\label{figceilingjet}%
\end{center}
\end{figure}

The File/Bounds dialog box has additional controls used to chop or
hide data. The settings used in Fig. \ref{figBOUNDSset}\ were
used to generate the ceiling jet visualized in
Fig.~\ref{figceilingjet}. Data values less than 360~\degC\ are
chopped or not drawn.

Slice file data may be time averaged or smoothed over a user
selectable time interval.  This option is also implemented from
the Slice File section of the File/Bounds dialog box (see
Fig. \ref{figBOUNDSset}).

\begin{figure}[bph]
\centerline{
\includegraphics[width=5.53472222in]{\SMVfigdir/figBOUNDSplot3d}
}\ \caption[Dialog box for setting Plot3D file
options.] {Dialog box for setting Plot3D file
options. Select a variable and bound type. Enter a lower
and/or upper bound if set bound type was selected.  Data may be excluded from the plot by
selecting a {\em Truncate data}\ bound. Select the type of contour
plot to be displayed.}\ \label{figBOUNDSplot3d}
\end{figure}

The {\tt Plot3D}\ portion of the File/Bounds dialog box as illustrated in Figure \ref{figBOUNDSplot3d} has controls for specifying how Plot3D vectors and isosurfaces appear.

The bounds dialog for Plot3D display allows one to select between
three different types of contour plots:  shaded, stepped and line
contours.

The {\tt Boundary File}\ portion of the File/Bounds dialog
box has an {\em Ignition}\ checkbox which allows one to visualize
when and where the blockage temperature exceeds its ignition
temperature.

The {\tt Particle file}\ portion of the File/bounds dialog box
as illustrated in Figure \ref{figBOUNDSpart}\
has controls for specifying whether particle files are loaded in parallel.
The user may also specify the number files that are loaded simultaneously

\begin{figure}[bph]
\centerline{
\includegraphics[width=5.5345in]{\SMVfigdir/figBOUNDSpart}
}\ \caption[Dialog box for setting Particle file
options.] {Dialog box for setting Particle file
options. Select a variable and bound type. Enter a lower
and/or upper bound if set bound type was selected. Particle files may be loaded in parallel by
selecting the Fast loading checkbox and setting the number of files to load in parallel.
 }\ \label{figBOUNDSpart}
\end{figure}


%---------------------------------------------------------------------------------
%---------------------------------------------------------------------------------
\section{3D Smoke Options}
\begin{figure}[bph]
\centerline{\includegraphics[width=7.195in]{\SMVfigdir/fig3DSmoke}
}\ \caption[Dialog box for setting slice rendered 3D smoke options]
{Dialog box for setting slice rendered 3D smoke options.  Fire color may be specified for hrrpuv values above a specified cutoff.
Smoke color may be specified for hrrpuv values below the same cutoff. }
\label{fig3DSmoke}
\end{figure}
\begin{figure}[bph]
\centerline{\includegraphics[width=2.9791666in]{\SMVfigdir/fig3DSmokeB}
}\ \caption[Dialog box for setting volume rendered 3D smoke
options] {Dialog box for setting volume rendered 3D smoke options.
Fire color may be specified for temperature values above a specified cutoff.
Smoke color may be specified for hrrpuv values below the same cutoff.
}\ \label{fig3DSmokeB}
\end{figure}
Figures \ref{fig3DSmoke}\ and Figure \ref{fig3DSmokeB}\ show a dialog box for controlling the display of
slice and volume rendered smoke.
The user may specify parameters such as fire and smoke color, smoke albedo and an hrrpuv cutoff value used
to determine what is colored as smoke and fire.  The user may also specify cutoff values used to determine when to load smoke
and fire data files.
Red, green and blue color values range between 0 and 255.
The {\em hrrpuv cutoff}\ parameter refers to
the heat release rate required before Smokeview will
color a node as fire rather than smoke. The {\em 50\% flame
depth}\ allows one to specify the transparency or optical thickness
of the fire (for visualization purposes only). A small value
results in opaquely drawn fire while a large value results in a
transparently drawn fire. The {\em Absorption Parameter}\ setting
refers to how the smoke slices are drawn.  The {\em adjust
off-center}\ setting causes Smokeview to account for non-axis
aligned paths. The {\em adjust off-center + zero at boundary}
accounts for off center path lengths and zeros smoke density at
boundaries in order to remove graphical artifacts.

%---------------------------------------------------------------------------------
%---------------------------------------------------------------------------------
\section{Plot3D Viewing Options}\ Plot3D files are more
complicated to visualize than time dependent files such as
particle, slice or boundary files. For example, only the
transparency and color characteristics of a time file may be
changed. With Plot3D files however, many attributes may be
changed. One may view 2D contours along the {\tt X}, {\tt Y}\
and/or {\tt Z}\ axis of up to six\footnote{ The FDS software stores
temperature, three components of velocity (denoted $u$, $v$ and
$w$) and heat release per unit volume.  If at least one velocity
component is stored in a Plot3D file, then Smokeview adds speed to
the Plot3D variable list.}\ different simulated quantities, view
flow vectors and iso or 3D contours. Plot3D file visualization is
initiated by selecting the desired entry from the {\em
Load/Unload}\ {\tt Plot3D}\ sub-menu and as with time files one may
change color and transparency characteristics.

%%-----------------------------------------------------------------------
\subsection{2D contours}
Smokeview displays a 2D contour slice midway along the {\tt Y}\
axis by default when a Plot3D file is first loaded, To step the
contour slice up by one grid cell along the {\tt Y}\ axis, depress
the {\em\tt space bar}. Similarly to step the contour slice down
by one grid cell along the {\tt Y}\ axis, depress the ``{\tt -}''
key. To view a contour along either the {\tt X}\ or {\tt Z}\ axis,
depress the {\tt x}\ or {\tt z}\ keys respectively.  Depressing the
{\tt x}, {\tt y}\ or {\tt z}\ keys while the contour is visible will
cause it to be hidden. The Plot3D variable viewed may be changed
by either depressing the ``{\tt p}'' key or by selecting the {\em
Solution Variable}\ sub-menu of the {\em Show/Hide}\ menu.

%%-----------------------------------------------------------------------
\subsection{Iso-Contours}Iso-contours also called 3D contours or level surfaces may be
viewed by depressing the ``{\tt i}\ key or by selecting the {\tt
Plot3D>3D Contours}\ sub-menu of the {\em Show/Hide}\ menu.

%%-----------------------------------------------------------------------
\subsection{Flow vectors}If at least one velocity component is present in the Plot3D
file then the ``{\tt v}'' key may be depressed in order to view
flow  vectors. The length and direction of the vector indicates
the flow direction and speed. The vector color indicates the value
of the currently displayed quantity. A small dot is drawn at the
end of the line to indicate flow direction. The vector lengths as
drawn may be changed by depressing the ``{\tt a}'' key. Vector
plots may be very dense when the grid is finely meshed. The ``{\tt
s}'' key may be depressed in order to skip vectors.  For example,
all vectors are displayed by default.  If the ``{\tt s}'' is
depressed then every other vector is skipped.

%---------------------------------------------------------------------------------
%---------------------------------------------------------------------------------
\section{Display Options}
%%-----------------------------------------------------------------------
\subsection{General}
\begin{figure}[bph]
\centerline{\includegraphics[width=3.847222in]{\SMVfigdir/figProperties}
}\ \caption [Dialog box for setting miscellaneous Smokeview scene
properties.] {Dialog box for setting miscellaneous Smokeview scene
properties.}\ \label{figProperties}
\end{figure}
The Display dialog box, illustrated in Fig.
\ref{figProperties}, allows one to set various options to control
the scene display such as toggling the visibility of the colorbar, timebar, title {\em etc.}.  The  dialog box may be
invoked by selecting the {\em Dialogs$>$Display}\ menu item.

%%-----------------------------------------------------------------------
\subsection{Setting window parameters}
Controls in the {\em Window Properties}\ region of the Motion/View/Render dialog box as illustrated in Fig. \ref{figMOTIONwindow},
allow one to change the scene magnification or zoom factor, the
projection method used to draw objects (perspective or size preserving) and the window size.
Perspective and size preserving projections differ in how objects are displayed
at a distance.  A perspective projection for-shortens or draws an
object smaller when at a distance. An isometric or size preserving projection
on the other hand draws objects the same size regardless of
where it occurs in the scene.

The {\em zoom}\ and {\em aperture}\ edit boxes allow one to change
the magnification of the scene or equivalently the angle of view
across the scene.  The relation between these two parameters is
given by
\begin{eqnarray}
\mbox{zoom}=\tan(45^\circ/2)/\tan(\mbox{aperture}/2)
\end{eqnarray}
A default aperture of $45^\circ$ is chosen so that Smokeview
scenes have a normal perspective.

The size selection list gives the user several pre-defined choices for changing window size or one may alter
the width and height spinners to construct a window with a custom size.

\begin{figure}[bph]
\centerline{
\begin{tabular}{cc}
\includegraphics[width=2.3194444in]{\SMVfigdir/figMOTION3}
\end{tabular}
}\ \caption[Dialog box for specifying window properties.]{Dialog box for specifying window properties.
 The Windows portion of the Motion/View/Render dialog box allows one to set the window size, projection type (perspective or size preserving) and zoom level.}\ \label{figMOTIONwindow}
\end{figure}

%%-----------------------------------------------------------------------
\subsection{Scaling Scenes}
Controls in the {\em Scaling Depth}\ portion of the Motion/View dialog box, as
illustrated in Fig. \ref{figMOTIONscale}, allow one to scale the
Smokeview scene.  The x, y and z scene dimensions may be scaled independently.  For example, a tunnel scenario could be scaled to make the tunnel's {\em long}\ dimension
appear the same size on the screen as the {\em height}\ dimension. The near and far depth planes are used by OpenGL for setting up the depth buffer which in turn is used for determining
when objects hidden by {\em closer}\ objects.
\begin{figure}[bph]
\centerline{
\begin{tabular}{cc}
\includegraphics[width=2.3194444in]{\SMVfigdir/figMOTION6}
\end{tabular}
}\ \caption[Dialog box for setting scaling and depth parameters.]{Dialog box for setting scaling and depth parameters.
The Scaling/Depth portion of the Motion/View/Render dialog box allows one to specify scaling parameters for x, y and z scene dimensions and
to specify the near and far depth planes.
}\ \label{figMOTIONscale}
\end{figure}

%%-----------------------------------------------------------------------
\subsection{Stereo}
\label{section:stereo}\ Smokeview implements several methods for displaying scenes in stereo or 3D. These methods are temporal (sending odd frames to the left eye and even frames to right eye), spatial (drawing two frames side by side) and color (super imposing red/blue or magenta/cyan frames).  Each method then creates two versions of the scene, one version for each eye. Figure \ref{figstereodialog}\ shows the dialog box used to configure this option.  The {\em shuttered checkbox}\ is enabled if the {\em -stereo}\ command line
option is used when invoking Smokeview and the video card supports shuttered stereo display.

\begin{figure}[bph]
\begin{center}
\includegraphics[width=2.9305in]{\SMVfigdir/figSTEREO}
\caption{Dialog box for specifying stereo view options.}
\label{figstereodialog}
\end{center}
\end{figure}

The first method, denoted sequential stereo, works by displaying
images for the left and right eye alternately in time.  Shuttered
glasses  synchronized with the monitor are used to ensure that
only the left eye sees the left image and only the right eye sees
the right image.  A monitor displaying this type of stereo should
have a refresh rate of at least 120 frames per second (60 frames
per second for each eye) otherwise flickering is noticeable.
Unfortunately, most of today's LCD flat panel monitors typically
do not have refresh rates faster than 60 to 80 frames per second.
This method (for Smokeview) requires a video card that supports
OpenGL {\em QUAD buffering}. This Smokeview stereo option may be
enabled from the command line by using the {\tt -stereo}\ option.

The second method, denoted left/right stereo, displays the two
images side by side.  With practice, one can merge both images
without requiring specialized glasses (though they are available
if desired) especially if the images are small and not separated
by a large angle. A trick for seeing the stereo effect is to place
a finger from each hand in the center of each picture.  Then relax
your eyes while trying to {\em merge}\ your two fingers together.
Figure \ref{figlrstereo}\ show an example of the left/right method
for generating a stereo image.  This method can generate full
colored images and requires no equipment (for most people) to view
but results in smaller images.
\begin{figure}[bph]
\begin{center}
\includegraphics[height=5.0in]{SCRIPT_FIGURES/thouse5_lr_stereo}
\caption[Stereo pair view of a townhouse kitchen fire.]{ Stereo
pair view of a townhouse kitchen fire. To aid in viewing the
stereo effect, place a finger in front of each image.  Relax your
eyes allowing your two fingers and stereo pair images to merge
into one. }\ \label{figlrstereo}
\end{center}
\end{figure}

The third method, uses color to separate left and right images.
One method denoted red/blue stereo, displays red and blue versions
of each image.  Glasses with a red left lens and a blue right lens
are required to view the image.  As with the shuttered glasses for
sequential stereo, the colored glasses {\em separate}\ the images
enabling each eye to see only one image.  Red/blue colored glasses
may be obtained inexpensively. They also may be made using
red and blue cellophane or by coloring clear plastic with read and
blue marking pens.  Figure \ref{figrbstereo}\ uses the red/blue
method for generating a stereo image.  This method generates full
size images, requires only inexpensive glasses to view but can
only display monochrome images. The red/cyan method for displaying
stereo images works similarly to the red/blue method.  The main
difference is that since cyan is the made up of green and blue
(the {\em opposite}\ in some sense of red), the combination of red
and cyan lenses allow all colors to pass to your eyes.

Figures \ref{figrcstereo}\ uses the red/cyan method for generating
a stereo image.  As with red/blue, this method generates full size
images.  This method allows Smokeview
scenes to be displayed in full color.

\begin{figure}[bph]
\begin{center}
\includegraphics[width=4.5in]{SCRIPT_FIGURES/thouse5_iso_rb_stereo}
\caption[Red/blue stereo pair view of a townhouse kitchen fire.]{
Red/blue stereo pair view of a townhouse kitchen fire. Red/blue
glasses are required to see the 3D stereo effect. }
\label{figrbstereo}
\end{center}
\end{figure}

\begin{figure}[bph]
\begin{center}
\includegraphics[width=4.5in]{SCRIPT_FIGURES/thouse5_iso_rc_stereo}
\caption[Red/cyan stereo pair view of a townhouse kitchen fire.]{
Red/cyan stereo pair view of a townhouse kitchen fire. Red/cyan
glasses are required to see the 3D stereo effect. }
\label{figrcstereo}
\end{center}
\end{figure}

%---------------------------------------------------------------------------------
%---------------------------------------------------------------------------------
\section{Rendering Scenes - Creating Image Files}
The {\em Render}\ portion of the Motion/View dialog box,
as illustrated in Fig. \ref{figMOTIONrender}, is used to convert a {\em Smokeview scene}\ to one or more image files.  Image file formats may be either PNG or JPEG. One frame is rendered for each time step in a time dependent files unless a skipping parameter is specified.
A skipping interval may be selected to generate fewer images using the {\em Where frames}\ control.  One frame is rendered for static files.  Higher resolution images may be generated by selecting a multiplier factor using the {\em Resolution multiplier}\ control. For example, a factor of 3 generates an image with 3 times the original scene resolution.  Unwanted portions of a scene may be removed or clipped before it is rendered by specifying a clipping region. A clipping region is specified in terms of left, right, bottom and top pixel locations.

\begin{figure}[bph]
\begin{center}
\includegraphics[width=2.3194in]{\SMVfigdir/figRENDER}
\caption[Dialog box for creating images of the Smokeview scene.]
{Dialog box for creating images of the Smokeview scene. The Render portion of the Motion/View/Render dialog box allows one to create images of the Smokeview scene. One may also clip or crop the rendered image.}
\label{figMOTIONrender}
\end{center}
\end{figure}

%---------------------------------------------------------------------------------
%---------------------------------------------------------------------------------
\section{Setting Viewpoints}
Controls in the {\em Viewpoint}\ portion of the Motion/View dialog box, as
illustrated in Fig. \ref{figMOTIONviewpoints}, allow one to save the position and orientation of the scene.  These saved position/orientations are called viewpoints.  Viewpoints may then be selected resulting in the scene returning to a previously saved position and orientation.

Six default viewpoints are created by Smokeview labeled XMIN, XMAX, YMIN, YMAX, ZMIN and ZMAX.  These viewpoints show the scene from the near and far X, Y and Z positions looking towards the center of the scene.  To define a user viewpoint, manipulate the scene to the desired position and orientation. Then press
{\em Add}\ button.  This adds the new viewpoint to a list of available viewpoints.
The {\em Replace}\ button replaces the currently active viewpoint (as named by the {\em Select}\ list item) with the current position and orientation.

To change the view to a currently saved
viewpoint, use the {\em Select}\ listbox to select the desired
viewpoint. The {\em Cycle Default}\ , {\em Cycle User}\ and {\em Cycle All}\ buttons are used to cycle through viewpoints. The {\em Delete}\ button, as one would expect, removes the
viewpoint.  The {\em Edit}\ text box is used to change the name of the currently  selected viewpoint.
The {\em view at startup}\ button is used to specify the
viewpoint that should be active when Smokeview starts up.



\begin{figure}[bph]
\centerline{
\begin{tabular}{cc}
\includegraphics[width=3.5in]{\SMVfigdir/figMOTION5}
\end{tabular}
}\ \caption[Dialog box for specifying scene viewpoints.]{Dialog box for specifying scene viewpoints. The viewpoint portion of the  Motion/View/Render dialog box allows one to define and control viewpoints. }\ \label{figMOTIONviewpoints}
\end{figure}

%---------------------------------------------------------------------------------
%---------------------------------------------------------------------------------
\section{Clipping Scenes}
\label{section:clipping}

\begin{figure}[bph]
\begin{center}
\includegraphics[width=4.48111in]{\SMVfigdir/figCLIP}
\end{center}
\caption[Clipping dialog box.]{Clipping dialog box.
Minimum and maximum clip plane values are set for X, Y and Z planes.
When clipping, one may clip data, geometry or both. }
\label{figCLIP}
\end{figure}

\begin{figure}[bph]
\begin{center}
\begin{tabular}{c}
\includegraphics[width=3.25in]{SCRIPT_FIGURES/thouse5_smoke_noclip}\\
a) no clipping\\
\includegraphics[width=3.25in]{SCRIPT_FIGURES/thouse5_smoke_clip_blockages}\\
b) clip blockages\\
\includegraphics[width=3.25in]{SCRIPT_FIGURES/thouse5_smoke_clip_blockages_data}\\
c) clip blockages and data\\
\end{tabular}
\end{center}
\caption[Clipping a scene.]{Three views of a scene. The first view
is drawn without clipping, the second view shows the scene
clipping only the geometry (blockages), the third view shows the
scene clipping both the geometry and the data.}\ \label{figCLIPPED}
\end{figure}

It is difficult to view the interior of a scene when modeling
complicated geometries.  To alleviate this problem, one may change the blockage view
to {\em outline}\ with the Show/Hide>Blockages menu or one may clip the scene.
Portions of
the scene may be hidden or clipped by setting up to six clipping
planes. The scene is then drawn on one side of a clipping plane but
not the other. In general, a clipping plane may have any
orientation. Smokeview defines six clipping planes, two clipping planes for each of the
three coordinate axes.   The two x axis clipping planes clip
regions with $x$ coordinates smaller than `$x_{min}$' (in FDS coordinates) and larger than `$x_{max}$' .
Clipping planes for the y and z axis behave similarly.
Clipping plane values are
specified using the Clipping dialog box which is opened by
selecting the {\em Dialogs$>$Clip Geometry}\ menu item. Figure
\ref{figCLIP}\ shows this dialog box with the $y_{max}$ plane
active. Figure \ref{figCLIPPED}\ shows three versions of a scene.
Figure \ref{figCLIPPED}a is drawn with no clipping. Figure
\ref{figCLIPPED}b is drawn clipping just the geometry (blockages).
Figure \ref{figCLIPPED}c is drawn clipping both the geometry and
the data.

The clipping dialog box also allows one to hide blockages.  Blockages for any given mesh may be hidden
by selecting the appropriate checkbox in the {\em Hide blockages}\ rollout panel.

The scene may also be clipped using keyboard shortcuts.
The M key is used to toggle command line scene clipping on and off.  When turned on, the
cursor and page up/down keys can be used to move the clipping planes.
Upper and lower clipping plane are activated using the x/y/z or X/Y/Z keys.
The lower case keys toggle the lower clipping plane.  The upper case
keys toggle the upper clipping planes.
The W key is used to toggle clipping plane modes (turned off, clip only geometry, clip geometry and data
{\em etc}).


%---------------------------------------------------------------------------------
%--------------- Creating Custom Objects -----------------------------------------
%---------------------------------------------------------------------------------

\chapter{Creating Custom Objects}
\label{chap:devices}\ Smokeview visualizes FDS devices such as heat
and smoke detectors using instructions found in a file named {\tt
objects.svo}. Smokeview also uses these instructions to represent
people (avatars) in FDS-EVAC simulations and to represent trees
and shrubs in FDS WUI simulations. The Smokeview implementation of
FDS devices is referred to as objects in this chapter.

The instruction file is located in the  Smokeview installation
directory\footnote{The current objects.svo file containing
documentation and a listing of object definitions is listed in
Appendix \ref{section:objects}}. The instructions correspond to
OpenGL library calls, the same type of calls Smokeview uses to
visualize FDS cases. Smokeview then acts as an interpreter
executing OpenGL commands as specified in the object definition
file. Efficiency is attained by compiling these instructions into
display lists, terminology for an OpenGL construct for storing and
efficiently drawing collections of OpenGL commands.  New objects
may be designed and drawn without requiring modifications to
Smokeview and more importantly may be created by someone other
than the Smokeview developer.

An object's appearance may be fixed or it may be altered based
upon data specified in an FDS input file.  The {\tt sensor}\
object is drawn as a small green sphere with a fixed diameter.
Its appearance is the same regardless of how an FDS input file is
set up. The appearance of the {\tt tsphere}\ object (t for
texture) depends on data specified in the FDS input file.  One may
specify the diameter of the sphere and an image to cover it with (
the image is known as a texture map).

As with preference or {\tt .ini}\ files, Smokeview looks for
object definition files in three locations: in a file named {\tt
objects.svo}\ located in the Smokeview installation directory, in
a file named {\tt objects.svo}\ located in the casename directory
and in a file named {\tt casename.svo}\ also located in the
casename directory where {\tt casename}\ is the name of the case
being visualized.

This chapter describes how to create new objects. Though all of
the examples are given for drawing FDS devices, the intent of this
procedure is to be more general allowing Smokeview to draw other
types of objects such as people walking.

%---------------------------------------------------------------------------------
%---------------------------------------------------------------------------------
\section{Object File Format}
The first statement in an object definition is the keyword {\tt
OBJECTDEF}\ (or {\tt AVATARDEF}\ when defining a {\em person}). The
next statement is the name or label for the object. Following this
are the instructions used for creating the object. Each
instruction consist of zero or more data values followed by a
command. Comments may be placed anywhere in the object definition
file by adding text after a double slash `//`.

Data from FDS may be optionally passed to the object definition by
placing a series of labels, written as {\tt :var1 ... :varn}, at
the beginning of the definition.  These data values may then be
accessed later in the definition using {\tt \$var1 ... \$varn}\
respectively. The data placed in {\tt :vari}\ labels is specified
in the FDS input  file using the {\tt SMOKEVIEW\_PARAMETERS}\
keyword on the {\tt \&PROP}\ input line.

There are two types of instructions for drawing basic geometric
objects.  Instructions for drawing objects such as cubes, disks,
spheres etc.  and instructions for manipulating these objects
through transformations such as scaling, rotation and translation.
Collectively these instructions specify the type, location and
orientation of objects used to represent objects.  The important
feature of this process is that new objects may be designed and
drawn without the need to modify Smokeview.

Some examples of argument/instruction pairs are {\tt d drawsphere}
for drawing a sphere of diameter {\tt d}\ or {\tt x y z translate}\
for translating an object by $(x,y,z)$. The symbols {\tt d}, {\tt
x}, {\tt y}\ and {\tt z}\ are specified in the object file using a
numerical constant such as 1.23 or using a reference such as \$var
to data located elsewhere.

Transformation commands are cumulative, each command builds on the
effects of the previous one.  The commands {\tt push}\ and
{\tt pop}\ isolate these effects by saving and restoring the geometric state.

The format for an object definition file is given in more detail
in Fig. \ref{figobjectdef}.  Each object definition consists of
one or more frames.  A frame is used to represent various states
of the object. Objects such as thermocouples which do not activate
use just one frame. Other objects such as sprinklers or smoke
detectors which do activate use two frames, the first for normal
conditions and the second for when the object has activated.

\begin{figure}[bph]
{\small
\begin{lstlisting}[frame=single,rulecolor=\color{yellow},
framerule=1pt,framesep=1pc,fillcolor=\color{yellow}]
// ************ object file format ********************

//  1. comments and blank lines may be placed anywhere
//  2. any line not beginning with "//" is part of the definition.
//  3. the first non-comment line after OBJECTDEF is the object name
//  4. an object definition may contain, labels, numerical constants
//     (a number), string constants (enclosed in " ") and/or
//     commands (beginning with a-z)
//  5. a label begins with ':' as in :dx
//  6. the label :dx may be accessed afterward using $dx
//  7. An object may contain multiple frames or states.  A new frame within
//     an object is defined using NEWFRAME

// OBJECTDEF // OBJECTDEF begins the object definition

//   object_name // name or label for object
//   :var1 ... :varn  // a series of labels may be specified for use by
//                    // the object definition.  Data is copied to these
//                    // label locations using the SMOKEVIEW_PARAMETERS
//                    // &PROP keyword or from a particle file. The data
//                    // in :varn may be referenced  elsewhere in the
//                    // definition using $varn

//   // A series of argument/command pairs are specified on one or
//   // more lines.

//   arg1 ... argn command1 arg1 ... argn command2 ...

//   // An argument may be a numerical constant (e.g., 2.37), a string
//   // (e.g., "SKYBLUE"), a label (e.g., :var1),  or a reference to a
//   // label located elsewhere (e.g., $var1)

//  NEWFRAME    // beginning of next frame
//   more argument/command pairs for the next object frame
//   ....
\end{lstlisting}
}
\caption{Object file format.}
\label{figobjectdef}%
\end{figure}

Figure \ref{figsensor}\ illustrates a simple example of an object definition
used to draw a sensor along with the corresponding Smokeview view.
The definition uses just one frame. A sphere is drawn with color
yellow and diameter 0.038 m. Push and pop commands are not necessary
because there is only one object and no transformations are used.

\begin{figure}[bph]
{\small
\begin{lstlisting}[frame=single,rulecolor=\color{yellow},
framerule=1pt,framesep=1pc,fillcolor=\color{yellow}]
OBJECTDEF
 sensor
 1.0 1.0 0.0  setcolor
 0.038 drawsphere
\end{lstlisting}
}
\begin{center}
\includegraphics[height=\figheightA]{SCRIPT_FIGURES/sensorplain}\\
\end{center}
\caption{Instructions for drawing a sensor along with the corresponding Smokeview view.}
\label{figsensor}%
\end{figure}

The example illustrated in Fig. \ref{figsprinkler}\ is more complicated.
It shows a definition of a heat detector along with a corresponding
Smokeview view. The definition uses two frames. The first frame represents
the heat detector's inactive state, the second frame represents the active
state (commands after the {\tt NEWFRAME}\ keyword). This definition uses
disks, a truncated cone and spheres. The scale and translate commands are
used to draw these objects at the proper size. The translate command then
positions them properly.  Two frames are defined for both the inactive and
active (after the heat detector has activated.) states.

\begin{figure}[bph]
{\bf Heat detector Instructions}\\
{\small
\begin{lstlisting}[frame=single,rulecolor=\color{yellow},
framerule=1pt,framesep=1pc,fillcolor=\color{yellow}]
OBJECTDEF
 heat_detector         // label, name of object

 // The heat detector has three parts
 //   a disk, a truncated disk and a sphere.
 //   The sphere changes color when activated.

 0.8 0.8 0.8 setcolor  // set color to off white
 push 0.0 0.0 -0.02 translate 0.127 0.04 drawdisk pop
 push 0.0 0.0 -0.04 translate 0.06 0.08 0.02 drawtrunccone pop
 0.0 1.0 0.0 setcolor
 push 0.0 0.0 -0.03 translate  0.04 drawsphere pop
 // push and pop are not necessary in the last line
 //   of a frame.  Its a good idea though, to prevent
 //   problems if parts are added later.
NEWFRAME  // beginning of activated definition
 0.8 0.8 0.8 setcolor
 push 0.0 0.0 -0.02 translate 0.127 0.04 drawdisk pop
 push 0.0 0.0 -0.04 translate 0.06 0.08 0.02 drawtrunccone pop
 1.0 0.0 0.0 setcolor
 push 0.0 0.0 -0.03 translate 0.04 drawsphere pop
\end{lstlisting}
}
\begin{center}
\begin{tabular}{cc}
 \includegraphics[height=\figheightA]{SCRIPT_FIGURES/heatdetector_inact}&
 \includegraphics[height=\figheightA]{SCRIPT_FIGURES/heatdetector_act}\\
inactive&active\\
 \end{tabular}
 \end{center}
\caption{Instructions for drawing an inactive and active heat detector
along with the corresponding Smokeview view.}
\label{figsprinkler}%
\end{figure}

Figure \ref{figball}\ shows an example of a definition used to draw a
scaled sphere using scalings obtained from an FDS input file along
with the corresponding Smokeview view.  This definition is set up
so that if the label value 'D' has a value greater than 0.0 then a
sphere is drawn with diameter D otherwise an ellipsoid is drawn with
dimensions 'DX', 'DY' and 'DZ'. This definition uses just one frame.
The scaled sphere/ellipsoid is drawn using data specified on the
{\tt SMOKEVIEW\_PARAMETERS}\ keyword in the FDS input file.

\begin{figure}[bph]
{\small
\begin{lstlisting}[frame=single,rulecolor=\color{yellow},
framerule=1pt,framesep=1pc,fillcolor=\color{yellow}]
OBJECTDEF // object for a general ball
 ball
 :R=0 :G=0 :B=0 :DX :DY :DZ :D=-.1
 $D 0.0 :DGT0 GT
 $R $G $B setrgb
 $DGT0 IF
  $D drawsphere
  ELSE
  $DX $DY $DZ scalexyz 1.0 drawsphere
 ENDIF
 NO_OP
\end{lstlisting}
}
{\small
\begin{lstlisting}[frame=single,rulecolor=\color{yellow},
framerule=1pt,framesep=1pc,fillcolor=\color{yellow}]
FDS input lines to create ball

The data labels (:R=0 :G=0 :B=0 :DX :DY :DZ :D=-.1) in the object file
correspond to the SMOKEVIEW_PARAMETERS inputs in the FDS input file
though the order may be different.

&PROP ID='ball' SMOKEVIEW_PARAMETERS(1:5)='R=0','G=0','B=255',
                       'DX=0.25','DY=.5','DZ='1.0' SMOKEVIEW_ID='ball' /
&DEVC XYZ=0.5,0.8,2.5, QUANTITY='TEMPERATURE' PROP_ID='ball' /

\end{lstlisting}
}
\begin{center}
\includegraphics[height=\figheightA]{../SMV_User_Guide/SCRIPT_FIGURES/ball}\\
\end{center}
\caption{Instructions for drawing the dynamic object, ball,
along with the corresponding FDS input lines and the Smokeview view.}
\label{figball}%
\end{figure}

\newcommand{\devfig}[1]{
\includegraphics[height=1.7in]{SCRIPT_FIGURES/#1}
}

Figure \ref{figdevices}\ gives Smokeview views for several objects
defined in the {\tt objects.svo}\ file. A more complete list is
found in the FDS User's Guide~\cite{FDS_Users_Guide}.

\begin{figure}[bph]
\begin{center}
\begin{tabular}{cc}
 \devfig{sprinkler_inact}&\devfig{sprinkler_act}\\
 inactive up-right sprinkler&active up-right sprinkler\\

 \devfig{smokedetector_inact}& \devfig{smokedetector_act}\\
 inactive smoke detector&active smoke detector\\

 \devfig{sensor}&\devfig{target}\\
 sensor&target
 \end{tabular}
 \end{center}
\caption{Smokeview view of several objects defined in the
objects.svo file. }
\label{figdevices}%
\end{figure}

%---------------------------------------------------------------------------------
%---------------------------------------------------------------------------------
\section{Elementary Geometric Objects}
\label{svocommands}\ The objects described in this section are the
building blocks used to construct more complex objects.
Each command used to draw an elementary geometric object
consists of one or more arguments followed by the command,
for example, the command sequence {\tt 0.3 drawsphere}\ draws
a sphere with diameter 0.3 (all units as with FDS are in meters).

%\infigr{circle}{0.5 drawcircle}
\infigr{drawarcdisk}{60.0 0.25 0.50 drawarcdisk}
\paragraph{drawarcdisk}\ The command, {\tt a d h drawarcdisk},
draws a portion of circular disk with angle a, diameter d and
height h. The origin is located at the center of the disk's base.\vspace{0.25in}

\paragraph{drawcircle}\ The command, {\tt d drawcircle },
draws a circle with diameter d.  The origin is located at
the center of the circle.

\infigl{drawcone}{0.50 0.30 drawcone}
\paragraph{drawcone}
The command, {\tt d h drawcone}, draws a right circular cone
where d is the diameter of the base and h is the height.
The origin is located at the center of the cone's base.

\infigr{drawcube}{0.25 drawcube}
\paragraph{drawcube}\ The command, {\tt s drawcube}, draws
a cube where s is the length of the side.  The origin is
located at the center of the cube.  An oblong box, a box
with different length sides, may be drawn by using
{\tt scale}\ along with {\tt drawcube}.  For example,
{\tt 1.0 2.0 4.0 scale 1.0 drawcube}\ creates a box
with dimensions $1\times 2\times 4$.

\infigl{drawcube}{0.25 drawcubec}
\paragraph{drawcubec}\ The command, {\tt s drawcubec},
is the same as {\tt s drawcube}\ except that the origin
is located at the front, left, bottom corner of the cube
rather than at the cube center.  An oblong box, a box
with different length sides, may be drawn by using
{\tt scale}\ along with {\tt drawcubec}.  For example,
{\tt 1.0 2.0 4.0 scale 1.0 drawcube}\ creates a box with
dimensions $1\times 2\times 4$.

\infigr{drawdisk}{0.25 0.50 drawdisk}\ \paragraph{drawdisk}
The command, {\tt d h drawdisk}, draws a circular disk with
diameter d and height h. The origin is located at the center
of the disk's base.\vspace{0.25in}

\infigl{drawcdisk}{0.25 0.50 drawcdisk}\ \paragraph{drawcdisk}
The command, {\tt d h drawcdisk}, draws a circular disk with
diameter d and height h. The origin is located at the center
of the disk. This command is a shortcut for
{\tt h 2.0 :hd2 div \$hd2 offsetz d h drawdisk}.\vspace{0.25in}

\infigr{drawhexdisk}{0.5 0.25 drawhexdisk}\ \paragraph{drawhexdisk}
The command, {\tt d h drawhexdisk}, draws a hexagonal disk with diameter d and height h.
The origin is located at the center of the hexagon's base.\vspace{0.25in}

%\infigl{line}{0.0 0.0 -0.5 0.0 0.0 0.5}
\paragraph{drawline}\ The command, {\tt x1 y1 z1 x2 y2 z2 drawline}, draws a
line between the points $(x_1,y_1,z_1)$ and $(x_2,y_2,z_2)$.\vspace{0.25in}

\parpic[l]{
\begin{tabular}{cc}
\includegraphics[height=\infigheight]{SCRIPT_FIGURES/drawnotchplate}&
\includegraphics[height=\infigheight]{SCRIPT_FIGURES/drawnotchplate2}\\
{\tiny\tt  0.5 0.1 0.2 1 drawnotchplate}&
{\tiny\tt 0.5 0.1 0.2 -1 drawnotchplate}
\end{tabular}
}
\paragraph{drawnotchplate}\ The command, \\ {\tt d h nh dir drawnotchplate},
draws a  notched plate.  This object is used to represent a
portion of a sprinkler where d is the plate diameter, h is the plate height
(not including notches), nh is the height of the notches and dir indicates the notch orientation (1 for vertical, -1 for horizontal).  The origin is located at the center of the plate's base.

\paragraph{drawpoint}\ The command, {\tt drawpoint}, draws a point (small square).
The command, {\tt s setpointsize}\ may be used to
change the size of the point.  The default size is 1.0 .

\infigl{drawpolydisk}{5 0.35 0.15 drawpolydisk}\ \paragraph{drawpolydisk}
The command, {\tt n d h drawpolydisk}, draws an n-sided polygonal
disk with diameter d and height h.  The origin is located at the
center of the polygonal disk's base.  The example to the left is a pentagonal disk.

\infigr{drawring}{0.3 0.5 0.1 drawring}\ \paragraph{drawring}
The command, {\tt di do h drawring}, draws a ring where
{\tt di}\ and {\tt do}\ are the inner and outer ring diameters and h is the height of the ring.
The origin is located at the center of the ring's base.

\infigl{drawsphere}{0.25 drawsphere}\ \paragraph{drawsphere}
The command, {\tt d drawsphere}, draws a sphere with diameter d.
The origin is located at the center of the sphere.
As with an oblong box, an ellipsoid may be drawn by using {\tt scale}\ along with
{\tt drawsphere}. For example, {\tt 1.0 2.0 4.0 scale 1.0 drawsphere}\ creates an
ellipsoid with semi-major axes of length 1, 2 and 4.
This is how the ball at the bottom of the heat detector in
Fig. \ref{figsprinkler}a is drawn.

\infigr{drawtrunccone}{0.5 0.2 0.4 drawtrunccone}\ \paragraph{drawtrunccone}
The command, {\tt d1 d2 h drawtrunccone}, draws a right circular truncated
cone where d1 is the diameter of the base, d2 is the diameter of the
truncated portion of the cone and h is the height or distance between the
lower and upper portions of the truncated cone.  The origin is located at
the center of the truncated cone's base.

\vspace{0.25in}
%---------------------------------------------------------------------------------
%---------------------------------------------------------------------------------
\section{Visual Transformations}
As with geometric commands, transformation commands consist of zero or more
arguments followed by the command. Transformation commands are used to
directly or indirectly change how drawn objects appear.  Visual transformations make changes directly, changing the location and orientation of drawn objects,  setting drawing attributes such as point size, line width or object color or saving and restoring the geometric state. Arithmetic transformations, described in the next section, make changes indirectly by operating on data which in turn is used as inputs to various drawing commands.

Visual transformation commands map directly to counterparts in OpenGL.
The rotate and translate commands change the origin (translate) or
orientation of the x,y,z axes (rotate). The offsetx, offsety and
offsetz commands translate objects along just one axis.
The PUSH command is then used to save the origin or axis
orientation while the POP command is used to restore the
origin and axis orientation.

\blist
\hitem{gettextureindex}\ The command
\begin{lstlisting}
"texture_file" :texture_index gettextureindex
\end{lstlisting}
finds the index in an internal Smokeview table containing the entry
texture\_file (a file containing a texture map image). This index
is used by other object drawing routines that support texture
mapping (presently drawtsphere).

\hitem{gtranslate}\ The command, {\tt x y z gtranslate}, translates
objects in a global reference frame, the same reference frame used
to define FDS geometry.  Objects drawn after the gtranslate command
are moved by x, y and z along the x, y and z Cartesian axes respectively.
Equivalently, one can think of think of {\tt x y z gtranslate}\ as
translating the origin by (-x,-y,-z).

\hitem{offsetx}\ The command, {\tt x offsetx}, translates objects drawn
afterwards by x along the x axis.

\hitem{offsety}\ The command, {\tt y offsety}, translates objects drawn
afterwards by y along the y axis.

\hitem{offsetz}\ The command, {\tt z offsetz}, translates objects drawn
afterwards by z along the z axis.

\hitem{orienx, orieny, orienz}\ The command, {\tt x y z orienx}, rotates
the scene so that the vector $u=(1,0,0)$ in the
original scene maps to $w=(x,y,z)$. The commands {\tt orieny}\ and {\tt orienz}
are similar to {\tt orienx}\ mapping $(0,1,0)$ and
$(0,0,1)$ to $(x,y,z)$ instead.

\hitem{pop}\ The command, {\tt pop}, restores the origin and axis orientation
saved using a previous {\tt push}\ command.  The total
number of {\tt pop}\ and {\tt push}\ commands must be equal, otherwise a
fatal error will occur.  Smokeview detects this problem and draws a red
sphere instead of the incorrectly defined object.

\hitem{push}\ The command, {\tt push}, saves the origin and axis orientation.
(see above comment about number of {\tt push}\ and
{\tt pop}\ commands).

\hitem{randxy, randxz, randyz}\ The command {\tt flag randxy}\ performs a
random rotation about the z axis (in the xy plane) if
{\tt flag}\ is 1. If {\tt flag}\ is anything else this command has no effect.
The commands {\tt randxz}\ and {\tt randyz}\ are similar,
rotating within the xz and yz planes instead of the xy plane.

\hitem{rotateaxis}\ The command, {\tt angle x y z rotateaxis}, rotates objects
drawn afterwards by {\tt angle}\ degrees about an
axis defined by the vector $(x,y,z)$.

\hitem{rotatexyz}\ The command, {\tt x y z rotatexyz}, rotates objects from the
vector $(0,0,1)$ to the vector $(x,y,z)$ .  The
axis of rotation computed internally by Smokeview is $(0,0,1)\times
(x,y,z)=(-y,x,0)$ (a vector perpendicular to the plane formed by vectors
$(0,0,1)$ and $(x,y,z)$) . The cosine of the angle of rotation is $z/\sqrt{x^2+y^2+z^2}$

\hitem{rotatex}\ The command, {\tt r rotatex}, rotates objects drawn afterwards
{\tt r}\ degrees about the x axis.

\hitem{rotatey}\ The command, {\tt r rotatey}, rotates objects drawn afterwards
{\tt r}\ degrees about the y axis.

\hitem{rotatez}\ The command, {\tt r rotatez}, rotates objects drawn afterwards
{\tt r}\ degrees about the z axis.  A cone or any
object for that matter may be drawn upside down by adding a {\tt rotatez}\
command as in {\tt 180 rotatez 1.0 0.5 drawcone}.

\hitem{scalexyz}\ The command, {\tt x y z scalexyz}, stretches objects drawn
afterwards by x, y and z respectively along the x, y
and z axes. The {\tt scalexyz}\ along with the {\tt drawsphere}\ commands
would be used to draw an ellipsoid by stretching a sphere
along one of the axes.

\hitem{scale}\ The command, {\tt xyz scale}, stretches objects drawn afterwards
xyz along each of the x, y and z axes (equivalent
to {\tt xyz xyz xyz scalexyz}\ ).

\hitem{setbw}\ The command, {\tt gray setbw}, sets the red, green and blue components
of color to gray (equivalent to gray gray gray
setcolor ).  As with the {\tt setcolor}\ command, {\tt setbw}\ is only required when
the gray level changes, not for each object drawn.

\hitem{setcolor}\ The command, {\tt ``color name'' setcolor}, obtains sets the color
to the red, green and blue components of
the FDS standard color {\tt color~name}.

\hitem{setlinewidth}\ The command, {\tt w setlinewidth}\, sets the width of lines
drawn with the {\tt drawline}\ and {\tt drawcircle}\ commands.

\hitem{setpointsize}\ The command, {\tt s setpointsize}, sets the size of points
drawn with the {\tt drawpoint}\ command.

\hitem{setrgb}\ The command, {\tt r g b setrgb}, sets the red, green and blue
components of the current color.  Any objects drawn
afterwards will be drawn with this color. This command is not required for
each object part drawn. The color component values range from 0 to 255.

\hitem{translate}\ The command, {\tt x y z translate}, translates objects
drawn afterwards by x, y and z along x, y and z axes respectively relative to the current (local) reference frame.

\elist


\newcommand{\valclipped}{{\tt \mbox{val\_clipped}}}
\newcommand{\valin}{{\tt \mbox{val\_in}}}
\newcommand{\valmin}{{\tt \mbox{val\_min}}}
\newcommand{\valmax}{{\tt \mbox{val\_max}}}
\newcommand{\valone}{{\tt \mbox{val\_1}}}
%---------------------------------------------------------------------------------
%---------------------------------------------------------------------------------
\section{Arithmetic Transformations}
Arithmetic transformation commands allow one to indirectly change how objects are
drawn using information passed from FDS. This information is passed using the
{\tt SMOKEVIEW\_PARAMETERS}\ keyword on the {\tt \&PROP}\ namelist statement.
These commands transform data to change the inputs of subsequent object commands.

\blist
\hitem{add}
The command, \\
\\
{\tt a b :val add}, \\
\\
is used to compute the value, $val=a+b$, where $a$ and $b$ are either numerical
constants or references to previously defined data.  The result, $val$ is
placed in the label {\tt :val}\ accessible later in the definition file using \$val.

\hitem{clip}
The command, \\
\\
\valin\ \valmin\ \valmax\ :\valclipped\ {\tt clip}, \\
\\
is used to clip a value \valin\ between \valmin\ and
\valmax\ using
\begin{eqnarray}
\valclipped=\max(\valmin,\min(\valin,\valmax))
\end{eqnarray}

\noindent The inputs, \valin, \valmin\ and \valmax\ are either numerical
constants or references to previously defined data.  The clipped result
is placed in the label :\valclipped\ accessible later in the definition
file using \$\valclipped.

\hitem{div}
The command, \\
\\
{\tt a b :val div}, \\
\\
is used to compute the value, $val=a/b$, where $a$ and $b$ are either
numerical constants or references to previously defined data.  If the
denominator, {\tt b}, is zero then the result, $val$, returned is zero
and is placed in the label {\tt :val}\ accessible later in the definition
file using \$val.

\hitem{eq}
The command, \\
\\
{\tt a b eq}, \\
\\
is used to copy data from the label b to a, {\em ie}\ performs the operation a=b.

\hitem{gett}
The command, \\
\\
{\tt :time gett}, \\
\\
is used to obtain the current simulation time.
The simulation time is placed in the label {\tt :time}\ accessible later
in the definition file using \$time.

\hitem{mirrorclip}
The command, \\
\\
\valin\ \valmin\ \valmax\ :\valclipped\ mirrorclip, \\
\\
is used to clip a value \valin\ between \valmin\
and \valmax\ using
\begin{eqnarray}
\valone&=&\mbox{mod}( \valin-\valmin,2(\valmax-\valmin) )\\
\valclipped&=&\left\{
\begin{array}{cc}
  \valmin+\valone & \valone\le \valmax-\valmin \\
  \valmax-\valone & \valone> \valmax-\valmin
\end{array}
\right.
\end{eqnarray}

\hitem{mult}
The command, \\
\\
{\tt a b :val mult}, \\
\\
is used to compute the value, $val=ab$, where $a$ and $b$ are either
numerical constants or references to previously defined data.
The result, $val$ is placed in the label {\tt :val}\ accessible
later in the definition file using \$val.

\noindent The inputs, \valin, \valmin\ and \valmax\ are either
numerical constants or references to previously defined data.
The clipped result is placed in the label :\valclipped\ accessible
later in the definition file using \$\valclipped.

\hitem{multiaddt}
The command, \\
\\
{\tt a b :val multiaddt}, \\
\\
is used to compute the value, $val=at+b$, where $t$ is the simulation
time and $a$ and $b$ are either numerical constants or references to
previously defined data. The result, $val$ is placed in the label
{\tt :val}\ accessible later in the definition file using \$val.
This allows one to change how an object appears as a function of
time (changing its size, rotating it, changing its color, etc.).

The command, {\tt a b :val multiaddt}, is a shortcut for\\
\\
{\tt :time gett a \$time :at mult \$at b :val add}\\
\\


\hitem{periodicclip}
The command, \\
\\
\valin\ \valmin\ \valmax\ :\valclipped\ {\tt periodicclip}, \\
\\
is used to clip a value \valin\ between \valmin
and \valmax\ using
\begin{eqnarray}
\valclipped&=&\valmin+\mbox{mod}( \valin-\valmin,\valmax-\valmin )\\
\end{eqnarray}

\noindent The inputs, \valin, \valmin\ and \valmax\ are either numerical
constants or references to previously defined data.  The clipped result is placed
in the label :\valclipped\ accessible later in the definition file using \$\valclipped.

\hitem{sub}
The command, \\
\\
{\tt a b :val sub}, \\
\\
is used to compute the value, $val=a-b$, where $a$ and $b$ are either
numerical constants or references to previously defined data.
The result, $val$ is placed
in the label {\tt :val}\ accessible later in the definition file using \$val.
\elist

%---------------------------------------------------------------------------------
%---------------------------------------------------------------------------------
\section{Logical and Conditional Operators}
Logical and conditional operators are used in conjunction to test values and
execute portion of an object definition depending on the results of the
test.  Logical operators return 1 if the test is true and 0 if the test
is false.

\blist
\hitem{and}\  The command\\
\\
{\tt a b :val AND}\\
\\
returns 1 in :val if both a
and b are true (any value other than 0), otherwise it returns 0.

\hitem{gt}\  The command\\
\\
{\tt a b :val GT}\\
\\
returns 1 in :val if a is
greater than b, otherwise it returns 0.

\hitem{ge}\  The command\\
\\
{\tt a b :val GE}\\
\\
returns 1 in :val if a is
greater than or equal to b, otherwise it returns 0.

\hitem{if,else,endif}\ Consider the object command sequence
 \begin{lstlisting}
 $val IF
  arg1 arg2 command1 arg1 arg2 command2 ....
 ELSE
  arg1 arg2 command3 arg1 arg2 command4 ....
 ENDIF
 \end{lstlisting}
The value {\tt \$val}\ is typically generated from a previous logical
operation ({\em ie}\ with {\tt GE, LT}, etc.).  The commands between the
{\tt IF}\ and {\tt ELSE}\ operators are executed
 if {\tt \$val}\ is not 0 otherwise the commands between {\tt ELSE}\ and
 {\tt ENDIF}\ are executed.  The {\tt ELSE}\ operator is optional.

\hitem{lt}\  The command\\
\\
{\tt a b :val LT}\\
\\
returns 1 in :val if a is less
than b, otherwise it returns 0.

\hitem{le}\  The command\\
\\
{\tt a b :val LE}\\
\\
returns 1 in :val if a is less
than or equal to b, otherwise it returns 0.

\hitem{or}\  The command\\
\\
{\tt a b :val OR}\\
\\
returns 1 in :val if either a
or b are true (any value other than 0), otherwise it returns 0.
\elist

%---------------------------------------------------------------------------------
%--Manipulating the Scene Automatically - The Touring Option ---------
%---------------------------------------------------------------------------------

\chapter{Manipulating the Scene Automatically - The Touring
Option}\ \label{chapter:touring}

\begin{figure}[bph]
\begin{center}
\includegraphics[width=4.4444in]{SCRIPT_FIGURES/tour_Circular_tour}\\
\end{center}
\caption[Overhead view of the townhouse example showing the
default {\em Circle}\ tour.]{Overhead view
of townhouse example showing the default {\em Circle}\ tour.
The square dots indicate the keyframe
locations. Keyframes may be edited using the Touring
dialog box or by clicking and dragging with the mouse.}
 \label{figTOUREXAMPLE}
\end{figure}

The touring option allows one to specify a path or tour through a
Smokeview scene.  One may then view the scene from the vantage point of an
observer moving along this path. Smokeview creates a tour surrounding
the scene by default.  This tour is named {\em Circular}.
An example for the townhouse case is illustrated
in Fig. \ref{figTOUREXAMPLE}.
The user may create a new tour or modify an existing tour using the Tour
dialog box illustrated in Fig. \ref{figTOUR}. The user creates a tour by
defining two or more keyframes. Each keyframe defines a 
position, view direction, pause time and optionally a time.
By default, movement along a tour path occurs at a constant rate or velocity. 
In this case, keyframe arrival times are calculated by smokeview.  You may also specify
arrival times at any keyframe (overriding the constant velocity assumption) 
by selecting the {\em Set time}\ checkbox.
The default view direction is towards the position (0.0,0.0,0.0) or lower front left of the scene and the default pause time is 0.0~s.
Smokeview creates a smooth
path through these positions using piecewise cubic Hermite polynomials.

\begin{figure}[bph]
\begin{center}
\includegraphics[width=4.41111in]{\SMVfigdir/figTOUR}
\end{center}
\caption[Touring dialog box.]{The Touring dialog
box may be used to select tours or keyframes, change the
position or view direction at each keyframe and change the tension
of the tour path. }
 \label{figTOUR}
\end{figure}

%---------------------------------------------------------------------------------
%---------------------------------------------------------------------------------
\section{Tour Settings}
A new tour is created
by clicking the \frameit{New Tour}\ button in the Edit Tour dialog box.
The newly created tour has two keyframes.  The tour
goes through the middle of the Smokeview scene starting at the
front left and finishing at the back right.
A tour may be modified by selecting the {\em Edit Tour}\ checkbox.
Tour characteristics such as keyframe positions and view directions at those positions
are saved in the configuration file, {\tt casenamea .ini}.

The {\em View From Tour Path} checkbox is used to control how one observes the
scene using a Tour.  If {\em View From Tour Path}\ is checked then one moves through the
scene along the currently specified tour.  Unchecking this option
returns control of scene movement to the user allowing one to have a global view of the tour.

By default Keyframe times are  proportional to the path length
between keyframes so that the speed traversed along the tour is constant.
This behavior may be overridden by specifying an arrival time at any keyframe after selecting
the {\em Set time} checkbox.


%---------------------------------------------------------------------------------
%---------------------------------------------------------------------------------
\section{Keyframe Settings}
A tour is created from a series of keyframes.  One may specify
 the position, view direction and pause time at each keyframe.  Smokeview then
obtains positions and view directions between keyframes by interpolating
using piecewise cubic Hermite cubic splines.
A tour is created by pressing the
\frameit{Add Tour}\ button.  This tour has two keyframes located
at opposite ends of the Smokeview scene.  Additional keyframes may
be created by selected the \frameit{Add}\ button. A tour may also be
created by pressing the \frameit{Copy} button which makes a copy of the
currently selected tour.  A orientation of a tour may be reversed by selecting
the \frameit{Reverse} button.

The position and viewpoint of a keyframe may be adjusted.  First
it must be selected.  A keyframe may be selected by either
clicking on it with the left mouse button or by {\em moving}\ through
the keyframes using the \frameit{Next}\ or \frameit{Previous}\
buttons. The active keyframe as drawn within the Smokeview scene changes color from red to green.
Keyframe positions may then be modified by changing
data in the X, Y or Z edit boxes or dragging the keyframe rectangle with the mouse.  A different view direction
may also be set.

A keyframe may also be moved with the mouse.  Clicking on a
keyframe node then dragging the mouse left/right and up/down
moves the keyframe horizontally and vertically.  Pressing the
<CTRL> key while dragging the mouse
restricts keyframe movement to a horizontal direction (with respect to the mouse).
Pressing the <ALT> key while dragging the mouse
restricts keyframe movement to a vertical direction (again with respect to the mouse).

A new keyframe is created by clicking the \frameit{Add}\ button.
It is formed by averaging the positions and view directions of the
current and next keyframes. If the selected keyframe is the last
one in the tour then a new keyframe  is added beyond the last
keyframe.  A keyframe may also be created by pressing the 'a' key.

A keyframe may be deleted by clicking the \frameit{Delete}\
button. The currently selected keyframe may also be deleted by pressing the 'd' key.
There is no \frameit{Delete Tour}\ button. A tour may be
deleted by either deleting all of its keyframes or by deleting its
entry in the casename.ini file.

A view direction may be defined at each keyframe by either setting
direction angles relative to the path (an azimuthal and an elevation
angle) or by setting a direction relative to the scene geometry (a
Cartesian (X,Y,Z) view direction).

Path view directions are absolute.  One selects the (x,y,z) view position by
editing the x,y and/or z edit boxes in the View direction panel.

Cubic Hermite polynomials  are uniquely
specified for each interval using function and slope values
at each endpoint of
the interval ({\em i.e., 4 data values}).


%---------------------------------------------------------------------------------
%---------------------------------------------------------------------------------
\section{Setting up a tour}
\begin{figure}[bph]
\begin{center}
\begin{tabular}{cc}
\includegraphics[height=\figheightA]{SCRIPT_FIGURES/tour_start_tour}&
\includegraphics[height=\figheightA]{SCRIPT_FIGURES/tour_example_tour1}\\
a) Initial tour&b) First and last step set with 5 keyframes\\
\includegraphics[height=\figheightA]{SCRIPT_FIGURES/tour_example_tour2}\\
c) All keyframe positions set
\end{tabular}
\end{center}
\caption [Tutorial examples for Tour option.] {Tutorial examples for Tour option.}
\label{figTutorial}%
\end{figure}


The following steps give a simple example of setting up a tour in the
townhouse scenario.  The tour will begin at the back of the house,
go towards the front door and then end at the top of the stairs.
These steps are illustrated in Fig. \ref{figTutorial}.

\begin{enumerate}
\item Start by clicking the {\em Dialog$>$View$>$Create/edit tours...}\ menu item
which opens up the Edit Tours dialog box.

\item  Click on the \frameit{New Tour}\ button in the Edit Tour dialog
box. This creates a tour, illustrated in Fig.
\ref{figTutorial}a, starting at the front left of the scene and
ending at the back right. This tour has two keyframes.  The
elevation of each keyframe is halfway between the bottom and top
of the scene.

\item Click on the {\em Edit Tour Path}\ checkbox. This activates
buttons that allows the user to edit the properties of each
individual keyframe. Click on the square dot at the back of the
townhouse. This is the first keyframe. Change the ``Z'' value to
1.0.  Click on the second dot and change its ``Z'' value to 1.0.


\item  Click on the \frameit{Add}\ button, found inside the {\em Edit
Keyframe's Position}\ panel, three times. This will add three more
keyframes to the tour which will be needed so that the path bends
up the stairs. You should now have five keyframes.

\item Move the first keyframe at the back of the townhouse near the
double door by setting X, Y, Z positions to (3.8,-1.0,1.6).
Move the last keyframe to the top
of the steps by setting X, Y, Z positions to (6.1,3.6,4.1).
The path should now look like
Figure \ref{figTutorial}b.

\item Move the second, third and fourth keyframes to positions
(4.0,4.0,1.6), (4.4,6.8,1.6) and (5.9,6.1,2.0).  The path should
now look like Fig. \ref{figTutorial}c.

\item Click on the \frameit{Save Settings}\ button to save the results
of your editing changes.

\item To see the results of the tour, click on the {\em View From
Tour Path}\ checkbox.
\end{enumerate}

%---------------------------------------------------------------------------------
%-----Running Smokeview Automatically - The Scripting Option ----------------
%---------------------------------------------------------------------------------

\chapter{Running Smokeview Automatically - The Scripting Option}
\label{chapter:scripting}
%---------------------------------------------------------------------------------
%---------------------------------------------------------------------------------
\section{Overview}
Smokeview may be run in an automatic or batch mode using
instructions found in a text file.
The intent of the scripting option is to allow one to reproducibly document a case.
A script may be re-run resulting in newly generated images guaranteed
to correspond properly with the previously generated
ones whenever changes occur in the FDS input file or in the FDS or Smokeview applications.

Script instructions direct Smokeview to perform actions such as
loading data files, moving the scene to a specified view point,
setting the time and rendering the scene.
Smokeview settings such as font sizes, file bounds, label visibility,
etc. are set by using the script command LOADINI to load a custom named .ini file.
A simplified scripting language results by allowing
most customizations to be performed through the use of .ini files.


%---------------------------------------------------------------------------------
%---------------------------------------------------------------------------------
\section{Creating a Script}
Scripts may be created by Smokeview using the script recorder
feature or may be created by editing a text file using commands
described in the glossary that follows. A script may be run using
three methods.  It may be run from within Smokeview using the {\em
Load/Unload$>$Script Option}\ menu or from the Scripts panel of
the File/Bounds dialog box illustrated in Fig.
\ref{figSCRIPT}. It may also be run from a Windows command shell
using the command

\begin{lstlisting}
smokeview -runscript casename
\end{lstlisting}

\noindent where casename is the name specified by the {\tt CHID}
keyword defined in the FDS input data file.

\begin{figure}[bph]
\centerline{
\includegraphics[width=3.5069444in]{\SMVfigdir/figSCRIPT}
}\ \caption[Script dialog box.]{Script dialog box.
The Script dialog box allows one to setup and run Smokeview
scripts. }\ \label{figSCRIPT}
\end{figure}

The recorder is turned on using the
{\em Load/Unload$>$Script Option}\ menu and selecting {\em Start Recording}.
After performing a sequence of steps, it is turned off and the
script is saved. Typically steps involve loading data files,
setting view points, setting times and rendering images.

%%-----------------------------------------------------------------------
\subsection{Example 1}

This example describes the steps used to create a simple script.  This script
will load a slice file and then display and render it at 10 s, 20 s, 30 s and 40 s.
The script corresponding to the steps listed below is given
in Fig. \ref{figsimplescripttext}\ and the resulting generated images are given
in Fig. \ref{figsimplescriptimages}.
.
\begin{figure}[bph]
\fdsinput{script_slice_test.ssf}
\caption{Script commands generated using the Smokeview script recorder option.}
\label{figsimplescripttext}%
\end{figure}

\begin{figure}[bph]
\begin{center}
\begin{tabular}{cc}
\includegraphics[height=4.0in]{SCRIPT_FIGURES/script_slice_test_10}&
\includegraphics[height=4.0in]{SCRIPT_FIGURES/script_slice_test_20}\\
10.0 s&20.0 s\\
\includegraphics[height=4.0in]{SCRIPT_FIGURES/script_slice_test_30}&
\includegraphics[height=4.0in]{SCRIPT_FIGURES/script_slice_test_40}\\
30.0 s&40.0 s\\
\end{tabular}
\end{center}
\caption{Smokeview images generated using script detailed in
 Fig. \ref{figsimplescripttext}}
\label{figsimplescriptimages}%
\end{figure}


Note that the keyword, {\tt RENDERDIR}, may used to {\em direct}\ that rendered images
be placed in any directory not just the current one.  Also, the {\tt RENDERONCE}\ keywords
in this script have a blank line afterwards (put there by default by the Smokeview script
recorder).
In this case, Smokeview uses the default name for the resulting rendered image file.
If this line is not blank, it is then used for the file name.

\begin{enumerate}
\item Obtain the test case {\tt script\_slice\_test.fds}\ from the
    {\tt Verification/Visualization}\ directory in
the FDS$-$SMV repository.
\item Run the case with FDS
\item After opening the case in Smokeview,
select the {\em Load/Unload$>$Script Options$>$Start Recording}\ menu item.
\item Load a slice file (doesn't matter which one).
\item Move the timebar to 10 s and
then press the `r' key.  Repeat for 20 s, 30 s, and 40 s
\item Unload the slice file. (Not necessary, this step
just makes the script action more obvious.)
\item Select the {\em Load/Unload$>$Script Options$>$Stop Recording}\
menu item.  This is very important.  The script will not be saved
if you exit Smokeview without selecting this option.
\item Run the script using the
  {\em Load/Unload$>$Script Options}\ menu .
\end{enumerate}

%%-----------------------------------------------------------------------
\subsection{Example 2}
This example describes the steps used to create a script that is more involved.
It is listed
in Fig. \ref{figscripttext}\ which in turn was used to
create the images
illustrated in Fig. \ref{figscriptimages}.  The script built here
will create three images,
a slice file viewed and clipped from the left at 5 s,
the same slice file viewed from the center at 10 s,
and again the same slice file viewed and clipped from the right at 15 s.
The center slice is not clipped.

Several preliminary steps need to be performed before script actions may be recorded.
In particular a left and right view point will be defined and
an .ini file will be setup that contains clipping values for the
left and right slice file images.

\subsubsection{Obtaining and setting up the example}

\begin{enumerate}
\item Obtain the test case {\tt script\_test.fds}\ from the
    {\tt Verification/Visualization}\ directory in
the FDS$-$SMV repository.  Copy this file to a separate directory
 if a local copy of the repository already exists
 Of course, these steps may be repeated
 for any test case that have data files defined.
\item Run the case with FDS \item Open the case in Smokeview \item
Open the Scripts/Config panel of the File/Bounds dialog
box, the Clip Geometry dialog box and the Viewpoints panel
of the Motion/View/Render dialog box.
\end{enumerate}

\subsubsection{Preliminary Steps - Setting up the viewpoints}

\begin{enumerate}
\item Rotate the scene slightly to the right of center so that you
can see the left side of the geometry.  In the Viewpoints panel of
the Motion/View/Render dialog change {\tt new view}\ to
{\tt left}\
    then click on the \frameit{Add}\ button.
\item Rotate the scene slightly to the left of center so that you
can see the right side of the geometry.  In the Viewpoints panel
of the Motion/View/Render dialog box change {\tt new view}\ to
{\tt right}\
    then click on the \frameit{Add}\ button.
\item Click the \frameit{Save Settings}\ button.
\end{enumerate}

An .ini file has now been saved with two custom view points defined named left and right.

\subsubsection{Preliminary Steps - Defining clip planes and creating additional .ini files}

Defining the left clipping plane.
\begin{enumerate}
\item Click on the Clip Blockages + Data radio button, \item
change the {\em Clip Lower x}\ value 0.5 after checking the
checkbox next to edit field. \item Save an .ini file named
script\_test\_left.ini by entering the text {\tt left}\ in the
suffix field of the Config files section of the
Scripts/Config dialog box. \item Click on the \frameit{Set}\ button
then the \frameit{Save script\_test\_left.ini}\ button.
\end{enumerate}

Defining the right clipping plane.
\begin{enumerate}
\item Click on the Clip Blockages + Data radio button, \item
change {\em Clip Upper x}\ value 1.0 after checking checkbox next
to edit field. \item Save an .ini file named
script\_test\_right.ini by entering the text {\tt right}\ in the
suffix field of the Config files section of the
Scripts/Config dialog box. \item Click on the \frameit{Set}\ button
then the \frameit{Save script\_test\_right.ini}\ button.
\end{enumerate}

Two .ini files named {\tt scripts\_test\_left.ini}\
and {\tt scripts\_test\_right.ini}\ have
now been created.
\subsubsection{Recording the Script}
The script may be recorded now that the .ini files and viewpoints
have been created.  The following steps reference the
Scripts/Config dialog box.

\begin{enumerate}
\item Click on the \frameit{Start Recording}\ button \item Load
the $y=0.8$ temperature slice from the {\em Load/Unload}\ menu. \item
Generate the left image
\begin{enumerate}
\item Select the script\_test\_left.ini file and click on Load
\item Select the left view from the View menu. \item Set the time
to 5.0 \item Set the render suffix to left\_05 and press the
\frameit{Render}\ button
\end{enumerate}
\item Generate the center image
\begin{enumerate}
\item Select the script\_test.ini file and click on Load \item
Select external from the View menu. \item Set the time to 10.0
\item Set the render suffix to right\_10 and press the
\frameit{Render}\ button
\end{enumerate}
\item Render the right image
\begin{enumerate}
\item Select the script\_test\_right.ini file and click on the
\frameit{Load}\ button \item Select the right view from the View
menu. \item Set the time to 15.0 \item Set the render suffix to
right\_15 and press the \frameit{Render}\ button
\end{enumerate}
\item Click on the \frameit{Stop Recording}\ button
\end{enumerate}



\begin{figure}[bph]
\fdsinput{script_test.ssf}
\caption{Script commands generated using the Smokeview script recorder option.}
\label{figscripttext}%
\end{figure}


\begin{figure}[bph]
\begin{center}
\begin{tabular}{ccc}
 \includegraphics[width=2.25in]{SCRIPT_FIGURES/script_test_left_05}&
 \includegraphics[width=2.25in]{SCRIPT_FIGURES/script_test_center_10}&
 \includegraphics[width=2.25in]{SCRIPT_FIGURES/script_test_right_15}\\
  \end{tabular}
\end{center}
 \caption{Smokeview images generated using script detailed in
 Fig. \ref{figscripttext}}
\label{figscriptimages}%
\end{figure}

%---------------------------------------------------------------------------------
%---------------------------------------------------------------------------------
\section{Script Glossary}

This section contains documentations for the script commands.
Commands fall into three logical categories.  Commands
to load data files, commands to position the scene in both time and space
and commands to output the scene to image files.

%%-----------------------------------------------------------------------
\subsection{Loading and Unloading Files}
\subsubsection{3D Smoke Files}
\blist

%%+++++++++++++++++++++++
\hitemssf{LOAD3DSMOKE}
Load a 3D smoke file.  The types supported are
SOOT DENSITY, HRRPUV,\\
TEMPERATURE and CARBON DIOXIDE DENSITY.
Usage:
\begin{lstlisting}
LOAD3DSMOKE
  type (char)
\end{lstlisting}

%%+++++++++++++++++++++++
\hitemssf{LOADVOLSMOKE}\ Load files needed to view volume rendered
smoke.  One may either load files for all meshes or for one
particular mesh.  Usage:
\begin{lstlisting}
LOADVOLSMOKE
 mesh number (-1 for all meshes) (int)
 \end{lstlisting}


%%+++++++++++++++++++++++
\hitemssf{LOADVOLSMOKEFRAME}\ Load a volume rendered smoke frame.
Usage:
\begin{lstlisting}
LOADVOLSMOKEFRAME
 mesh_index (int) frame_index (int)
\end{lstlisting}
If the mesh\_index is positive then volume rendered smoke for that
mesh index is loaded. If the mesh\_index is negative then volume
rendered smoke for all meshes is loaded.

\elist

\subsubsection{Boundary Files}

\blist

%%+++++++++++++++++++++++
\hitemssf{LOADBOUNDARY, LOADBONDARYM}\ Load a boundary file of a particular type.
The type is the same as what Smokeview displays in the Load menus for boundary files.
{\tt LOADBOUNDARY}\ loads boundary files for all meshes, {\tt LOADBOUNDARYM}\
loads a boundary file for one mesh.
Usage:
\begin{lstlisting}
LOADBOUNDARY
   type (char)
LOADBOUNDARYM
   type (char)
   mesh_number (int)
\end{lstlisting}

%%+++++++++++++++++++++++
\hitemssf{SETBOUNDBOUNDS}
Use {\tt SETBOUNDBOUNDS}\ to set minimum and maximum bounds for boundary files.
{\tt SETBOUNDBOUNDS}\ must be used before a boundary file is loaded.
\begin{lstlisting}
  SETBOUNDBOUNDS
  ivalmin (int) valmin (float) ivalmax (int) valmax quantity_label (char)
\end{lstlisting}

The usage of the
{\tt ivalmin, valmin, ivalmax, valmax}\ and {\tt quantity\_label}
parameters is the same as those used for the
{\tt V2\_BOUNDARY}\ ini keyword and are describe in Appendix \ref{section:timedatabounds}.


\elist


\subsubsection{General Files}
\blist

%%+++++++++++++++++++++++
\hitemssf{LOADFILE}
Use LOADFILE to load a particular file.  Smokeview will determine
what kind of file it is (3d smoke, slice, etc.) and call the
appropriate routine to load the data.

Use other LOAD commands to load files of the specified type for all meshes.
Usage:
\begin{lstlisting}
LOADFILE
  file (char)
\end{lstlisting}

%%+++++++++++++++++++++++
\hitemssf{LOADINIFILE}
Use LOADINIFILE to load a configuration ({\tt .ini}) file.
Usage:
\begin{lstlisting}
LOADINIFILE
  file (char)
\end{lstlisting}

%%+++++++++++++++++++++++
\hitemssf{UNLOADALL}Unload all data files currently loaded.
Usage:
\begin{lstlisting}
UNLOADALL
\end{lstlisting}
\elist

%%-----------------------------------------------------------------------
\subsubsection{HVAC Files}
\blist

%%+++++++++++++++++++++++
\hitemssf{HIDEHVACVALS}This keyword hides hvac duct and node values.
Usage:
\begin{lstlisting}
HIDEHVACVALS
\end{lstlisting}

%%+++++++++++++++++++++++
\hitemssf{SHOWHVACDUCTVAL}This keyword shows the specified hvac duct quantity.
Usage:
\begin{lstlisting}
SHOWHVACDUCTVAL
  quantity (char)
\end{lstlisting}

%%+++++++++++++++++++++++
\hitemssf{SHOWHVACNODEVAL}This keyword shows the specified hvac node quantity.
Usage:
\begin{lstlisting}
SHOWHVACNODEVAL
  quantity (char)
\end{lstlisting}
\elist

%%-----------------------------------------------------------------------
\subsubsection{Slice and Vector Slice Files}
\blist

%%+++++++++++++++++++++++
\hitemssf{LOADSLCF}Load a slice file.
This script command uses one or more of the keywords ID, PBX, PBY, PBZ, QUANTITY, CELL\_CENTERED and VECTOR to specify slice files to load.
These keywords are defined in the same way as when used with an \&SLCF namelist in an FDS input file.
ID specifies an identifier.  PBX, PBY, PBZ specify a position and orientation.
PB3D, not a \&SLCF namelist keyword, is used to load 3D slice files.
CELL\_CENTERED indicates that the slice is cell centered as opposed to node centered
and VECTOR indicates that slice files for 3 components of velocity in addition to a slice file for QUANTITY is to be loaded.

The ID keyword may be used to specify a slice file instead of QUANTITY and one of PBX, PBY, PBZ and PB3D.
QUANTITY contains a string enclosed in single quotes such as
TEMPERATURE. U-VELOCITY, or SOOT DENSITY {\em etc}.
Blanks are allowed in an ID or QUANTITY value.  They must be enlosed by single not double quotes.
Any value used in an FDS input file may be used here. You may use blanks or commas to separate keywords.
Note, that these files need to exist in order to be loaded.

In the following examples assume that your input file has these  \&SLCF lines.

\begin{lstlisting}
&SLCF PBX=1.0, QUANTITY='TEMPERATURE', VECTOR=T ID='x slice 1' /
&SLCF PBY=2.0, QUANTITY='TEMPERATURE', CELL_CENTERED=T ID='y slice 1' /
&SLCF XB=0.0,1.0,0.0,1.0,0.0,1.0, QUANTITY='TEMPERATURE' /
\end{lstlisting}


\noindent To load the first slice without specifying the location or quantity, use the ID with
\begin{lstlisting}
LOADSLCF
 ID='x slice 1'
\end{lstlisting}

\noindent To load the same slice by specifying a quantity and location use:
\begin{lstlisting}
LOADSLCF
 PBX=1.0, QUANTITY='TEMPERATURE'
\end{lstlisting}

\noindent The first \&SLCF line has VECTOR=T parameter set so has velocity components for vector slice files.
To load a vector slice file use:
\begin{lstlisting}
LOADSLCF
 ID='x slice 1' VECTOR=T
\end{lstlisting}

\noindent or

\begin{lstlisting}
LOADSLCF
 PBX=1.0, QUANTITY='TEMPERATURE' VECTOR=T
\end{lstlisting}

\noindent The second \&SLCF namelist is cell centered.  To load a cell centered slice file use
\begin{lstlisting}
LOADSLCF
 ID='y slice 1' CELL_CENTERED=T
\end{lstlisting}

The third \&SLCF namelist is a 3D slice file.  The LOADSLCF does not support the XB keyword.
To load a 3D slice use PB33D as in the following.

\begin{lstlisting}
LOADSLCF
 QUANTITY='TEMPERATURE' PB3D=T
\end{lstlisting}

%%+++++++++++++++++++++++
\hitemssf{LOADSLICE, LOADSLICEM} Loads slice files.  The LOADSLICE keyword loads slice files for
all meshes where the slice occurs.  The LOADSLICDM keyword loads the slice for a specified mesh. LOADSLICE usage:
\begin{lstlisting}
LOADSLICE
  quantity (char)
  1/2/3 (int)dir  (float)position
\end{lstlisting}

\noindent LOADSLICEM usage:
\begin{lstlisting}
LOADSLICEM
  quantity (char)
  1/2/3 (int)dir  (float)position
  mesh number (int)
\end{lstlisting}

%%+++++++++++++++++++++++
\hitemssf{LOADVFILE}
Use LOADVFILE to load a particular vector slice file.  Smokeview will load the file
specified along with the corresponding U, V and W velocity slice files if they are available.
Usage:
\begin{lstlisting}
LOADVFILE
  file (char)
\end{lstlisting}

%%+++++++++++++++++++++++
\hitemssf{LOADVSLICE, LOADVSLICEM}Loads vector slice files.  The LOADVSLICE keyword loads vector slice files for all meshes where the vector slice occurs.  The LOADVSLICDM keyword loads the vector slice for a specified mesh. LOADVSLICE usage:
\begin{lstlisting}
LOADVSLICE
  quantity (char)
  1/2/3 (int)dir  (float)position
\end{lstlisting}

\noindent LOADVSLICEM usage:
\begin{lstlisting}
LOADVSLICEM
  quantity (char)
  1/2/3 (int)dir  (float)position
  mesh number (int)
\end{lstlisting}

%%+++++++++++++++++++++++
\hitemssf{OUTPUTSLICEDATA}
Use {\tt OUTPUTSLICEDATA}\ to output slice data to a spread
sheet file whenever the `r' key is pressed to render an image.
Usage:
\begin{lstlisting}
  OUTPUTSLICEDATA
  flag (int)
\end{lstlisting}

where flag is 1 to turn on slice data output and 0 to turn it off.

%%+++++++++++++++++++++++
\hitemssf{SETSLICEAVERAGE}
Use {\tt SETSLICEAVERAGE}\ to average slice data when loading.
{\tt SETSLICEAVERAGE}\ must be used before a slice
file is loaded. Usage:
\begin{lstlisting}
  SETSLICEAVERAGE
  flag (int) interval (float)
\end{lstlisting}

where flag is 1 to turn slice averaging on and 0 to turn it off.
The parameter interval is specified the time interval used
to perform the average.

%%+++++++++++++++++++++++
\hitemssf{SETSLICEBOUNDS}
Use {\tt SETSLICEBOUNDS}\ to set minimum and maximum bounds for slice files.
{\tt SETSLICEBOUNDS}\ must be used before a slice
file is loaded.
\begin{lstlisting}
  SETSLICEBOUNDS
  ivalmin (int) valmin (float) ivalmax (int) valmax quantity (char)
\end{lstlisting}

The usage of the
{\tt ivalmin, valmin, ivalmax, valmax}\ and {\tt quantity\_label}
parameters is the same as those used for the
{\tt V2\_SLICE}\ ini keyword and are describe in Appendix \ref{section:timedatabounds}.

\elist

\subsubsection{Particle Files}
\blist

%%+++++++++++++++++++++++
\hitemssf{LOADPARTICLES}Load particle files.  Only particle files created with FDS version 5 or
later are supported.
Usage:
\begin{lstlisting}
LOADPARTICLES
\end{lstlisting}

%%+++++++++++++++++++++++
\hitemssf{PARTCLASSCOLOR}Show a particular particle class.  Class names supported for a given
run are displayed in the Particle Class Smokeview menu.
Usage:
\begin{lstlisting}
PARTCLASSCOLOR
   color (char)
\end{lstlisting}

%%+++++++++++++++++++++++
\hitemssf{PARTCLASSTYPE}Show a particular particle type.  Type names supported for a given
run are displayed in the Particle Type Smokeview menu.
Usage:
\begin{lstlisting}
PARTCLASSTYPE
   type (char)
\end{lstlisting}
\elist

\subsubsection{PLOT3D Files}
\blist

%%+++++++++++++++++++++++
\hitemssf{LOADPLOT3D}Load a Plot3D file for a given mesh at a specified time.
Usage:
\begin{lstlisting}
LOADPLOT3D
  mesh number (int) time (float)
\end{lstlisting}
where
\begin{itemize}
\item mesh index - is an integer from 0 to the number of meshes.  If 0 is specified then all meshes at a given time are loaded.
If 1 to the number of meshes is specified then the mesh with that mesh index is loaded.
\item time - time within 0.5 s of mesh to load
\end{itemize}

%%+++++++++++++++++++++++
\hitemssf{PLOT3DPROPS}Specifies Plot3D plot properties that
apply to all Plot3D plots currently being displayed.
Usage:
\begin{lstlisting}
PLOT3DPROPS
  variable_index (int) showvector (0/1) (int) vector_length_index
  (int) display_type (int) vector_length (float)
\end{lstlisting}
where
\begin{itemize}
\item variable\_index - is an integer from 1 to the number of
Plot3D file components (usually 5),
\item showvector - is 1 to draw vectors, 0 otherwise
\item vector\_length\_index, is an integer index from 0 to 6
pointing to an internal Smokeview array used to determine vector length.
\item display\_type - is 0 for stepped contours, 1 for line
contours and 2 for continuous contours
\item vector\_length - if vector\_length\_index is negative
then Smokeview uses vector\_length set the PLOT3D vector length
\end{itemize}

%%+++++++++++++++++++++++
\hitemssf{SHOWPLOT3DDATA}Specifies a particular Plot3D plot to be
displayed (mesh number, orientation, variable, whether visible or not, and position)
Usage:
\begin{lstlisting}
SHOWPLOT3DDATA
  mesh number (int) plane orientation (int)  plot3d variable (int) show/hide (0/1) (int) position (float)
\end{lstlisting}
where
\begin{itemize}
\item mesh number - the mesh number (ranges from 1 to the number of meshes),
\item orientation - direction or orientation of the plane being plotted, 1 for YZ planes, 2 for XZ planes and 3 for XY planes .
An orientation of 4 indicates that an isosurface is drawn.
\item Plot3D variable index (1 to 6).  There are only 5 components in a Plot3D file.
If 6 is specified then smokeview constructs a velocity component if any of the U, V or W velocity components are present.
\item display - 0 if Plot3D plane is hidden, 1 if it is displayed
\item position - position of Plot3D plane.  If an orientation of 4 is specified then position is an integer index representing the isosurface level.
\end{itemize}
\elist

\subsubsection{Isosurface Files}
\blist

%%+++++++++++++++++++++++
\hitemssf{LOADISO, LOADISOM}Load iso-surface files for a specified type for all meshes if LOADISO is used and for
a specified mesh if LOADISOM is used.
The type is the same as what Smokeview displays in the Load menus for iso-surface files.
Usage:
\begin{lstlisting}
LOADISO
  type (char)

LOADISOM
  type (char)
  mesh number (int)
\end{lstlisting}

\elist

%%-----------------------------------------------------------------------
\subsection{Showing/Hiding Device}

\blist

%%+++++++++++++++++++++++
\hitemssf{HIDEALLDEVS, SHOWALLDEVS}Hides or shows all devices.
Usage:
\begin{lstlisting}
 HIDEALLDEVS
 SHOWALLDEVS

\end{lstlisting}

%%+++++++++++++++++++++++
\hitemssf{HIDEDEV, SHOWDEV}Hide or shows a device with specified device id.
Usage:
\begin{lstlisting}
 HIDEALLDEVS
   device_id

 SHOWDEV
   device_id
\end{lstlisting}

%%+++++++++++++++++++++++
\hitemssf{OUTPUTSMOKESENSORS}Output smoke sensor data to a csv file named casename\_ss.csv.
Usage:
\begin{lstlisting}
 OUTPUTSMOKESENSORS
\end{lstlisting}


\elist
%%-----------------------------------------------------------------------
\subsection{Controlling Colorbars}

\blist

%%+++++++++++++++++++++++
\hitemssf{CBARFLIP}\ Display the colorbar in the opposite orientation from which it was defined.
Usage:
\begin{lstlisting}
 CBARFLIP
\end{lstlisting}

%%+++++++++++++++++++++++
\hitemssf{CBARNORMAL}Display the colorbar in the same orientation as it was defined.
Usage:
\begin{lstlisting}
 CBARNORMAL
\end{lstlisting}

%%+++++++++++++++++++++++
\hitemssf{HILIGHTMINVALS}Hilight data values below the colorbar minimum.
Usage:
\begin{lstlisting}
 HILIGHTMINVALS
  0/1 r g b
\end{lstlisting}
where r, g, b is the color used to hilight values below the colorbar minimum.

%%+++++++++++++++++++++++
\hitemssf{HILIGHTMAXVALS}Hilight data values above the colorbar maximum.
Usage:
\begin{lstlisting}
 HILIGHTMAXVALS
  0/1 r g b
\end{lstlisting}
where r, g, b is the color used to hilight values above the colorbar maximum.

%%+++++++++++++++++++++++
\hitemssf{SETCBAR}Set the colorbar to the specified quantity.
Usage:
\begin{lstlisting}
 SETCBAR
   quantity (char)
\end{lstlisting}

%%+++++++++++++++++++++++
\hitemssf{SHOWCBAREDIT, HIDECBAREDIT}Show or hide the colorbar edit dialog box.
Usage:
\begin{lstlisting}
 SHOWCBAREDIT

 HIDECBAREDIT
\end{lstlisting}

%%+++++++++++++++++++++++
\hitemssfN{SETCBARLAB, SETCBARRGB}

\elist

%%-----------------------------------------------------------------------
\subsection{Touring}

\blist

%%+++++++++++++++++++++++
\hitemssf{LOADTOUR}Load a tour of a given name.
Usage:
\begin{lstlisting}
LOADTOUR
 type (char)
\end{lstlisting}

%%+++++++++++++++++++++++
\hitemssf{SETTOURKEYFRAME}Set view position to the tour keyframe closest to the specified time.
Usage:
\begin{lstlisting}
SETTOURKEYFRAME
 time (float)
\end{lstlisting}

%%+++++++++++++++++++++++
\hitemssf{SETTOURVIEW}Set tour view properties.
Usage:
\begin{lstlisting}
SETTOURVIEW
 edit tour (0/1) view from tour (0/1) show avatar (0/1)
\end{lstlisting}

%%+++++++++++++++++++++++
\hitemssf{UNLOADTOUR}Unload a tour.
\begin{lstlisting}
UNLOADTOUR
\end{lstlisting}
\elist

%%-----------------------------------------------------------------------
\subsection{Controlling the Scene}

\blist

%%+++++++++++++++++++++++
\hitemssf{EXIT}Cause Smokeview to quit.
Usage:
\begin{lstlisting}
 EXIT
\end{lstlisting}


%%+++++++++++++++++++++++
\hitemssf{GSLICEVIEW}Show general oriented slice plane, and/or triangles and normals used to represent slice.
Usage:
\begin{lstlisting}
 GSLICEVIEW
 show_gslice, show_triangles, show_triangulation, show_normals (all ints)
\end{lstlisting}

%%+++++++++++++++++++++++
\hitemssf{GSLICEPOS}Set position (xpos, ypos, zpos) of general oriented slice plane.
Usage:
\begin{lstlisting}
 GSLICEPOS
 xpos (float), ypos (float), zpos (float)
\end{lstlisting}


%%+++++++++++++++++++++++
\hitemssf{GSLICEORIEN}Set orientation (azimuth elevation orientation angles) of general oriented slice plane.
Usage:
\begin{lstlisting}
 GSLICEORIEN
 azimuth (float), elevation (float)
\end{lstlisting}

%%+++++++++++++++++++++++
\hitemssf{KEYBOARD}Passes a keyboard character to Smokeview
Usage:
\begin{lstlisting}
 KEYBOARD
  c
\end{lstlisting}
or
\begin{lstlisting}
 KEYBOARD
  ALT c
\end{lstlisting}
where {\tt c}\ is any keyboard character (recognized by Smokeview) and {\tt ALT}\ is the ALT key.

%%+++++++++++++++++++++++
\hitemssf{PROJECTION}Set the projection type, 1 for perspective, 2 for size preserving.
Usage:
\begin{lstlisting}
 PROJECTION
  (1/2)
\end{lstlisting}

%%+++++++++++++++++++++++
\hitemssf{SETCLIPMODE, SCENECLIP}Sets clipping mode 0 - clipping off, 1 - clip blockages and data,
2 - clip only blockages, 3 - clip only data.
Usage:
\begin{lstlisting}
 SCENECLIP
  (0/1/2/3)
\end{lstlisting}


%%+++++++++++++++++++++++
\hitemssf{SETCLIPX, SETCLIPY, SETCLIPZ}Set or unset the x, y, z min or max clipping plane positions.
Usage:
\begin{lstlisting}
 SETCLIPX
  0/1 min position (float) 0/1 max position (float)
\end{lstlisting}
The SETCLIPY and SETCLIPZ are defined similarly.

%%+++++++++++++++++++++++
\hitemssf{SETTIMEVAL}Set the time for displaying data to a specified value.
Usage:
\begin{lstlisting}
SETTIMEVAL
  time (float)
\end{lstlisting}

%%+++++++++++++++++++++++
\hitemssf{SETVIEWPOINT}Set a view point .  The view point must have been previously defined and
saved in an .ini file.
Usage:
\begin{lstlisting}
SETVIEWPOINT
  viewpoint (char)
\end{lstlisting}

%%+++++++++++++++++++++++
\hitemssf{XYZVIEW}Sets the position (x, y and z location) and view direction (azimuth and elevation angle in degrees).
Usge:
\begin{lstlisting}
XYZVIEW
 xpos ypos zpos az elev
\end{lstlisting}

%%+++++++++++++++++++++++
\hitemssf{VIEWXMIN,VIEWYMIN,VIEWZMIN}The {\tt VIEWXMIN}\ script keyword sets the view so that the scene
is viewed towards the XMIN direction.  It sets
the scene view location to (x,ymid,zmid)
and the scene view direction to $(-1.0,0.0,0.0)$ where
x is chosen so that the entire scene is in view and ymid, zmid
are the middle of the scene along y and z directions.  The {\tt VIEWYMIN}\ and {\tt VIEWZMIN}\ script
keywords are defined similarly.  Usage:
\begin{lstlisting}
VIEWXMIN
VIEWYMIN
VIEWZMIN
\end{lstlisting}

%%+++++++++++++++++++++++
\hitemssf{VIEWXMAX,VIEWYMAX,VIEWZMAX}The {\tt VIEWXMAX}\ script keyword sets the view so that the scene
is viewed towards the XMAX direction.  It sets
the scene view location to (x,ymid,zmid)
and the scene view direction to $(+1.0,0.0,0.0)$ where
x is chosen so that the entire scene is in view and ymid, zmid
are the middle of the scene along y and z directions.  The {\tt VIEWYMAX}\ and {\tt VIEWZMAX}\ script
keywords are defined similarly.  Usage:
\begin{lstlisting}
VIEWXMAX
VIEWYMAX
VIEWZMAX
\end{lstlisting}

%%+++++++++++++++++++++++
\hitemssf{XSCENECLIP, YSCENECLIP, ZSCENECLIP}\ Sets clipping planes along the X, Y or Z axis.
Portions of the scene before {\tt min}\
or after {\tt max}\ planes are hidden if the corresponding setmin or setmax are set to 1 respectively.
Usage:
\begin{lstlisting}
  XSCENECLIP
    setxmin xmin setxmax xmax

  YSCENECLIP
    setymin ymin setymax ymax

  ZSCENECLIP
    setzmin zmin setzmax zmax
\end{lstlisting}

\elist

%%-----------------------------------------------------------------------

\subsection{Rendering Images}


\blist

%%+++++++++++++++++++++++
\hitemssf{ISORENDERALL, LOADSLICERENDER, LOADSMOKERENDER}\ Render a sequence
of isosurface, slice or 3D smoke frames. As with {\tt RENDERALL}, this command by default
renders every frame starting with the first. One may also specify
a starting frame index (default: 0) and a skip value (default: 1)
indicating the difference in indices between rendered frames.

This command differs from the RENDERALL command by automatically
loading files but only one  frame (time
step) at a time as they are rendered.  This allows one to create
isosurface movies for data sets too large to be viewed within
Smokeview.

\begin{lstlisting}
ISORENDERALL
  skip first
  file name base (char)
\end{lstlisting}

\begin{lstlisting}
LOADSLICERENDER
  quantity (char)
  1/2/3 position(float)
  file name base (char)
  skip (int) first (int)
\end{lstlisting}

\begin{lstlisting}
LOADSLICERENDER
  quantity (char)
  1/2/3 position(float)
  file name base (char)
  skip (int) first (int)
\end{lstlisting}

%%+++++++++++++++++++++++
\hitemssf{MAKEMOVIE}Make a movie using ffmpeg (if it is installed) using jpeg or png frames
previously rendered.
Usage:
\begin{lstlisting}
MAKEMOVIE
  movie_name (char)
  frame_prefix (char)
  framerate (float)
\end{lstlisting}

%%+++++++++++++++++++++++
\hitemssf{MOVIETYPE}Specify the format of the movie generated with the MAKEMOVIE keyword
Usage:
\begin{lstlisting}
MOVIETYPE
  AVI/MP4/WMV (char)
\end{lstlisting}

%%+++++++++++++++++++++++
\hitemssf{RENDER360ALL}Render frames using 360 degree format (all view directions are recorded) .
Every skip frames are rendered.  Usage:
\begin{lstlisting}
RENDER360ALL
 skip (int)
 file name base (char) (or blank to use smokeview default)
\end{lstlisting}

%%+++++++++++++++++++++++
\hitemssf{RENDERALL}\ Render a sequence of frames.  By default this
command renders every frame starting with the first. One may also
specify a starting frame index (default: 0) and a skip value
(default: 1) indicating the difference in indices between rendered
frames. Usage:
\begin{lstlisting}
RENDERALL
  skip first
  file name base (char) (or blank to use the Smokeview default)
\end{lstlisting}

The command
\begin{lstlisting}
RENDERALL
  1 0
  casename
\end{lstlisting}
would render all frames while the command
\begin{lstlisting}
RENDERALL
  3 1
  casename
\end{lstlisting}
renders every third frame starting with the second (index 1).

%%+++++++++++++++++++++++
\hitemssf{RENDERCLIP}\ Specify clip offsets in pixels when
rendering a scene.  Usage:
\begin{lstlisting}
RENDERCLIP
  flag left right bottom top
\end{lstlisting}
where clipping is turn on if {\tt flag}\ is set to 1 and
turned off if {\tt flag}\ is set to 0.

%%+++++++++++++++++++++++
\hitemssf{RENDERDIR}\ Specify a
directory where rendered files should go. Usage:
\begin{lstlisting}
RENDERDIR
  directory name
\end{lstlisting}
Smokeview automatically converts directory separators (`/' for Linux/Mac systems
and `//' for Windows systems) to the separator appropriate for the host system.

%%+++++++++++++++++++++++
\hitemssf{RENDERONCE}\ Render the current scene.
Usage:
\begin{lstlisting}
RENDERONCE
 file name (optional)
\end{lstlisting}
Smokeview will assign the filename automatically if the
entry after the {\tt RENDERONCE}\ keyword is blank.

%%+++++++++++++++++++++++
\hitemssf{RENDERDOUBLEONCE}\ Render the current scene at double resolution.
Usage:
\begin{lstlisting}
RENDERDOUBLEONCE
 file name (optional)
\end{lstlisting}
As with {\tt RENDERONCE}, Smokeview will assign the filename
automatically if the entry after the {\tt RENDERDOUBLEONCE}\ keyword is blank.

%%+++++++++++++++++++++++
\hitemssf{RENDERHTMLALL, RENDERHTMLONCE}Render scene in html format.
RENDERHTMLALL renders all times frames, RENDERHTMLONCE renders the current time frame.
Usage:
\begin{lstlisting}
RENDERHTMLALL
  base file name (char)

RENDERHTMLONCE
  base file name (char)
\end{lstlisting}

%%+++++++++++++++++++++++
\hitemssf{RENDERHTMLDIR}Set the directory containing html files to be rendered. Usage:
\begin{lstlisting}
RENDERHTMLDIR
  directory name (char) (where rendered files will go)
\end{lstlisting}

%%+++++++++++++++++++++++
\hitemssfN{RENDERHTMLGEOM}
\hitemssfN{RENDERHTMLOBST}
\hitemssfN{RENDERHTMLSLICECELL}
\hitemssfN{RENDERHTMLSLICENODE}

%%+++++++++++++++++++++++
\hitemssf{RENDERSIZE}Set the width and heightt of rendered images.
Usage:
\begin{lstlisting}
RENDERSIZE
 width height (int)
\end{lstlisting}

%%+++++++++++++++++++++++
\hitemssf{RENDERSTART}Set the first frame and the number of frames to skip for the frames to be rendered. Usage:
\begin{lstlisting}
RENDERSTART
 start_frame (int) skip_frame (int)
\end{lstlisting}

%%+++++++++++++++++++++++
\hitemssf{RENDERTYPE}Set the type of image to be rendered to jpg or png. Usage:
\begin{lstlisting}
RENDERTYPE
 jpg or png  (char)
\end{lstlisting}

%%+++++++++++++++++++++++
\hitemssf{VOLSMOKERENDERALL}\ Render a sequence of volume rendered
frames. As with {\tt RENDERALL}, this command by default command
renders every frame starting with the first. One may also specify
a starting frame index (default: 0) and a skip value (default: 1)
indicating the difference in indices between rendered frames.

This command differs from the RENDERALL command by automatically
loading volume rendered smoke files but only one  frame (time
step) at a time as they are rendered.  This allows one to create
volume rendered movies for data sets too large to be viewed within
Smokeview.

\begin{lstlisting}
VOLSMOKERENDERALL
  skip first
  file name base (char) (or blank to use the Smokeview default)
\end{lstlisting}

\elist

%%-----------------------------------------------------------------------

\subsection{Miscellaneous}

\blist

%%+++++++++++++++++++++++
\hitemssf{GPUOFF}Turn off the GPU if is active and being used (in smoekview)
\begin{lstlisting}
  GPUOFF
\end{lstlisting}

%%+++++++++++++++++++++++
\hitemssf{LABEL}Add a comment to script output. Usage:
\begin{lstlisting}
LABEL
  text
\end{lstlisting}

%%+++++++++++++++++++++++
\hitemssf{RGBTEST}Test color value at an $(x,y,z)$ location. Usage:
\begin{lstlisting}
RGBTEST
  x y z r g b delta
\end{lstlisting}
\noindent where r, g and b are red green and blue color components each ranging from 0 to 255 and delta is an error tolerance.
If any color component difference is found to be greater than delta, a warning message is output.

%%+++++++++++++++++++++++
\hitemssf{SETTOURKEYFRAME}Set the time position along the current tour to time. Usage:
\begin{lstlisting}
SETTOURKEYFRAME
  time (float)
\end{lstlisting}

\elist


%---------------------------------------------------------------------------------
%------------------------ Advanced Features --------------------------------------
%---------------------------------------------------------------------------------

\part{Miscellaneous Topics}


\chapter{Customizing Colorbars}
\label{chap:colorbar}
%---------------------------------------------------------------------------------
%---------------------------------------------------------------------------------
\section{Overview}
\begin{figure}[bph]
\begin{center}
\begin{tabular}{cc}
\includegraphics[width=3.5in]{SCRIPT_FIGURES/colorbar2_rainbow}&
\includegraphics[width=3.5in]{SCRIPT_FIGURES/colorbar2_blue_green_red}\\
a) Rainbow&b) blue-$>$green-$>$red\\
\includegraphics[width=3.5in]{SCRIPT_FIGURES/colorbar2_split}&
\includegraphics[width=3.5in]{SCRIPT_FIGURES/colorbar2_black_white}\\
c) split&d) black-$>$white
\end{tabular}
\end{center}
\caption[Colorbar Examples]{
Colorbar Examples.  Each example shows a 1D strip
of colors to the right and a corresponding path in RGB space to the left
where the x, y, z path position
represent the red, green, blue color component of the colorbar.
Each color corresponds to a
data value.
The `blue->red split' colorbar is discontinuous useful for highlighting data
values at the discontinuity.
The other three colobars are continuous.
}\
\label{figCOLORBAR_EXAMPLES}
\end{figure}

A colorbar is used to associate
data with color.  A colorbar is represented in Smokeview as a table of 256 rows and 3 columns
of red, green and blue color components, each component ranging from 0 to 255.
When constructing a colorbar, it is useful to represent these components as spatial coordinates within a cube
where the distance along a cube side corresponds to a color component value.  A colorbar
may then be thought of as a path within this cube.
The lower left front corner
is colored black (all components are 0) and the upper right back corner is colored
white (all components are 255).  Other corners are colored red, green, blue, cyan, magenta and
yellow.
Figure \ref{figCOLORBAR_EXAMPLES}\ gives several examples of colorbars defined
by Smokeview.

To construct a colorbar, one defines a set of nodes forming a path
within the color cube.  Smokeview then interpolates colors between these nodes
to form the colorbar.  Though most colorbar paths used by smokeview are continuous, a colorbar
path need not be.  A colorbar path may have a large break or jump (split colorbar)
or an abrupt change of direction (divergent colorbar).  Split colorbars
are useful for hilighting regions in a simulation with a particular
property, for example where the temperature exceeds the boiling point of
water or in a terrain map where a shoreline with zero elevation occurs.
Figure \ref{figCOLORBAR_EXAMPLES}c
gives an example of a colorbar with a break.  This colorbar jumps in the
middle from  blue to  green.

A goal in constructing a colorbar is for equal changes in data to correspond to
equal changes in color. Problems occur when large changes in data correspond
to small changes in color. This results in flat spots in the colorbar with details in the data  obscured.
The opposite problem occurs when small changes in data correspond to large changes in color.
Features or artifacts occur in the visualization that are not present in the data.
One can minimize these problems by using perceptually uniform colorbars\cite{kovesi2015good}.
To aid in constructing colorbars with this property, the CIELab colorspace is used to interpolate
between color nodes and to choose color spacing between nodes
since this color space was designed so that equal distances between data correspond to equal differences in color perception (greater the data distance greater the perceived color difference).
Colorbars (except for those colorbars that are discontinuous)
are generated so that the distances between each point in CIELab space are equal.

Colorbars are grouped in smokeview menus as rainbow, linear and divergent.
Other groups used are original (colorbars that existed in the previous versions of smokeview),
deprecated (colorbars likely to be
removed in future versions of smokeview) and user (colorbars created by users). Several of the rainbow, linear and divergent
colorbars were obtained from Ref. \cite{ColorCET}.  All of these colorbars except for those with split in the name were adjusted
so that they are perceptually uniform, distances between adjacent colorbar colors in CIELab space are the same.

\begin{figure}[bph]
\begin{center}
\includegraphics[width=4.236111in]{\SMVfigdir/figCOLORBAR_EDITOR}
\end{center}
\caption[Colorbar node editing.]{ Colorbar node editing.
Dialog box for editing simple colorbars (1 to 5 nodes) or colorbars with more nodes.
One may create or edit an existing colorbar  by
specifying red, green and/or blue node components each ranging from 0 to 255.  }\ \label{figCOLORBAR}
\end{figure}


%---------------------------------------------------------------------------------
%---------------------------------------------------------------------------------
\section{Using the Colorbar Editor}
\label{section:colorbar}

The Colorbar Editor dialog box, illustrated in Figures \ref{figCOLORBAR},
is opened by selecting the {\em Dialogs$>$View$>$Edit colorbar}\ menu entry. When this menu item is
selected, a spatial representation of the currently selected
colorbar is displayed along with the Colorbar Editor dialog box.
Simple colorbars with up to 5 nodes can be edited using the 1->5 tab.
The General(2->256) tab is used to edit
colorbars with more nodes or if one wishes to delete or add nodes.
\begin{figure}[bph]
\begin{center}
\includegraphics[width=5.5in]{\SMVfigdir/figCOLORBAR_EDITOR2}
\end{center}
\caption[Colorbar Editor display.]{Colorbar Editor display.
Dialog box for specifying the color space (RGB or CIELab) used to display a colorbar or to
convert between RGB and CIELab color coordinates.
}\ \label{figCOLORBAR2}
\end{figure}
The Display tab illustrated in Figure \ref{figCOLORBAR2} is used to
change the coordinate system used to view the colorbar (RGB or CIELab),
to toggle between colorbars allowing one to more easily view colorbar differences,
to revert a colorbar back to its original state and
to output a colorbar to a csv file for examining in a spreadsheet program.

A colorbar is represented visually in several ways.  First, as a series
of colored nodes and lines within a cube. The cube has three axes which are
red, green and blue if the RGB color space is selected or
or L (lightness), a (red/green) and b (yellow/blue) if the CIELab color space is selected.
The nodes and lines are a spatial representation of the colors.
The color components (whether RGB or CIELab) are mapped to x, y, z spatial coordinates.
The second way a colorbar is represented is as a rectangle with a series of colored squares and numbered indices displayed along the side. This rectangle is equivalent to the colorbar
displayed beside a regular Smokeview scene.  The numbered indices indicate
a position within the colorbar , from 0 to 255, where the node color occurs.
Once the node indices and colors are defined, Smokeview interpolates between them to
form a table of colors. This table has  256 rows and 3 columns.

The Colorbar Editor dialog box contains a list of colorbars
defined by Smokeview and others if defined by the user (and read in from the .ini file). A new
colorbar is created by selecting the \frameit{New}\ button. The
new colorbar is created initially with two nodes.
The first node is black (0,0,0) and the second node is white (255,255,255).
Once created, it may be altered by
adding/deleting nodes with the \frameit{Add/Delete}\ buttons and
altering color with the red, green, blue spinners.

%---------------------------------------------------------------------------------
%---------------------------------------------------------------------------------
\section{Examples}
The examples in this section were created using the Edit Colorbar dialog box which is
opened by selecting the Dialogs>View>Edit Colorbar menu item.

%%-----------------------------------------------------------------------
\subsection{Constant Colorbar}

\newcommand{\cbsize}{3.25in}
\begin{figure}[bph]
\begin{center}
\begin{tabular}{cc}
\includegraphics[width=\cbsize]{SCRIPT_FIGURES/colorbar2_constant_rgb}&
\includegraphics[width=\cbsize]{SCRIPT_FIGURES/colorbar2_constant_lab}\\
RGB&CIELab\\
\end{tabular}
\end{center}
\caption[Constant Colorbar]{
Constant Colorbar.
}\
\label{fig:colorbar_constant}
\end{figure}

To create a constant colorbar, a colorbar with just one color:
\begin{enumerate}
 \setlength{\itemsep}{1pt}
  \setlength{\parskip}{0pt}
  \setlength{\parsep}{0pt}
\item Press the \frameit{New} button in the Colorbar panel. This creates a 2 node colorbar with colors ranging
from black to white.
\item Select the \frameit{constant} radio button.
\item Specify red, green and blue color components for node 1.
\item Specify a label, say constant\_01. Press the \frameit{Update label} button in the Colorbar panel.
\item Press the \frameit{Save settings} button. This saves the colorbar just created to the casename.ini
configuration file.  The next time this case is opened, the colorbar constant\_01 will appear as a user defined
colorbar.
\end{enumerate}

A constant colorbar using (0,0,255) for node 1 is illustrated in Figure \ref{fig:colorbar_constant}.
The steps for creating other types of colorbars are similar.

%%-----------------------------------------------------------------------
\subsection{Linear Colorbar}

\begin{figure}[bph]
\begin{center}
\begin{tabular}{cc}
\includegraphics[width=\cbsize]{SCRIPT_FIGURES/colorbar2_linear_rgb}&
\includegraphics[width=\cbsize]{SCRIPT_FIGURES/colorbar2_linear_lab}\\
RGB&CIELab\\
\end{tabular}
\end{center}
\caption[Linear Colorbar]{
Linear Colorbar.
}\
\label{fig:colorbar_linear}
\end{figure}

A linear colorbar is a colorbar with two nodes that transition smoothly
from one color at node 1 to a second color at node 5.
Node 1 and node 5 colors are converted from RGB to CIELab coordinates internally in smokeview.
These coordinates are then interpolated
linearly between nodes 1 and 5 to create 256 CIELab coorindates.
Since the linear interpolation occurs in CIELab space the paths are straight lines in the CIELab
images while curved in the RGB images.
Finally these 256 CIELab coordinates are converted back to RGB to form the colorbar.
To create a linear colorbar:
\begin{enumerate}
 \setlength{\itemsep}{1pt}
  \setlength{\parskip}{0pt}
  \setlength{\parsep}{0pt}
\item Press the \frameit{New} button in the Colorbar panel.
\item If not already selected, select the \frameit{linear (2 nodes)} radio button.
\item Specify red, green, and blue color components for nodes 1 and 5.
\item Specify a label, say linear\_01. Press the \frameit{Update label} button in the Colorbar panel.
\item Press the \frameit{Save settings} button to save the colorbar linear\_01 to the .ini file.
\end{enumerate}

A linear colorbar using (0,0,255) for node 1 and (255,0,0) for node 5 is illustrated in Figure \ref{fig:colorbar_linear}.

%%-----------------------------------------------------------------------
\subsection{Split Colorbar}

\begin{figure}[bph]
\begin{center}
\begin{tabular}{cc}
\includegraphics[width=\cbsize]{SCRIPT_FIGURES/colorbar2_split_rgb}&
\includegraphics[width=\cbsize]{SCRIPT_FIGURES/colorbar2_split_lab}\\
RGB&CIELab\\
\end{tabular}
\end{center}
\caption[Split Colorbar]{
Split Colorbar.
}\
\label{fig:colorbar_split}
\end{figure}

A split colorbar has a discontinuity or abrupt change in the middle.
To create a split colorbar:
\begin{enumerate}
 \setlength{\itemsep}{1pt}
  \setlength{\parskip}{0pt}
  \setlength{\parsep}{0pt}
\item Press the \frameit{New} button in the Colorbar panel.
\item Select the \frameit{split/divergent} radio button.
\item Specify red, green and blue color components for nodes 1, 2, 3 and 5.
Colors will smoothly transition between nodes 1 and 2, change abruptly between nodes 2 and 3
and smoothly transition between nodes 3 and 5.
\item Specify a label, say split\_01. Press the \frameit{Update label} button in the Colorbar panel.
\item Press the \frameit{Save settings} button to save the colorbar split\_01 to the .ini file.
\end{enumerate}

A split colorbar using (64,0,0) for node 1, (0,0,255) for node 2 (255,0,0) for node 3 and (255,255,0) for node 5 is illustrated in Figure \ref{fig:colorbar_split}.

%%-----------------------------------------------------------------------
\subsection{Divergent Colorbar}

\begin{figure}[bph]
\begin{center}
\begin{tabular}{cc}
\includegraphics[width=\cbsize]{SCRIPT_FIGURES/colorbar2_divergent_rgb}&
\includegraphics[width=\cbsize]{SCRIPT_FIGURES/colorbar2_divergent_lab}\\
RGB&CIELab\\
\end{tabular}
\end{center}
\caption[Divergent Colorbar]{
Divergent Colorbar.
}\
\label{fig:colorbar_divergent}
\end{figure}
A divergent colorbar is similar to a split colorbar.  It it continuous but has a sharp bend or change in direction in the middle when viewed as a path in a color space.
To create a divergent colorbar:
\begin{enumerate}
 \setlength{\itemsep}{1pt}
  \setlength{\parskip}{0pt}
  \setlength{\parsep}{0pt}
\item Press the \frameit{New} button in the Colorbar panel.
\item Select the \frameit{split/divergent} radio button.
\item Specify red, green and blue color components for nodes 1, 2, 3 and 5.
Select the same components for nodes 2 and 3.  The split and divergent colorbar dialog forces the discontinuity or sharp bend
to occur in the middle of the color bar.
These nodes should not occur in a straight line if you want a divergent colorbar.
\item Specify a label, say divergent\_01. Press the \frameit{Update label} button in the Colorbar panel.
\item Press the \frameit{Save settings} button to save the colorbar divergent\_01 to the .ini file.
\end{enumerate}

A divergent colorbar using (64,0,192) for node 1, (192,192,192) for node 2 and (192,0,64) for node 5 is illustrated in Figure \ref{fig:colorbar_divergent}.

%---------------------------------------------------------------------------------
%------------------------ Smokeview - Demonstrator Mode --------------------------
%---------------------------------------------------------------------------------

\chapter{Smokeview - Demonstrator Mode}
A simplified version of Smokeview may be invoked in order to present a fire
scenario for training or demonstration
purposes.  All actions are performed using one unified dialog box,
illustrated in Fig. \ref{figDEMO}.  This dialog box is opened for the
user at startup and allows the user to select data to be viewed, tours to
travel along, viewpoints to observe and scene manipulation to perform.
Smokeview loads data when it starts up.
The intent is to allow one not using Smokeview daily to more easily
make use of Smokeview's capabilities.


\begin{figure}[bph]
\begin{center}
\includegraphics[width=1.67777in]{\SMVfigdir/figDEMO}
\end{center}
\caption[Demonstrator dialog box.]{Demonstrator dialog box.
This dialog box allows the user to 1) switch between temperature,
oxygen and realistic views of the data,
select tours and viewpoints
 and to manipulate the scene using translations and rotations.}\ \label{figDEMO}
\end{figure}

In order to setup this demonstration mode, several tasks need to be performed.
The results of these tasks are
recorded in the {\tt casename.ini}\ file.
These tasks are detailed below.

\begin{enumerate}
  \item Define one or more tours that give the user an overview of the data or that highlight important
  aspects of the scenario.   Tours are setup using the Touring dialog box.
  \item Define one or more viewpoints that highlight some important
  aspect of the simulation scenario.  The viewpoint is defined by manipulating the scene
  as desired and then selecting the {\em View>Save}\ menu item.  The viewpoint label may
  be changed by using the Motion/View/Render dialog box.
  \item Pick the data to be viewed from a set of temperature and oxygen slice files
  and a set of 3D smoke and HRRPUV files.
  \begin{enumerate}
    \item Load the desired files into Smokeview.
    \item Select these files for {\em auto-loading}\ by selecting the {\em Auto Load Now}\ panel
    in the File/Bounds dialog box and pressing the
    \frameit{Save Auto Load File List}\ button.
    \item Compress these files with Smokezip using the {\tt -auto}\ option .
    This option will only compress files selected with Smokeview for {\em autoloading}.
    Note that compression can either be performed at a command line by typing
    {\tt Smokezip -auto casename}\ or by using the {\em Load/Unload>Compression}\ menu item.
      \end{enumerate}
  \item Save the settings and choices selected by saving a {\em casename.ini}
  configuration file for the case.
  \item Create a {\tt .svd}\ file by copying the {\tt casename.smv}\ to {\tt casename.svd}\ .
  \item Copy all the compressed files and the files: {\tt casename.ini}, {\tt casename.end}\ and {\tt casename.svd}\ file to a separate directory.  This directory is then is what one would distribute to be demonstrated.
\end{enumerate}

The demonstrator mode of Smokeview is activated by double-clicking on {\tt casename.svd}.
Smokeview treats this file just like {\tt casename.smv}\ except that it opens up
the Demonstrator Mode dialog box and hides the standard Smokeview menus.
Smokeview then
loads the selected slice, 3D smoke and HRR files and opens the dialog box illustrated in
Fig. \ref{figDEMO}.

This dialog box is used to toggle the data viewed by pressing the
\frameit{Smoke/Fire}, \frameit{Temperature}\ or \frameit{Oxygen}\
buttons. The scene may be manipulated by clicking the mouse in one
of the {\em arrow}\ buttons and dragging.  The scene may also be
manipulated as before by pressing the mouse within the scene and
dragging. Views and/or tours may be selected using the
corresponding {\em pull down}\ box.

%---------------------------------------------------------------------------------
%---------------------------------- Texture Maps ---------------------------------
%---------------------------------------------------------------------------------

\chapter{Texture Maps}\ \label{chapter:texturemaps}\ Texture mapping is a technique used by
Smokeview to make a scene appear more realistic by pasting images
onto obstructions or vents. For example, to apply a wood paneling
image to a wall, add the keywords {\tt
TEXTURE\_MAP='paneling.jpg', TEXTURE\_WIDTH=1., TEXTURE\_HEIGHT=2.
}\ to the {\tt \&SURF}\ line where {\bf paneling.jpg}\ is the JPEG
file containing the texture map and {\bf TEXTURE\_WIDTH}\ and {\bf
TEXTURE\_HEIGHT}\ are the characteristic dimensions of the texture
map in meters. Note that the image will not appear when Smokeview
first starts up. The user must select the texture maps using the
{\tt Show/Hide}\ menu.

One can create texture maps using a digital camera or obtain them
commercially.  The maps should be {\em seamless}\ so that no
breaks or seams appear when the maps are tiled on a blockage or
vent.  This is important, because Smokeview replicates the image
as often as necessary to cover the blockage or vent.

When the texture does have a pattern, for example windows or
bricks, the keyword {\tt TEXTURE\_ORIGIN}\ may be used to specify
where the pattern should begin.  For example,
\begin{lstlisting}
&OBST XB=1.0,2.0,3.0,4.0,5.0,7.0, SURF_ID='wood paneling',
      TEXTURE_ORIGIN=1.0,3.0,5.0 /
\end{lstlisting}
\noindent will apply paneling to an obstruction whose dimensions
are 1 m by 1 m by 2 m, such that the image of the paneling will be
positioned at the point (1.0,3.0,5.0). The default value of {\tt
TEXTURE\_ORIGIN}\ is (0,0,0), and the global default can be changed
by added a {\tt TEXTURE\_ORIGIN}\ statement to the {\tt MISC}\ line.

\begin{figure}[bph]
\centerline{\includegraphics[width=4.8125in]{SCRIPT_FIGURES/sillytexture}
}\ \caption [Texture map example.] {
Texture map example.  The same texture was applied to two different
blockages and a vent (with different widths) by assigning different {\tt TEXTURE\_WIDTH}\
parameters in the input file.
}\ \label{figTextures}
\end{figure}
Figure \ref{figTextures}\ shows a simple application of a texture
applied to two different blockages and a vent.  The same jpeg file
was used in two different {\tt \&SURF}\ lines so that the texture
could be stretched by differing amounts (using the {\tt
TEXTURE\_WIDTH}\ parameter.)  The FDS data file used to create
this Figure follows.

\fdsinput{sillytexture.fds}

%---------------------------------------------------------------------------------
%------------------ Using Smokeview to Debug FDS Input Files ---------------------
%---------------------------------------------------------------------------------

\chapter{Using Smokeview to Debug FDS Input Files}\ One of the most difficult
tasks in setting up an FDS input file is defining the geometry
(blockages, vent locations, etc.) properly. Smokeview may be used
to debug FDS input files by making short model runs and observing
whether blockages, vents and other geometric features of a model
run are located correctly. Blockages may then be created or
changed using a text editor and location information provided by
the Examine geometry dialog box called from the {\em \tt
Dialogs$>$View$>$Examine geometry}\ menu.

The following is a general procedure for identifying problems in
FDS input files. Assume that the FDS input data file is named {\tt
testcase1.fds}.
\begin{enumerate}
\item In the FDS input file, set the stop time to $0.0$ using {\tt
TWFIN=0.0}\ on the {\tt \&TIME}\ line. This causes FDS to read the
input file and create a {\tt .smv}\ file without  performing
lengthy startup calculations.

\item Run the FDS model (for details see the FDS User's
Guide~\cite{FDS_Users_Guide})

\noindent\ FDS creates a file named {\tt testcase1.smv}\ containing
information that Smokeview uses to visualize model.

\item To visualize the model, open {\tt testcase1.smv}\ with
Smokeview by either typing {\tt Smokeview testcase1}\ at a command
shell prompt or if on the PC by double-clicking the file {\tt
testcase1.smv}.

\item Make corrections to the FDS data file, if necessary. Using the {\tt COLOR}\ or {\tt
RGB}\ option of the
{\tt OBST}\ keyword to more easily identify blockages to be edited.
For example, to change a blockage's color to red use:
\begin{lstlisting}
&OBST XB=0.0,1.0,0.0,1.0,0.0,1.0, COLOR='RED' /
\end{lstlisting}
\noindent or
\begin{lstlisting}
&OBST XB=0.0,1.0,0.0,1.0,0.0,1.0 RGB=255,0,0 /
\end{lstlisting}

\noindent Save testcase1.fds file and go back to step 2.

\item If corrections are unnecessary, then change the {\tt TWFIN}
keyword back to the desired final simulation time, remove any
unnecessary FDS {\tt COLOR}\ keywords and run the case.
\end{enumerate}

%---------------------------------------------------------------------------------
%---------------------------------------------------------------------------------
\section{Examining Blockages}\  Blockages locations and SURF
properties may be examined by selecting the menu item {\em Examine
Blockages}\ which opens up the dialog box illustrated in Fig.
\ref{figEDIT}. Note, clipping planes need to be turned off when
using this dialog box. Associating unique colors with each surface
allows the user to quickly determine whether blockages are defined
with the proper surfaces. One can then verify that these modeling
elements have been defined and positioned as intended. Position
coordinates are displayed {\em snapped}\ to the nearest grid line
or as specified in the input file.

\begin{figure}[bph]
\centerline{
\centerline{\includegraphics[width=4.597in]{\SMVfigdir/figEDIT}\ }
}\ \caption{Examine blockages dialog box.}
 \label{figEDITBLOCKS}
\label{figEDIT}
\end{figure}

%---------------------------------------------------------------------------------
%--------------------------------- Making Movies ---------------------------------
%---------------------------------------------------------------------------------

\chapter{Making Movies}\ \label{section:movie}\ A movie may be made of a Smokeview animation
by converting the scene into a
series of still images, one image for each time step and then
combining the images into a movie file. The images may be combined using a commercial program such as
Antechinus Media Editor
(\hhref{http://www.c-point.com}),
Apple Quicktime Pro (\hhref{http://www.quicktime.com}),
 or
Adobe Premiere Pro (\hhref{http://www.adobe.com}) or the public domain program ffmpeg (\hhref{https://www.ffmpeg.org}).  Smokeview will add a dialog box for making a movie if ffmpeg is installed.
The steps to making a movie are:

\begin{enumerate}
\item Set up Smokeview by orienting the scene and loading the
desired data files.

\item Select the {\em Options/Render}\ menu and pick the
desired frame skip value. The more frames you include in
the animation, the smoother it will appear. Of course, more
frames result in larger file sizes.  Choose fewer frames
if the movie is to appear on a web site.

The dialog box illustrated
in Figure \ref{figRENDER}\ may also be used for generating an image
sequence.  Widgets exist for selecting the image type, the number of
frames to skip between images and for creating the image sequence.
One may also select a clipping region making the final image size smaller.

\item Use a program such as the Antechinus Media Editor,
Apple Quicktime Pro
 or
Adobe Premiere Pro,
to assemble the
JPEGS or PNGS rendered in the previous step into a movie
file. If ffmpeg is installed use the Movie dialog illustrated in Figure \ref{figRENDER}.

\end{enumerate}

\begin{figure}[bph]
\begin{center}
\begin{tabular}{cc}
\includegraphics[width=2.3194444in]{\SMVfigdir/figRENDER}&
\includegraphics[width=2.3194444in]{\SMVfigdir/figMOVIE}
\end{tabular}
\end{center}
\caption[Render and Movie dialog box.]{Render and Movie dialog box. These dialog boxes allow one to specify
options for rendering an image sequence of the displayed scene and to combine these images into a movie.  The Movie dialog box is available if the program ffmpeg is installed.}
\label{figRENDER}
\end{figure}

The default Smokeview image size is $640\times 480$ .  This size
is fine if the movie is to appear in a presentation located on a
local hard disk.  If the movie is to be placed on a web site then
care needs to be taken to insure that the generated movie file is
a reasonable size.  Two suggestions are to reduce the image size
to $320\times 240$ or smaller by modifying the {\tt WINDOWWIDTH}
and {\tt WINDOWHEIGHT}\ \svini\ keywords  and to reduce the number
of frames to 300 or less by skipping intermediate frames {\em via}
the {\em Options/Render}\ menu.

Sometimes when copying or {\em capturing}\ a Smokeview scene it is
desirable, or even necessary, to have a margin around the scene.
This is because the capturing system does not include the entire
scene but itself captures an indented portion of the scene. To
indent the scene, either press the ``h'' key or select the {\em
Option$>$Viewpoint$>$Offset Window}\ menu item. The default
indentation is 45 pixels. This may be changed by adding/editing
the WINDOW\_OFFSET keyword in the \svini\ file.

Note, the Smokeview animation must be running when the render command is
selected or only one frame will be saved instead of the entire image sequence.

Volume rendered smoke files which are really 3D slice files can be quite large.
Normally Smokeview loads an entire data set before visualizing it.
When creating image sequences for volume rendered smoke files,
Smokeview allows one to load data and create an image one frame or time step at a time.
Figure \ref{figVOLRENDER}\ shows the dialog for doing this.  This dialog
is a part of the 3D smoke dialog box. One can specify the starting frame index
and the number of frames to skip.  This allows one to run multiple Smokeview's in
parallel.

\begin{figure}[bph]
\centerline{
\includegraphics[width=2.194444in]{\SMVfigdir/figVOLRENDER}
}
\caption[Volume Render dialog box.]{Volume Render dialog box.
This dialog box allows one to specify
options for rendering an image sequence of a volume rendered smoke file. Data is loaded
and an image is constructed one frame at a time allowing movies to be made of cases
with very large data sets.
}
\label{figVOLRENDER}
\end{figure}

%---------------------------------------------------------------------------------
%----------------------------- Annotating the Scene ------------------------------
%---------------------------------------------------------------------------------

\chapter{Annotating the Scene}
\label{section:annotate}\ \label{subsect_features}
%---------------------------------------------------------------------------------
%---------------------------------------------------------------------------------
\section{Overview}
Smokeview scenes may be annotated by adding {\em ticks}\ with associated
text documenting their location
or by adding text strings at arbitrary locations and time durations.
Both of these annotations
are added with the {\em User Ticks}\ and the User Label dialog
box respectively.  These dialog boxes are both invoked by selecting the
{\em Dialogs$>$Display}\ menu item.

%---------------------------------------------------------------------------------
%---------------------------------------------------------------------------------
\section{User Ticks Settings Dialog Box}
The User Ticks Settings dialog box allows one to place
ticks and labels along one or more coordinate axes. The
user may specify tick spacing, number of sub-tick intervals and
how far axes extend.  There is an automatic placement option that
allows the tick axes to be placed based upon the orientation of
the scene.  The user may specify which tick axes are visible if
the automatic placement option is not invoked.  Figure
\ref{figTICKSdialog}\ illustrates the User Ticks Settings
dialog box.  It is a panel of the Display dialog box.
Figure \ref{figTICKSdialogexample}\ shows the ticks and labels
resulting from the dialog box.

\begin{figure}[bph]
\centerline{
\includegraphics[width=4.9236in]{\SMVfigdir/figTICKS}
}\ \caption[Ticks dialog box.]{Ticks dialog box. The
Ticks dialog box is invoked by selecting {\em
Dialogs$>$Display}. }\ \label{figTICKSdialog}
\end{figure}

\begin{figure}[bph]
\begin{center}
\includegraphics[width=5.0in]{SCRIPT_FIGURES/thouse5_ticks}
\end{center}
\caption{Annotation example using the Ticks dialog box}
\label{figTICKSdialogexample}%
\end{figure}

%---------------------------------------------------------------------------------
%---------------------------------------------------------------------------------
\section{{\em User Label}\ Dialog Box}
The User Label dialog box allows one to place
text strings at arbitrary locations within the scene.  The user may also
control the time interval when they are visible.  Figure  and at arbitrary time intervals.
Figure \ref{figLABELdialog}\ illustrates the User Label
dialog box.  It is a panel within Display dialog box.
It has controls for specifying the text string, $(x,y,z)$ location,
start and stop time and color.

\begin{figure}[bph]
\centerline{
\includegraphics[width=4.9236in]{\SMVfigdir/figLABEL}
}\ \caption[User Label dialog box.]{User Label dialog
box. The User Label dialog box is invoked by selecting {\em
Dialogs$>$Display}. }\ \label{figLABELdialog}
\end{figure}


%---------------------------------------------------------------------------------
%---------------------------------------------------------------------------------
\section{TICKS and LABEL keywords}
Tick marks and label annotation
can be also placed within the 3D scene using the TICKS and LABEL keywords.
FDS places tick marks and labels
documenting the scene dimensions.  To replace or customize
these annotations add the {\tt TICK}\ keyword to a .smv file
using the following format:

\begin{lstlisting}
TICKS
xb yb zb xe ye ze nticks
ticklength tickdir r g b tickwidth
\end{lstlisting}

\noindent where {\tt xb}, {\tt yb}, and {\tt zb}\ are the x, y and
z coordinates of the first tick; {\tt xe}, {\tt ye}\ and {\tt ze}
are the x, y and z coordinates of the last tick and {\tt nticks}
is the number of ticks. The coordinate dimensions are in physical
units, the same units used to set up the FDS geometry. The
parameter {\tt ticklength}\ specifies the length of the tick in
physical units. The parameter {\tt tickdir}\ specifies the tick
direction.  For example 1(-1) places ticks in the
positive(negative) x direction. Similarly, 2(-2) and 3(-3) place
ticks in the positive(negative) y and positive(negative) z
directions.

The color parameters {\tt r}, {\tt g}\ and {\tt b}\ are the
red, green and blue components of the tick color each
ranging from 0.0 to 1.0. The foreground color (white by
default) may be set by setting any or all of the {\tt r},
{\tt g}\ and {\tt b}\ components to a negative number. The
{\tt tickwidth}\ parameter specifies tick width in pixels.
Fractional widths may be specified.

The {\tt LABEL}\ keyword allows a text string to be added
within a Smokeview scene.  The label color and start and
stop appearance time may also be specified. The format is
given by

\begin{lstlisting}
LABEL
x y z r g b tstart tstop
label
\end{lstlisting}

\noindent where ({\tt x, y, z}) is the label location in Cartesian
coordinates and {\tt r, g, b}\ are the red, green and blue color
components ranging from 0.0 to 1.0.  Again, if a negative value is
specified then the foreground color will be used instead (white is
the default).  The parameters, {\tt tstart}\ and {\tt tstop}
indicate the time interval when the label is visible. The text
string is specified on the next line ({\tt label}).

Figure \ref{figticklabels}\ shows how the {\tt TICKS}\ and
{\tt LABEL}\ keywords can be used together to create a
{\em ruler}\ with major and minor tick marks illustrated in Fig.
\ref{figticklabelexample}.

\begin{figure}[bph]
{\small
\begin{lstlisting}
TICKS
0.0 0.0 0.0 8.0 0.0 0.0 5
0.5 -2.0 -1. -1.0 -1.0 4.0
TICKS
1.0 0.0 0.0 9.0 0.0 0.0 5
0.25 -2.0 -1. -1.0 -1.0 4.0
TICKS
0.0 0.0 0.0 0.0 0.0 2.0 3
0.5 -1.0 -1. -1.0 -1.0 4.0
TICKS
0.0 0.0 0.0 0.0 4.0 0.0 5
0.5 -1.0 -1. -1.0 -1.0 4.0
LABEL
0.0 -0.6 0.0 -1.0 0.0 0.0 0.0 20.0
0
LABEL
2.0 -0.6 0.0 -1.0 0.0 0.0 0.0 20.0
2
LABEL
4.0 -0.6 0.0 -1.0 0.0 0.0 0.0 20.0
4
LABEL
6.0 -0.6 0.0 -1.0 0.0 0.0 0.0 20.0
6
LABEL
8.0 -0.6 0.0 -1.0 0.0 0.0 0.0 20.0
8
LABEL
9.5 -0.6 0.0 -1.0 0.0 0.0 0.0 20.0
m
\end{lstlisting}
}
\caption{ TICKS and LABEL commands used to create image in Fig. \ref{figticklabelexample}}
\label{figticklabels}%
\end{figure}

\begin{figure}[bph]
\begin{center}
\includegraphics[height=3.0in]{\SMVfigdir/ticklabels}
\end{center}
\caption{ Annotation example using the TICKS and LABEL keyword. }
\label{figticklabelexample}%
\end{figure}

%---------------------------------------------------------------------------------
%----------------------------------- Utilities -----------------------------------
%---------------------------------------------------------------------------------

\chapter{Utilities}
Several utilities are included with the FDS/Smokeview distribution allowing one
to more easily analyze and generate data.  Smokezip may be used to compress
FDS data files resulting in quicker load times in Smokeview.  Smokediff may be used
to compare two FDS cases.  Smokediff generates another .smv file and a set of data files
which can be viewed with Smokeview.  Background may be used to take advantage of multiple
core computers by running more than one FDS case at a time.  This is most useful when running
a long list of FDS cases. Background runs a case whenever the CPU load is below a specified level.

%------------------------ smokezip ------------------------------------------------

\input{smokezip}

%------------------------ smokediff ------------------------------------------------

\input{smokediff}

%------------------------ background ------------------------------------------------

\input{background}

%------------------------ wind2fds ------------------------------------------------

\input{wind2fds}

%------------------------ dem2fds ------------------------------------------------

%\input{dem2fds}

%---------------------------------------------------------------------------------
%------------------------ Summary ------------------------------------------------
%---------------------------------------------------------------------------------

\chapter{Summary}
Often fire modeling is looked upon with skepticism because of the
perception that eye-catching images shroud the underlying physics.
However, if the visualization is done well, it can be used to
assess the quality of the simulation technique. The user of FDS
chooses a numerical grid on which to discretize the governing
equations. The more grid cells, the better but more time-consuming
the simulation. The payoff for investing in faster computers and
running bigger calculations is the proportional gain in calculation accuracy and realism
manifested by the images. There is no better way to demonstrate
the quality of the calculation than by showing the realistic
behavior of the fire.

Up to now, most visualization techniques have provided useful ways
of analyzing the output of a calculation, like contour and
streamline plots, without much concern for realism. A
rainbow-colored contour map slicing down through the middle of a
room is fine for researchers, but for those who are only
accustomed to looking at real smoke-filled rooms, it may not have
as much meaning. Good visualization needs to provide as much
information as the rainbow contour map but in a way that speaks to
modelers and non-modelers alike. A good example is smoke
visibility. Unlike temperature or species concentration, smoke
visibility is not a local quantity but rather depends on the
viewpoint of the eye and the depth of field. Advanced simulators
and games create the illusion of smoke or fog in ways that are not
unlike the techniques employed by fire models to handle thermal
radiation. The visualization of smoke and fire by Smokeview is an
example of the graphics hardware and software actually computing
results rather than just drawing pretty pictures. A common concern
in the design of smoke control systems is whether or not building
occupants will be able to see exit signs at various stages of a
fire. FDS can predict the amount of soot is located at any given
point, but that doesn't answer the question. The harder task is to
compute on the fly within the visualization program what the
occupant would see and not see. In this sense, Smokeview is not
merely a {\em post-processor}, but rather an integral part of the
analysis.

The purpose of Smokeview is to help one gain insight into the results
of fire modeling simulations.
Some areas of future work pertaining to the technical aspects of
Smokeview include improving the visual modeling of smoke and fire
and improving Smokeview's ability to handle larger cases.
General strategies for improving Smokeview's ability to visualize
cases and therefore to improve the understanding of computed fire
flow are discussed in more detail in the
Smokeview Technical Guide~\cite{Smokeview_Tech_Guide}.




\bibliography{../../../fds/Manuals/Bibliography/FDS_general,../../../fds/Manuals/Bibliography/FDS_refs,../../../fds/Manuals/Bibliography/FDS_mathcomp,../Bibliography/sv_fire,../Bibliography/sv_graphics}
\addcontentsline{toc}{chapter}{References}

\part{Appendices}
\appendix
\addcontentsline{toc}{chapter}{Appendices}

%---------------------------------------------------------------------------------
%------------------------ Command Line Options -----------------------------------
%---------------------------------------------------------------------------------

\chapter{Command Line Options}
\label{sectioncommand}\ Smokeview may be run from a command shell.
Various command line options are available altering Smokeview's
startup behavior such as creating a configuration file, using
stereo, using the demo mode or running a script. To obtain a list
of command line options, type:
\begin{lstlisting}
smokeview -help
\end{lstlisting}
\noindent without any arguments which results in output similar to:\\

\lstinputlisting{SCRIPT_FIGURES/smokeview.help}

%---------------------------------------------------------------------------------
%------------------------ Menu Options -------------------------------------------
%---------------------------------------------------------------------------------

\chapter{Menus}
\label{sectionmenu}
% --------------------MENU OPTIONS -----------------------

The user interacts with
Smokeview using menus, dialog boxes and the keyboard.
This appendix gives a brief overview of menus and dialogs.
Appendix \ref{sectionkeyboard}\ documents keyboard shortcuts.

Menus are accessed by clicking the mouse anywhere in the scene with the right mouse button.
The top level menu contains the {\em Load/Unload}\ menu for loading
and unloading data files , the {\em Show/Hide}\ menu for showing and
hiding data files previously loaded and other scene elements, the {\em Options}\ menu
for setting options and performing actions such as rendering images of the scene or setting data units, the {\em Dialogs}\ menu for opening dialog boxes,
the {\em Help}\ menu and {\em Quit}\ menu items.

%---------------------------------------------------------------------------------
%---------------------------------------------------------------------------------
\section{Load/Unload}
\begin{figure}[bph]
\begin{center}
\includegraphics[width=4.875in]{\SMVfigdir/menu_load}
\caption{Load/Unload Menu.}
\label{fig_loadmenu}
\end{center}
\end{figure}

\label{sectload}The Load/Unload menu, illustrated in Fig.
\ref{fig_loadmenu}, is used to load or unload data files generated
by FDS, CFAST or any fire model outputting data files using file formats
documented in this or the FDS user's guide\cite{FDS_Users_Guide}.

A menu item for each file
generated by the fire model is present under {\em Load/Unload}.
Selecting one of these menu entries causes the corresponding data file
to load and be displayed. The data may be unloaded by
selecting an {\em Unload}\ menu item appearing under the file
list. Selecting {\em Unload All}\  unloads all
files. To hide a data file, select the {\em Show/Hide}\ menu
option corresponding to the  file type to be hidden.

The character ``{\tt *}'' occurring before a file name in the menu entry indicates
that the file is loaded. If a file is loaded
but not visible, use the appropriate {\em Show/Hide}\ option
to make it visible.

The following is a list of file types visualized by Smokeview.

\blist

\hitem{3D Smoke File ({\tt .s3d})}This menu item allows one to
load soot opacity and hrrpuv (heat release per unit volume) files.
Smokeview uses the information contained in these files to
visualize smoke realistically .

\hitem{Multi-Slice File ({\tt .sf})}This menu item
allows one to load all slices occurring in one plane (within a grid cell) simultaneously.
It also gives the option to unload the currently loaded multi-slices.

\hitem{Multi-Vector Slice File ({\tt .sf})}This menu item
allows one to load all vector slices occurring in one plane (within a grid cell) simultaneously.
It also gives the option to unload the currently loaded multi-slices.

\hitem{Slice File ({\tt .sf})}\ This menu item gives the
name and location of all available slice
files and also the option to
unload the currently loaded slice files.  \\

\hitem{Vector Slice File ({\tt .sf})}This menu item gives
the name of all slice files that have one or more
associated U, V and/or W velocity slice files. These slice
files must be defined for the same region (or slice) in the
simulation.

\hitem{Isosurface File ({\tt .iso})}This menu item gives the name
of all isosurface files and also the option to unload the
currently loaded isosurface file.

\hitem{Boundary File ({\tt .bf})}\ This menu item gives the
name of all boundary files and also the option to
unload the currently loaded boundary file.\\

\hitem{Particle File ({\tt .part})}\ This menu item gives the name
of all particle file and also the option to
unload the currently loaded particle file.\\

\hitem{Plot3D File ({\tt .q})}\ This menu item gives the name of
all Plot3D files and also the option to
unload the currently loaded Plot3D file.\\

\hitem{Configuration Files ({\tt .ini)}}The INI or preference file
contains configuration parameters that may be used to customize
Smokeview's appearance and behavior. This menu item allows one to
create (or overwrite) a preference file named either \svini\ or
{\tt casename.ini}. A preference file contains parameter settings
for defining how Smokeview visualizes data. This file may be
edited and re-read while Smokeview is running.

\hitem{Compression}3D smoke and boundary files may be compressed using this menu item.

\hitem{Script options}Smokeview scripts may be recorded or run using this menu.

\hitem{Show File Names}Load and Unload menus by default are
specified using the location and type of visual to be displayed.
This menu item adds file names to the Load and Unload menus.

\hitem{Reload}This menu item allows one to reload files at
immediately or at intervals of 1, 5 or 10 minutes.
The {\tt u}\ key may used to reload files from the keyboard.
This is useful when using
Smokeview to display a case that is currently running in
FDS.
\\
\hitem{Unload All}This option causes all data files to be unloaded.
\elist

%---------------------------------------------------------------------------------
%---------------------------------------------------------------------------------
\section{Show/Hide}
\label{sectshow}\ The {\em Show/Hide}\ menu allows one to show or hide various
parts of the simulation. These menu items only appear if they
pertain to the simulation.  For example the {\em Particles}
menu only appears if a particle file has been loaded.
The ``{\tt *}'' character is
used to indicate that the visualization feature corresponding
to that menu item is set or active.

%---------------------------------------------------------------------------------
%---------------------------------------------------------------------------------
\section{Options}
The option menu allows one to specify display units, rotation method, maximum frame rate, to render images, create tours and set font size.

\blist
\hitem{Units}Select alternate units for quantities such as temperature, velocity, distance {\em etc.}

\hitem{Rotation}Several rotation methods may be used to rotate the scene. The ``{\tt e}'' keyboard shortcut may be used to toggle between rotation methods.
\begin{itemize}
\item The {\em Eye Centered}\ method allows one to rotate the scene relative to the observer's point of view or {\em eye}.
Eye centered views make it easier to move around within the scene as in modern
computer games.

\item The {\em World Centered}\ rotation method allows one to rotate the scene relative to the scene's center.

\item The {\em World Centered/level}\ rotation method is the same as {\em World Centered}\ but with level rotations.
\end{itemize}

\hitem{Max Frame Rate}
This max frame rate option controls the rate at which image frames are displayed.
The sub-menus allow one to specify a maximum frame rate.  The
actual frame rate may be slower if the scene is complex and the
graphics card is unable to draw the scene sufficiently fast. The
{\em unlimited}\ menu item allows one to display frames as rapidly
as the graphics hardware permits.  The {\em Real Time}\ menu item
allows one to draw frames so that the simulation time matches real
time. The {\em step}\ menu item allows one to step through the
simulation one time step at a time. This menu item may be used in
concert with the {\em Render}\ menu item described below to create
images at the desired time and view orientation for inclusion into
reports. This is how figures were generated in this report.

\hitem{Render}
The {\em Render}\ menu allows one to create PNG or JPEG image files
of the currently displayed scene.

The {\em Render}\ menus allow one to specify an
integer indicating the number of frames
between rendered images. This allows one to generate images
encompassing the entire time duration of the simulation
which in turn can be converted into movie files ({\tt mpeg,
mov, avi}, etc) using software available on the internet.
Rendering may be stopped by selecting {\em Cancel}.

\hitem{Tours}The keyboard shortcut for the render option is {\tt r}.
The {\em Tour}\ menu allows one to show and hide available tours.


\hitem{Font Size}This option allows one to display text in either a normal or a large font.
\elist

%---------------------------------------------------------------------------------
%---------------------------------------------------------------------------------
\section{Dialogs}
\begin{figure}[bph]
\begin{center}
\includegraphics[width=6.1111in]{\SMVfigdir/menu_dialogs}
\caption{Dialogs Menu.}\ \label{fig_dialogmenu}
\end{center}
\end{figure}
The {\em Dialogs}\ menu, illustrated in Fig.
\ref{fig_dialogmenu}, allows one to select dialog boxes used for
setting various Smokeview features and configuration parameters.
Commonly used dialog box entries appear first.
These are  {\em Data bounds}\ for setting data file bounds (and other data
file characteristics),
{\em Display}\ for setting various parameters that control
how the scene appears, {\em Motion}\ for controlling scene
movement through rotation and translation and {\em Viewpoints}\
for defining and setting viewpoints.
Less commonly used dialog box entries appear at the bottom of the menu
under the sub-menus {\em Data}, {\em Files}, {\em View}\ and
{\em Window}.  Dialog menu entries and a short description for each are listed below.

\blist

\hitem{Data bounds}Dialog box for setting min/max data bounds, min/max clipping data bounds and other parameter settings related to data files.

\hitem{Display}Dialog box for setting various display parameters.
\hitem{Motion}Dialog box for rotating and translating the scene.
\hitem{Viewpoints}Dialog box for saving and setting viewpoints.
\elist

\noindent Dialog menu entries along with the sub-menu of {\em Dialogs}\ where they appear (enclosed in parenthesis) are listed below.
\blist

\hhitem{Auto load}{Files}Dialog for automatically loading data files.

\hhitem{Clip scene}{View}Dialog for clipping data and/or geometry.

\hhitem{Compress}{Files}
Dialog for compressing FDS generated data files using the program Smokezip.

\hhitem{Coloring}{Data}Dialog for setting data coloring characteristics.  One may select the colorbar used to color data, how the colorbar is displayed (continuous, stepped or discrete lines, color opacity level {\em etc}.

\hhitem{Configuration}{Files}\ Dialog box for saving or loading configuration files.

\hhitem{Device/Objects}{Data}\ Dialog box for scaling Smokeview objects and
showing data values associated with FDS devices.

\hhitem{Edit colorbar}{View}\ Dialog box for creating new and editing
existing colorbars.

\hhitem{Examine geometry}{View}Dialog box for examining FDS blockages.

\hhitem{Fonts}{Window}\ Dialog box for selecting the font size used to display text.  The choices are limited to small, large or scaled.

\hhitem{Labels}{Window}\ Dialog box for defining text labels and controlling
when and where they are placed.


\hhitem{Particle tracking}{Data}\ Dialog box for releasing and viewing particles
within the scene.

\hhitem{Render images}{Files}\ Dialog box for creating images of a Smokeview scene.  In addition, if the program {\tt ffmpeg}\ is present in the user path, one may create movies.

\hhitem{Scaling}{Window}\ Dialog box for scaling the X, Y and or Z dimensions of a Smokeview scene.

\hhitem{Scripts}{Files}Dialog box for recording or running a Smokeview script.

\hhitem{Show/Hide}{Data}Dialog box for showing or hiding data.

\hhitem{Slice motion}{Data}Dialog box for controlling the position and/or orientation of a 3D slice file.

\hhitem{Stereo parameters}{View}Dialog box for specifying the method used to display a Smokeview scene in stereo.

\hhitem{Time bounds}{Data}Dialog box for specifying the min/max time bounds for loading data.

\hhitem{Tours}{View}Dialog box for creating new tours and editing existing ones.

\hhitem{User ticks}{Display}Dialog box for specifying the placement of tick marks to appear along the edges of a Smokeview scene.


\hhitem{Window properties}{Window}Dialog box for specifying the characteristics (screen size, projection method) of the window containing the Smokeview scene.





\elist


%---------------------------------------------------------------------------------
%------------------- Keyboard Shortcuts -------------------------------------
%---------------------------------------------------------------------------------

\chapter{Keyboard Shortcuts}
\label{sectionkeyboard}\ Many menu commands have equivalent
keyboard shortcuts.  These shortcuts are described here and are
also briefly described under the {\em Help}\ menu item from within
Smokeview.

\blist


\kitem{a}Increase slice and PLOT3D vector length (use ALT a to decrease vector length).

\kitem{a}When in {\em eye centered}\ movement mode, slide to the left.

\kitem{ALT a}Shorten slice and PLOT3D vector lengths (use 'a' to lengthen vector lengths).

\kitem{A}Switch 2D plot display between device, HRRPUV,  both device and HRRPUV and none.

\kitem{b}Toggle boundary file visibility

\kitem{B}Show geometry and blockage bounding box outlines when moving scene.

\kitem{ALT b}Open the Bounds dialog box.

\kitem{c}Toggle 2D contour display between banded, continuously
shaded and line contours.

\kitem{c}Advance highlighted zone fire modeling compartment.

\kitem{C}Toggle 3D smoke culling.  If the scene is a zone fire modeling simulation,
toggle zone fire modeling compartment highlighting.

\kitem{ALT c}Open the Clipping dialog box.

\kitem{d}Activates {\tt CTRL}\ key when moving the scene with the mouse.
Pressing and releasing the {\tt d}\ key then moving the mouse causes the scene to go up and down until the
mouse button is released.

\kitem{D}Slide right when in {\em eye centered}\ movement mode.

\kitem{ALT d, D}Open the Display dialog box.

\kitem{e,E}Toggle how the scene is manipulated.  In {\em eye
view}\ scene motion is relative to the observer.  In {\em world
view}\ scene motion is relative to the scene center.

\kitem{ALT e}Open the Blockage info dialog box.

\kitem{f}Activates the {\tt ALT}\ key when moving the scene with the mouse.
Pressing and releasing the {\tt f}\ key then moving the mouse causes the scene to go in and out until the
mouse button is released.

\kitem{F}Toggle algorithm for hiding blockage overlaps.

\kitem{g}Toggle the grid visibility.  When the grid display option
is active, the x, y and z keys may be used to show or hide the
grid perpendicular to the x, y and z axes respectively.

\kitem{G}Toggle the use of the GPU (if present).

\kitem{ALT g}Open the Viewpoint dialog box.

\kitem{h}Display the colorbar as a histogram..

\kitem{H}Toggle slice and vector slice file  visibility.

\kitem{i}Toggle Plot3D iso-contour visibility.

\kitem{I}Toggle slice file visibility within obstacles.

\kitem{ALT i}Toggle device visibility.

\kitem{j,J}Increase size of Smokeview objects (use ALT j to decrease size).

\kitem{ALT j}Decrease size of Smokeview objects (use 'j' to increase the size).

\kitem{k}Toggle timebar visibility.

\kitem{K}Toggle fixing the window aspect ratio when changing its size.

\kitem{ALT k}Toggle device selection.

\kitem{m}Switch between meshes in multiple mesh cases.

\kitem{M}Toggle command line scene clipping.  When turned on, the
cursor and page up/down keys can be used to move the clipping planes.
The clipping plane usage can be toggled using the x/y/z or X/Y/Z keys.
The lower case keys toggle the lower clipping plane.  The upper case
keys toggle the upper clipping planes.
The W key is used to toggle clipping modes.

\kitem{ALT m}Open the Motion/View/Render dialog box.

\kitem{n}Toggle cface normal vector visibility.

\kitem{N}Force bound updates when loading files (assume fds is running).

\kitem{o}Switch between outline viewing modes. The modes are
\begin{enumerate}
\item outline of the current mesh
\item outline of the entire case
\item no outline
\end{enumerate}
This option may be used with the {\em m}\ key to highlight all meshes in sequence of a case.

\kitem{O}Toggle the blockage view state between 1) defined in input file and 2) outline only

\kitem{ALT o}Switch between various blockage view states.  States are:
1) defined in input file,
2) defined in input file + outline,
3) solid,
4) outline only and
5) hidden

\kitem{p, P}Cycle through particle file types or Plot3D file types (if particles are not loaded).
P will cycle through Plot3D file types if particles are loaded.

\kitem{q}Switch between blockage views.  These views are blocks
that are aligned on grid lines, blocks as specified by the user
(in the FDS input file) and blocks as generated by a CAD (computer
aided drawing) package.

Also switch between vent views.  When a circular vent is specified,
the {\em q}\ key will switch between how the vent is specified by the user, a circle
and how it is represented by FDS, a series of grid cell faces.

\kitem{Q}Toggle texture visibility.

\kitem{r,R}Render the current scene as a JPEG or a PNG file which
can be viewed in a web browser or inserted into a word processing
document.  If {\tt R}\ is selected then the
scene is rendered in with double the screen resolution (or greater if the image multiplier menu is selected).

\kitem{ALT r}Toggle research mode.  Research mode uses global min
and max bounds for coloring data and turns off smoothing when displaying colorbar labels.

\kitem{ALT R}Display scene in 360$^\circ$ view mode - all view directions of the scene are displayed in a $1024\times 512$ image.

\kitem{s}Increment the number of vectors skipped. This is useful
for making vector displays more readable when grids are finely
meshed.

\kitem{s}Move backwards when in {\em eye centered}\ movement mode.

\kitem{S}Change stereo modes (left/right, red/blue, none, etc.)

\kitem{ALT s}Open the 3D Smoke dialog box.

\kitem{t}Toggle the time stepping mode.  Time stepping mode allows
one to step through the simulation one time step at a time.

\kitem{T}Toggle the time bar time label between showing time as seconds
and time as hour, minutes and seconds.

\kitem{ALT t}Open the Edit Tours dialog box.

\kitem{u}Reload currently loaded files.  This is useful when
using Smokeview to display a case that FDS is currently running.

\kitem{ALT u}\ Toggle the option to draw a coarse portion of a 2D
slice file within an embedded mesh.

\kitem{U}Toggle between original and fast blockage drawing.

\kitem{v}Toggle Plot3D vector visibility.  This option is only active
when there are U, V and/or W velocity components present in the
Plot3D file.

\kitem{V}Toggle volume rendered smoke visibility.

\kitem{ALT v}Toggle the projection method used to visualize a
scene. The two projection methods are size preserving and
perspective.

\kitem{w}Toggle visibility of 3D node centered general slice.
When in {\em eye centered}\ movement
mode, move forward.

\kitem{W}Toggle between four clipping modes: 1) disabled, 2) blockages and data, 3) blockages and 4) data. The M key is used to toggle command line clipping on and off.

\kitem{ALT w}Open the WUI dialog box.

\kitem{x,X y,Y, z,Z}Toggle the visibility of the Plot3D data
planes perpendicular to the x, y and z axes respectively (parallel
to the yz, xz and xy planes).

\kitem{ALT x}Close all dialog boxes.

\kitem{ALT z}Open the Compress Files portion of the
File/Bounds dialog box.

\kitem{0}Reset a time dependent animation to the initial time.

\kitem{1-9}Number of frames to skip when viewing an animation.
 \elist

\blist

\kitem{\~}Level scene.

\kitem{!}Snap scene to nearest 45 degree rotation angle.

\kitem{@}Toggle display of FDS values (as floating point numbers)
when viewing vector centered slice files.

\kitem{\#}Save configuration settings to the casename.ini file.

\kitem{\$}Force 3D smoke/fire to be opaque.

\kitem{ALT \$}Toggle trainer or demonstrator mode.  When active,
displays a dialog box that provides a simple set of controls for
controlling the scene.

\kitem{\%}Toggle single stepping mode.  If activated, Smokeview
will execute a script one command at a time.

\kitem{\^ \ }\ When single stepping mode is activated, this key
causes the next script command to be executed.

\kitem{\&}Toggle line anti-aliasing (draw lines smoothly)

\kitem{ALT \&}Toggle HVAC metro view mode.

\kitem{*}Hide all 3D slice planes (aligned with 3 coordinate axes
and general slice planes).

\kitem{=}Toggle vertex selected in examine geometry dialog

\kitem{.}Toggle locking of left/right scene motion when using the SHIFT mouse to change the scene aperture.

\kitem{(}Toggle render clipping mode

\kitem{[}Turn on tour editing

\kitem{]}Turn off tour editing

\kitem{<, >}Increase, decrease vector point size

\kitem{ALT <, ALT >}Use previous/next colorbar.

\kitem{;}Flip colorbar

\kitem{/}Toggle parallel loading of particle files.  The number of threads used to load
particles files in parallel may be specified using the Settings portion of the particle bounds
dialog box.

\hitem{Left/Right Cursor}When the {\em eyeview}\ mode is {\em eye
centered}\ then these keys rotate the scene to the left or right
otherwise they increment/decrement the Plot3D plane location
displayed in the xz plane.

\hitem{Up/Down Cursor}Increment/decrement the Plot3D plane
location displayed in the yz plane.

\hitem{Page Up, Page Down}Increment/decrement the Plot3D plane
location displayed in the xy plane.


\kitem{-}Decrement Plot3D data planes, Plot3D iso-contour levels
or time step displayed.

\hitem{space bar}Increment Plot3D data planes, Plot3D iso-contour
levels or time step displayed.
 \elist

%---------------------------------------------------------------------------------
%------------------------ File Formats -------------------------------------------
%---------------------------------------------------------------------------------

\chapter{File Formats and Extensions}
%---------------------------------------------------------------------------------
%---------------------------------------------------------------------------------
\section{FDS and Smokeview File Extensions}

%%-----------------------------------------------------------------------
\subsection{FDS file extensions}
\blist
\hitem{.bf}\ File containing boundary file data.

\hitem{.end}\ File containing Endian information.

\hitem{.fds}\ File containing the FDS input file.

\hitem{.iso}\ File containing iso-surface data

\hitem{.out}\ File containing FDS output.

\hitem{.prt}\ File containing particle file data using FDS 4 and earlier.

\hitem{.prt5}\ File containing particle file data using FDS 5 and later.

\hitem{.q}\ File containing Plot3D data.

\hitem{.sf}\ File containing slice file data.

\hitem{.s3d}\ File containing 3D smoke, HRRPUV data.
\elist

%%-----------------------------------------------------------------------
\subsection{Smokeview file extensions}

\blist

\hitem{.bini}\ File containing percentile and global data bounds for
boundary files in referenced casename.smv.

\hitem{.ini}\ File containing Smokeview configuration settings.

\hitem{.smv}\ File containing Smokeview keyword data.

\hitem{.ssf}\ File containing a Smokeview script.

\hitem{.svz}\ File containing compressed boundary, slice or 3D smoke/fire data.
The .svz extension is appended to the .bf, .sf or .s3d extension respectively.

\hitem{.sz}\ File containing sizing information for uncompressed data files.
The .sz files contain information about each data frame used by Smokeview to allocate memory.

\hitem{.szz}\ File containing sizing information for compressed .svz files
(files compressed with Smokezip with a .svz extension).

\elist

%---------------------------------------------------------------------------------
%---------------------------------------------------------------------------------
\section{Smokeview Bound File Format (.bini files)}
The first time a user views a boundary file, Smokeview computes data bounds by inputting
all boundary file data of the same type.
Smokeview records the bound computations result in a casename.bini file
so that it does not need to be performed a second time.  The .bini file is
then used in subsequent Smokeview sessions for displaying boundary file data.
The .bini file contains one or more B\_BOUNDARY keywords .

{\bf B\_BOUNDARY}\ defines the global minimum and maximum and percentile minimum and
maximum boundary
data bounds used to convert boundary data values to color indices.
The {\tt B\_BOUNDARY}\ keyword also has a parameter allowing
one to specify the data type.  The format is given by
\begin{lstlisting}
B_BOUNDARY
 global_min percentile_min percentile_max global_max data_type
\end{lstlisting}


%---------------------------------------------------------------------------------
%---------------------------------------------------------------------------------
\section{Smokeview Configuration File Format (.ini files)}
\label{sectionconfig}
\label{appendixini}
% --------------------CONFIGURATION FILES -----------------------

Smokeview uses configuration files (.ini files) to set input parameters not
settable using menus or the keyboard and to save the visualization state.
Smokeview looks for configuration files in the installation directory
(directory where the smokeview program is located), a directory named .smokeview
in your home directory and the directory containing the case you are running.
The configuration file is named smokeview.ini in the installation
and  home/.smokeview directories and casename.ini in the case directory where casename is the name of the case.

The \svini\ file found in the home/.smokeview directory may be created by typing:
{\tt smokeview -ini}\
from the command line or by selecting the {{\tt Save settings(all cases)}\ menu item.
Likewise the casename.ini file may be created by selecting the {\tt Save settings(this case)}\ menu item.

Smokeview reads the installation .ini file first followed by the home/.smokeview .ini file followed by the .ini file in your case directory.
The global \svini\ file is used to
customize parameters for all Smokeview runs. The {\tt casename.ini}\ file is
used to customize parameters for only those Smokeview runs with
the prefix casename.

All configuration file parameters unless otherwise noted consist of a {\tt KEYWORD}
followed by a value, as in:
\begin{lstlisting}
KEYWORD
value
\end{lstlisting}
Descriptions of many of the .ini keywords follow.

%%\subsection{Miscellaneous Settings}
\hiteminiN{CACHE_DATA}
\hiteminiN{GVERSION}
\hiteminiN{INPUT_FILE}
\hiteminiN{LABEL}
\hiteminiN{LABELSTARTUPVIEW}
\hiteminiN{MESHOFFSET}
\hiteminiN{NORTHANGLE}
\hiteminiN{OFFSETSLICE}
\hiteminiN{OUTLINEMODE}
\hiteminiN{RENDERCLIP}
\hiteminiN{RENDERFILELABEL}
\hiteminiN{SCALEDFONT}
\hiteminiN{SENSORNORMCOLOR}
\hiteminiN{SKYBOX}
\hiteminiN{SPEED}
\hiteminiN{SPHERESEGS}
\hiteminiN{STARTUPLANG}
\hiteminiN{STEREO}
\hiteminiN{SURFCOLORS}
\hiteminiN{TICKLABEL}
\hiteminiN{TICKLINEWIDTH}
\hiteminiN{TICKS}
\hiteminiN{TOUR_AVATAR}
\hiteminiN{TOURCIRCLE}
\hiteminiN{TOURINDEX}
\hiteminiN{TRAINERMODE}
\hiteminiN{TRAINERVIEW}
\hiteminiN{TREECOLORS}
\hiteminiN{TREEPARMS}
\hiteminiN{USENEWDRAWFACE}
\hiteminiN{ZAXISANGLES}
\hiteminiN{ZIPSTEP}
\hiteminiN{ONEVIEW}


%%\subsection{Device Settings}
\hiteminiN{DEVICEBOUNDS}
\hiteminiN{DEVICENORMLENGTH}
\hiteminiN{DEVICEORIENTATION}
\hiteminiN{DEVICEVECTORDIMENSIONS}
\hiteminiN{WINDROSEDEVICE}
\hiteminiN{WINDROSEMERGE}
\hiteminiN{WINDROSESHOWHIDE}
\hiteminiN{BEAM}

%%subsection{Visibility Settings}
\hiteminiN{FRAMERATE}
\hiteminiN{APERATURE}
\hiteminiN{GRIDPARMS}
\hiteminiN{MESHVIS}
\hiteminiN{USER_ROTATE}
\hiteminiN{USERTICKS}
\hiteminiN{VIEWPOINT6}
\hiteminiN{MOVIEFILETYPE}
\hiteminiN{MOVIEPARMS}

%%-----------------------------------------------------------------------
\subsection{Color and lighting}
All colors are specified using a 3-tuple: r g b
where r, g and b are the red, green and blue components of the color respectively.
Each color component is a floating point number ranging
from 0.0 to 1.0 where 0.0 is the darkest shade and 1.0 is the
lightest shade.  For example the 3-tuple 1.0 0.0 0.0 is bright red, 0.0 0.0 0.0 is black and
1.0 1.0 1.0 is white.


\hiteminiN{BLOCKSHININESS}
\hiteminiN{BLOCKSPECULAR}
\hiteminiN{BACKGROUNDCOLOR}
\hiteminiN{BLENDMODE}
\hiteminiN{CO2COLOR}
\hiteminiN{CO2COLORMAP}
\hiteminiN{COLORBARFLIP}
\hiteminiN{COLORBAR_SPLIT}
\hiteminiN{COLORBARTYPE}
\hiteminiN{COLORTABLE}
\hiteminiN{DIRECTIONCOLOR}
\hiteminiN{ENABLETEXTURELIGHTING}
\hiteminiN{EXTREMECOLORS}
\hiteminiN{LIGHTANGLES0}
\hiteminiN{LIGHTANGLES1}
\hiteminiN{LIGHTFACES}
\hiteminiN{LIGHTING}
\hiteminiN{LIGHTPROP}
\hiteminiN{VECTORCOLOR}

\blist

\hitem{BACKGROUNDCOLOR}Sets the color used to visualize
the scene background.
(default: {\tt 0.0 0.0 0.0})

\hitemini{BLOCKCOLOR}Sets the color used to visualize
internal blockages.
(default: {\tt 1.0 0.8 4.0})

\hitemini{BOUNDCOLOR}Sets the color used to visualize
floors, walls and ceilings.
(default: {\tt 0.5 0.5 0.2})

\hitemini{COLORBAR}Entries for the color palette in RGB (red, green, blue) format where
each color component ranges from 0.0 to 1.0 .
The default values (rounded to 2 digits) are specified with:
\begin{lstlisting}
COLORBAR
12
0.00 0.00 1.00
0.00 0.28 0.96
0.00 0.54 0.84
0.00 0.76 0.65
0.00 0.91 0.41
0.00 0.99 0.14
0.14 0.99 0.00
0.41 0.91 0.00
0.65 0.76 0.00
0.84 0.54 0.00
0.96 0.28 0.00
1.00 0.00 0.00
\end{lstlisting}

\hitemini{COLOR2BAR}\ Miscellaneous colors used by Smokeview.  The
default values are specified using:
\begin{lstlisting}
COLOR2BAR
8
1.0 1.0 1.0 :white
1.0 1.0 0.0 :yellow
0.0 0.0 1.0 :blue
1.0 0.0 0.0 :red
0.0 1.0 0.0 :green
1.0 0.0 1.0 :magenta
0.0 1.0 1.0 :cyan
0.0 0.0 0.0 :black
\end{lstlisting}

\noindent where the 8 indicates the number of colors defined and
the character string after the {\tt ``:''}\ are ignored.

\hitemini{COLORBAR\_FLIP}\ Specifies whether the colorbar is flipped (1) or not flipped (0)
(default: {\tt 0}).

\hitemNULL{DIRECTIONCOLOR}
\hitemini{FLIP}\ Specifies whether to flip (1) or not to flip (0) the foreground and background
colors.  By default the background color is black and the foreground color
is white.  Setting FLIP to 1 has the effect of having a white background and black foreground.
(default: {\tt 0}).

\hitemini{FOREGROUNDCOLOR}Sets the color used to visualize
the scene foreground (such as text labels).
(default: {\tt 1.0 1.0 1.0})

\hitemini{HEATOFFCOLOR}Sets the color used to visualize
heat detectors before they activate.
(default: {\tt 1.0 0.0 0.0})

\hitemini{HEATONCOLOR}Sets the color used to visualize
heat detectors after they activate.
(default: {\tt 0.0 1.0 0.0})


\hitemini{ISOCOLORS}\ Colors and parameters used to display animated isocontours.
Default:
\begin{lstlisting}
ISOCOLORS
 10.000000 0.800000 : shininess, transparency
 0.700000 0.700000 0.700000 : specular
 3 : number of levels
 0.960000 0.000000 0.960000 0.800000 : red, green, blue, alpha (opaqueness)
 0.750000 0.800000 0.800000 0.800000
 0.000000 0.960000 0.280000 0.800000
\end{lstlisting}

\hitemini{SENSORCOLOR}Sets the color used to visualize sensors.
(default: {\tt 1.0 1.0 0.0})

\hitemini{SETBW}The parameter used to set whether color shades (0) or shades of gray (1)
are to used for coloring contours and blockages.
(default: {\tt 0})

\hitemini{SPRINKOFFCOLOR}Sets the color used to visualize
sprinklers before they activate.
(default: {\tt 1.0 0.0 0.0})

\hitemini{SPRINKONCOLOR}Sets the color used to visualize
sprinklers after they activate.
(default: {\tt 0.0 1.0 0.0})

\hitemini{STATICPARTCOLOR}Sets the color used to visualize static particles (particles
displayed in frame 0).
 (default: {\tt 0.0 1.0 0.0}).

\hitemini{TIMEBARCOLOR}\ Sets the color used to visualize the timebar. (default: {\tt 0.6 0.6 0.6})

\hitemini{VENTCOLOR}Sets the color used to visualize vents.
(default: {\tt 1.0 0.0 1.0})
\elist

%%-----------------------------------------------------------------------
\subsection{Size}
The parameters described in this section allow one to customize
the size of various Smokeview scene elements. \blist
\hiteminiN{GRIDLINEWIDTH}
\hiteminiN{VECTORLINEWIDTH}

\hitemNULL{GRIDLINEWIDTH}
\hitemini{ISOLINEWIDTH}Defines the width in pixels
of lines used to draw animated iso-surfaces in outline mode.
(default: $2.0$)

\hitemini{ISOPOINTSIZE}Defines the size in pixels
of iso-surface particles.
(default: $4.0$)

\hitemini{LINEWIDTH}Defines the width of
lines\footnote{Except lines used to draw vents}\ in pixels.
(default: $2.0$)

\hitemini{PARTPOINTSIZE}Defines the size in pixels
of smoke or tracer particles.
(default: $1.0$)

\hitemini{PLOT3DLINEWIDTH}Defines the width in pixels
of lines used to draw Plot3D iso-surfaces in outline mode.
(default: $2.0$)

\hitemini{PLOT3DPOINTSIZE}Defines the size in pixels
of Plot3D iso-surface particles.
(default: $4.0$)

\hitemini{SENSORABSSIZE}Defines the sensor size drawn
by Smokeview using the same units as used to specify the grid coordinates.
(default: $0.038$)

\hitemNULL{SENSORRELSIZE}

\hitemini{SLICEOFFSET}Defines an offset
distance\footnote{distance is relative to the maximum grid cell width}
animated slices are drawn from adjacent solid surfaces.
(default: $0.10$)

\hitemini{SMOOTHLINES}Specifies whether lines should be drawn (1) or not drawn (0) using
anti-aliasing  (default: {\tt 1}).

\hitemNULL{SPHERESEGS}

\hitemini{SPRINKLERABSSIZE}Defines the sprinkler size drawn
by Smokeview using the same units as used to specify the grid coordinates.
(default: $0.076$)

\hitemini{STREAKLINEWIDTH}Defines the width of a streak line.
(default: $1.0$)

\hitemNULL{VECCONTOURS}

\hitemini{VECLENGTH}Defines the length of Plot3D vectors. A vector
length of $1.0$ fills one grid cell.  Vector lengths may also be
changed from within Smokeview by depressing the ``{\tt a}'' key.
(default: $4.0$)

\hitemNULL{VECTORLINEWIDTH}

\hitemini{VECTORPOINTSIZE}Defines the size in pixels of the point
that is drawn at the end of a Plot3D vector. (default: $3.0$)

\hitemini{VENTLINEWIDTH}Defines the width of
lines used to draw vents in pixels.
(default: $2.0$)

\hitemini{VENTOFFSET}Defines a distance used to offset
vents drawn from adjacent surfaces.
(default: $0.10$ (units of fraction of a grid cell width))

\hitemini{WINDOWHEIGHT}Defines the initial window height in pixels.
(default: $480$)

\hitemini{WINDOWWIDTH}Defines the initial window width in pixels.
(default: $640$)

\hitemini{WINDOWOFFSET}Defines a margin offset around the
Smokeview scene for use when capturing images to video.
(default: 45)
\elist

%%-----------------------------------------------------------------------
\subsection{Geometry Settings}
\blist
\hitemini{GEOMAXIS}This keyword specifies the length and line width of an x, y, z coordinate axes when it is drawn.
\begin{lstlisting}
GEOMAXIS
  length (float) width (float)
\end{lstlisting}

\hiteminiN{GEOMBOUNDARYPROPS}This keyword specifies how the geometry is drawn. Usage:

\begin{lstlisting}
GEOMBOUNDARYPROPS
  show_boundary_shaded, show_boundary_outline, show_boundary_points, geomboundary_linewidth, geomboundary_pointsize, boundary_edgetype
\end{lstlisting}
where geometry is drawn as a solid shaded surface if {\tt show\_boundary\_shaded}\ is 1 .  Outlines and points are drawn
if {\tt show\_boundary\_outline}\ or {\tt show\_boundary\_points}\ are 1. Geometry outline line width and point size are specified as positive
floating numbers using {\tt geomboundary\_linewidth}\ and {\tt geomboundary\_pointsize}.
\
\hiteminiN{GEOMCELLPROPS}

\hiteminiN{GEOMDOMAIN}Specifies whether geometry is drawn inside the domain and outside the domain. Usage:
\begin{lstlisting}
GEOMDOMAIN
  inside_domain (0/1)  outside_domain (0/1)
\end{lstlisting}

\hiteminiN{SHOWGEOMTERRAIN}
\hiteminiN{GEOMSELECTCOLOR}
\hiteminiN{GEOMSHOW}
\hiteminiN{SHOWTRIANGLECOUNT}
\hiteminiN{SHOWTRIANGLES}
\hiteminiN{SHOWTETRAS}
\elist

%%-----------------------------------------------------------------------
\subsection{2D Plot Settings}
\hiteminiN{SHOWDEVICEPLOTS}
\hiteminiN{SHOWGENPLOTS}
\hiteminiN{GENPLOTLABELS}
\hiteminiN{SHOWHRRPLOT}
\hiteminiN{SHOWSLICEPLOT}

%%-----------------------------------------------------------------------
\subsection{Visibility Settings}
\blist
\hiteminiN{SHOOTER}
\hitemini{SHOWAVATAR}Specifies whether to show an avatar.
\begin{lstlisting}
SHOWAVATAR
  show_avatar (0/1)
\end{lstlisting}

\hitemini{SHOWBOUNDS}Specify if bounds are output for each mesh. Usage:
\begin{lstlisting}
  SHOWBOUNDS
  bounds_each_mesh (0/1) show_bound_diffs (0/1)
\end{lstlisting}

\hiteminiN{SHOWCADOPAQUE}

\hitemini{SHOWCHID}When drawing the title, specify whether the CHID is drawn.
\begin{lstlisting}
  SHOWCHID
  show_chid (0/1)
\end{lstlisting}

\hitemini{SHOWCVENTS}Specify whether circular vents are drawn. Usage:
\begin{lstlisting}
  SHOWCVENTS
  show_vent (0/1) show_outline (0/1)
\end{lstlisting}


\hiteminiN{SHOWDEVICES}
\hiteminiN{SHOWDEVICEVALS}
\hiteminiN{SHOWEVACSLICES}
\hiteminiN{SHOWFEDAREA}
\hiteminiN{SHOWFIRECUTOFF}
\hiteminiN{SHOWFRAMETIMELABEL}
\hiteminiN{SHOWGRAVVECTOR}
\hiteminiN{SHOWGRID}
\hiteminiN{SHOWHRRLABEL}
\hiteminiN{SHOWMISSINGOBJECTS}
\hiteminiN{SHOWOTHERVENTS}
\hiteminiN{SHOWROOMS}
\hiteminiN{SHOWSLICEVALS}
\hiteminiN{SHOWSZONE}
\hiteminiN{SHOWTARGETS}
\hiteminiN{SHOWTHRESHOLD}
\hiteminiN{SHOWTRACERSALWAYS}
\hiteminiN{SHOWTRANSPARENT}
\hiteminiN{SHOWVENTFLOW}
\hiteminiN{SHOWZONEFIRE}
\elist


%%-----------------------------------------------------------------------
%%\subsection{Data Settings}
\hiteminiN{FREEZEVOLSMOKE}
\hiteminiN{AVATAREVAC}
\hiteminiN{HRRPUVCUTOFF}
\hiteminiN{FED}
\hiteminiN{FEDCOLORBAR}
\hiteminiN{FIRECOLORMAP}
\hiteminiN{FIREPARAMS}
\hiteminiN{P3CONT3DSMOOTH}
\hiteminiN{PART5COLOR}
\hiteminiN{PART5PROPDISP}
\hiteminiN{PARTAUTO}
\hiteminiN{partclassdataVIS}
\hiteminiN{PARTFAST}
\hiteminiN{PATCHAUTO}
\hiteminiN{PATCHDATAOUT}
\hiteminiN{PERCENTILEMODE}
\hiteminiN{PLOT2DHRRBOUNDS}
\hiteminiN{PLOT3DAUTO}
\hiteminiN{PROPINDEX}
\hiteminiN{SMOKE3DCUTOFFS}
\hiteminiN{SMOKEALBEDO}
\hiteminiN{SMOKECOLOR}
\hiteminiN{SMOKEFIREPROP}
\hiteminiN{SMOKELOAD}
\hiteminiN{SMOKEPROP}
\hiteminiN{RESEARCHMODE}
\hiteminiN{S3DAUTO}
\hiteminiN{UNITCLASSES}
\hiteminiN{UNLOAD_QDATA}
\hiteminiN{BOUNDARYMESH}
\hiteminiN{BOUNDARYTWOSIDE}
\hiteminiN{CELLCENTERTEXT}
\hiteminiN{GSLICEPARMS}
\hiteminiN{GVECDOWN}
\hiteminiN{HISTOGRAM}
\hiteminiN{LOADFILESATSTARTUP}
\hiteminiN{SLICEAUTO}
\hiteminiN{SLICEAVERAGE}
\hiteminiN{SLICEDUP}
\hiteminiN{SLICESKIP}
\hiteminiN{SMOKESENSORS}
\hiteminiN{MSLICEAUTO}
\hiteminiN{VECCONTOURS}
\hiteminiN{VISBOUNDARYTYPE}
\hiteminiN{CONTOURTYPE}
\hiteminiN{ISOAUTO}
\hiteminiN{LOADINC}
\hiteminiN{SKIPEMBEDSLICE}

%%-----------------------------------------------------------------------
\subsection{Time and data bounds}
\hiteminiN{C_ISO}
\hiteminiN{V_ISO}
\hiteminiN{VOLSMOKE}
\hiteminiN{V_PARTICLES}
\hiteminiN{V_PLOT3D}
\hiteminiN{V_SLICE}
\hiteminiN{VSLICEAUTO}
\hiteminiN{V_TARGET}
\hiteminiN{V_ZONE}
\hiteminiN{V2_BOUNDARY}
\hiteminiN{V2_PARTICLES}
\hiteminiN{V2_SLICE}
\hiteminiN{V_BOUNDARY}



\label{section:timedatabounds}
This section describes the ini file keywords used by Smokeview:
\begin{enumerate}
\item to set the time interval for loading data (the TLOAD keyword),

\item to discard or chop data based upon data values (keywords beginning with C\_), and

\item to specify minimum and maximum data bounds
for computing color indices (keywords beginning with {\tt V2\_}).
Color indices are computed using an equation of the form
\begin{eqnarray*}
\mbox{color index} = 255\frac{\mbox{data~value - valmin}}{\mbox{valmax - valmin}}
\end{eqnarray*}
The user may specify {\tt valmin}\  and {\tt valmax}\  or specify
that Smokeview compute these values using global bounds
or percentile bounds from a histogram of the data that it also computes.

\end{enumerate}

\noindent Most of the keywords in this section have the form:
\begin{lstlisting}
KEYWORD
  setvalmin valmin setvalmax valmax quantity_label
\end{lstlisting}
\noindent where {\tt setvalmin}\ and {\tt setvalmax} are integer parameters
that determine how {\tt valmin}\ and {\tt valmax}\
are treated and the parameter {\tt quantity\_label}\ is the
file data quantity the bound parameters are applied to.

The {\tt setvalmin} and {\tt setvalmax} parameters can have values of  0 or 1 when used with
the {\tt TLOAD}\ and {\tt C\_filetype} keywords.
If {\tt setvalmin}\ is 0 then {\tt valmin}\ is ignored.
If {\tt setvalmin}\ is 1 then {\tt valmin}\ is used to load data.
The same for {\tt setvalmax}.
{\tt V2\_}\ keyword parameters have more values and are described in  Table \ref{tabV2bounds}.

\newcommand{\tabentry}[1]{\parbox[c]{3.5in}{\vspace{0.05in}#1\vspace{0.05in}}}
\begin{table}[bph]
\begin{center}
\caption[Description of parameters used by the V2\_filetype ini keywords.]
{Description of parameters used by the V2\_filetype  ini keywords where filetype may be
BOUNDARY, PARTICLES, PLOT3D or SLICE. The parameters {\tt valmin}\  and {\tt valmax}\  are obtained from
the ini file or are computed by Smokeview depending on the
value of {\tt setvalmin}\  and {\tt setvalmax}\ .}\ \vspace{0.1in}
\begin{tabular}{|l|l|l|}
\multicolumn{3}{l}{{V2\_filetype}} \\
\multicolumn{3}{l}{{\tt setvalmin valmin setvalmax valmax quantity\_label} } \\
\hline Parameter &  type & Description  \\
\hline {\tt setvalmin}\ /{\tt set{\tt valmax}\ }\  &  integer & \tabentry{
0 - use {\tt valmin/valmax}\  for the minimum/maximum bound when computing color indices\\
1 - Smokeview computes {\tt valmin/valmax}\  using the minimum/maximum of data in loaded files
with quantity {\tt quantity\_label}\  \\
2 - Smokeview computes {\tt valmin/valmax}\  using the minimum/maximum of data in both loaded and unloaded files
with the quantity {\tt quantity\_label}\  \\
3 - Smokeview computes a histogram of the data and uses a percentile bound, default 1\%
for {\tt valmin}\  and 99\% for {\tt valmax}\ .
The default values may be changed using the
  \frame{data distribution}\ tab of the bounds dialog box.
}.  \\

\hline {\tt valmin/valmax}\  &  float &
\tabentry{If {\tt setvalmin}\  or {\tt setvalmax}\ are   0,
use {\tt valmin}\ or {\tt valmax}\ for converting data values to color indices.
If they  are not 0 then smokeview computes these bounds.
}  \\

\hline {\tt quantity\_label}\  & character &
\tabentry{The parameter {\tt quantity\_label}\ is the label appearing in the
Smokeview colorbar, for example {\tt temp}\ for the quantity {\tt TEMPERATURE}.
This parameter is optional.  If it is not blank then {\tt setvalmin, valmin, setvalmax}\ and {\tt valmax}\
are applied to filetype files with quantity {\tt quantity\_label}.
If it is blank then these parameters are applied to all filetype files.
Possible values of {\tt quantity\_label}\ are given in Table \ref{tabBNDF} for the {\tt V2\_BOUNDBOUNDS}\ ini keyword and
Table \ref{tabSLCF} for the {\tt V2\_SLICEBOUNDS}\ ini keyword. The {\tt quantity\_label}\ label
is also found in the .smv file.
}\\
\hline

\end{tabular}
\label{tabV2bounds}
\end{center}
\end{table}

\blist

\hitemini{C\_BOUNDARY}\ Defines the minimum and maximum values used to
discard boundary file data in a visualization. To discard boundary data
below 70\degC\ and above 200\degC\ use:

\begin{lstlisting}
C_BOUNDARY
  1 70. 1 200. temp
\end{lstlisting}

\hitemini{C\_PARTICLES}\ Defines the minimum and maximum values used
to discard particle data in a visualization. To discard particle data
below 70\degC\ and above 200\degC\ use:
\begin{lstlisting}
C_PARTICLES
  1 70. 1 200. temp
\end{lstlisting}

\hitemini{C\_PLOT3D}\ Defines the minimum and maximum data values used
to discard or chop Plot3D data.  To cause Smokeview to set the minimum and
maximum chop values for the first Plot3D quantity (usually
temperature) to 100 and 300 use:

\begin{lstlisting}
C_PLOT3D
 5
1 1 100.0 1 300.0
2 0 1.0 0 0.0
3 0 1.0 0 0.0
4 0 1.0 0 0.0
5 0 1.0 0 0.0
\end{lstlisting}

\hitemini{C\_SLICE}\ Defines the minimum and maximum values used to
discard slice file data in a visualization. To discard slice data
below 70\degC\ and above 200\degC\ use:

\begin{lstlisting}
C_SLICE
  1 70. 1 200. temp
\end{lstlisting}

\hitemNULL{CACHE\_BOUNDARYDATA}

\hitemNULL{CACHE\_QDATA}

\hitemNULL{PATCHDATAOUT}

\hitemini{PERCENTILELEVEL} Defines the minimum and maximum percentile levels for setting data bounds.
The defaults are 0.01 for the minimum level and 0.99 for maximum level.

\begin{lstlisting}
PERCENTILLEVEL
 min_level max_level
\end{lstlisting}

\hitemNULL{SLICEDATAOUT}

\hitemNULL{TIMEOFFSET}

\hitemini{TLOAD}\ Defines the minimum and maximum time used to
load data in a visualization and to also skip data.
The general form of this keyword is
\begin{lstlisting}
TLOAD
  setvalmin valmin setvalmax valmax setskip skip
\end{lstlisting}

\noindent where data is not loaded for time
less than {\tt valmin}\ if {\tt setvalmin}\ is 1
and data is not loaded for time greater than {\tt valmax}\ if {\tt setvalmax}\ is 1.
Every {\tt skip}\ time frame is loaded if {\tt setskip}\ is 1.  To discard data before
10~s and after 50~s and to load every third time frame use:

\begin{lstlisting}
TLOAD
 1 10. 1 50. 1 3
\end{lstlisting}

\hitemini{V2\_BOUNDARY, V2\_PARTICLES, V2\_SLICE}Defines the minimum and maximum
data bounds used for converting boundary, particle or slice file data values to color indices.
The format of these keyword is
\begin{lstlisting}
V2_BOUNDARY
 setvalmin valmin setvalmax valmax quantity_label
\end{lstlisting}
where the parameters
{\tt setvalmin}, {\tt valmin}, {\tt setvalmax}, {\tt valmax}\ and {\tt quantity\_label}.
are described in Table \ref{tabV2bounds}.
The format of the {\tt V2\_PARTICLES}\ and {\tt V2\_SLICE}\ keywords is the same.
The parameter {\tt quantity\_label}\ is optional.  If it is blank then the bound
parameters are applied to all boundary, particle or slice files.
If it is not blank they
are applied to only boundary, particle or slice files with quantity {\tt quantity\_label}.
Possible quantity labels are given in Table \ref{tabBNDF} for boundary files and Table \ref{tabSLCF} or slice files.
Quantity labels may also be found in the .smv file for those quantities used
on a {\tt \&BNDF, \&PART}\ or {\tt \&SLCF}\ namelists in the FDS input file.

To specify boundary file bounds for temperature ranging from\
30.0~\degC\ to 600.0~\degC\ use:
\begin{lstlisting}
V2_BOUNDARY
0 30.000000 0 600.000000 temp
\end{lstlisting}
where {\tt temp}\ is the Smokeview colorbar label displayed when
showing a temperature boundary file.
To specify particle file bounds for the U component of velocity ranging from
-1.0~m/s to 2.0~m/s use:
\begin{lstlisting}
V2_PARTICLES
0 -1.0 0 2.0 u
\end{lstlisting}
where {\tt u}\ is the Smokeview colorbar labels displayed when
showing the particle file and the U component of velocity.
To specify slice file bounds for temperature ranging from
30.0~\degC\ to 600.0~\degC\ use:
\begin{lstlisting}
V2_SLICE
0 30.000000 0 600.000000 temp
\end{lstlisting}
where {\tt temp}\ is the Smokeview colorbar labels displayed when
showing a temperature slice file.
To specify slice file bounds for a minimum temperature of
30.0~\degC\ and the global maximum temperature use:
\begin{lstlisting}
V2_SLICE
0 30.000000 2 1.0 temp
\end{lstlisting}

\input{BNDFlabels.tex}

\input{SLCFlabels.tex}

\hitemini{V2\_PLOT3D}Defines the minimum and maximum
data bounds used for converting plot3D file data component values to color indices.
The format of this keyword is
\begin{lstlisting}
V2_PLOT3D
 ncomp
 component_index setvalmin valmin setvalmax valmax
 ....
 component_index setvalmin valmin setvalmax valmax
\end{lstlisting}
where ncomp is the number of entries,
component\_index is either 1 to 5 or 1 to 6 if speed is one of the components of
the plot3D file  and
the parameters {\tt setvalmin}\ , {\tt valmin}\ ,  {\tt setvalmax}\ , {\tt valmax}\  are the same
as those used with the {\tt V2\_BOUNDARY, V2\_PARITICLES} and {\tt V2\_SLICE}\ keywords and are
described in Table \ref{tabV2bounds}. Smokeview uses the {\tt component\_index}\ not {\tt quantity\_label}\
for determining which quantity to apply bounds to.
Assuming the first component is temperature, to specify file bounds ranging from 20.0~\degC\ to
600.0~\degC\ for the first component and global bounds for the other components use:
\begin{lstlisting}
V2_PLOT3D
 5
 1 0 20.0 0 600.0
 2 2  1.0 2 0.0
 3 2  1.0 2 0.0
 4 2  1.0 2 0.0
 5 2  1.0 2 0.0
\end{lstlisting}

\hitemNULL{V\_TARGET}

\elist

%%-----------------------------------------------------------------------
\subsection{Data loading}
\label{SECTDATALOADING}\ The keywords in this section may be used
to reduce the memory required to visualize FDS data. Keywords
exist for limiting particles and frames. Other keywords exist for
compressing particle data and skipping particles and frames.
\blist

\hitemini{BOUNDZIPSTEP}\ Defines the number of intervals or steps between
boundary file frames when
compressed by Smokezip. (default: 1)

\hitemNULL{FED}

\hitemNULL{FEDCOLORBAR}

\hitemini{ISOZIPSTEP}\ Defines the number of intervals or steps between
isosurface file frames when
compressed by Smokezip. (default: 1)

\hitemini{NOPART}Indicates that a particle file should not (1) or
should (0) be loaded when Smokeview starts up. This option is used
when one wants to look at other files besides the particle file.
(default: 1)

\hitemNULL{SHOWFEDAREA}

\hitemNULL{SLICEAVERAGE}

\hitemini{SLICEDATAOUT}When set to 1 will output data corresponding
to any loaded slice files whenever the scene is rendered.

\hitemini{SLICEZIPSTEP}\  Specifies the number of intervals or steps between
slice frames when compressed
by Smokezip.  (default: 1)

\hitemini{SMOKE3DZIPSTEP}

\hitemNULL{USER\_ROTATE}
\elist



%%-----------------------------------------------------------------------
\subsection{View}
\label{sect:iniview}
The keywords in this section define how a scene is viewed.
Keywords exist for showing or hiding various scene elements
and for modifying how various scene elements appear.

\blist

\hitemini{APERTURE}\ Specifies the viewing angle used to
display a Smokeview scene.  Viewing angles of 30, 45, 60,
75 and 90 degrees are displayed when APERTURE has the value
of 0, 1, 2, 3 and 4 respectively.  (default: 2)

\hitemini{BLOCKLOCATION}\ Specifies the location or method used to
draw blockages.  Blockages are drawn either snapped to the nearest
grid (5), drawn at locations as specified in the FDS input file
(6) or drawn as specified in a compatible CAD package (7)\footnote{There are various
third party tools that have been developed to help process obstruction data for FDS. See
the FDS/Smokeview website, \hhref{http://pages.nist.gov/fds/}, for details.}.   (default: 5)

\hitemNULL{BOUNDARYTWOSIDE}

\hitemini{CLIP}\ Specifies the near and far clipping plane
locations.  Dimensions are relative to the longest side of
an FDS scene. (default: 0.001 3.000)

\hitemNULL{CONTOURTYPE}

\hitemini{COMPRESSAUTO}\ Specifies that files that are to be auto-loaded
by Smokeview (and no other files) should be compressed by Smokezip.

\hitemini{CULLFACES}Hide (1) or show (0) the back side of various surfaces.

\hitemini{EYEVIEW}Specifies whether
the scene should be rotated relative to the observer ({\tt EYEVIEW}\ set to 1)
or the scene center ({\tt EYEVIEW}=0).
(default: 0)

\hitemini{FONTSIZE}\ Specifies whether small (0) or large (1)
fonts should be used to display text labels. (default: 0)

\hitemini{FRAMERATEVALUE}Specifies the maximum rate (frames per
second) that Smokeview should display frames. This value is an
upper bound.  Smokeview will not display frames faster than this
rate but may display frames at a slower rate if the scene to be
visualized is complex. (default: 1000000 (essentially unlimited))

\hitemini{GCOLORBAR}Specifies colorbars created with the colorbar editor.
The format of this keyword is
\begin{lstlisting}
GCOLORBAR
 ncolobars
 name1                        first colorbar
 m_nodes hilight_node_index
 index1  r1 g1 b1
 ...
 index_n rn gn bn
 ....
 ....

\end{lstlisting}
where ncolorbars is the number of colorbars, m\_nodes is the number of nodes
(can be different for each colorbar), index\_i, ri, gi and bi are the red, green, blue components for
node i.


\hitemini{ISOTRAN2}Specifies the transparency state for iso-surfaces. The choices
are
\begin{itemize}
\item all iso-surface levels are transparent (ALL\_TRANSPARENT=1),
\item the minimum iso-surface level is solid (MIN\_SOLID=2),
\item the maximum iso-surface level solid (MAX\_SOLID=3) (default)
\item all iso-surface levels transparent (ALL\_TRANSPARENT=4)
\end{itemize}


\hitemini{MSCALE}\ Specifies how dimensions along the X, Y
and/or Z axes should be scaled. (default: 1.0 1.0 1.0)

\hitemini{PROJECTION}\ Specifies whether a perspective (0) or orthographic (1) projection
is used to draw Smokeview scenes.
(default: 0)

\hitemini{P3DSURFACETYPE}Specifies how Plot3D isosurfaces should be
drawn.  If P3DSURFACETYPE is set to 1 then Plot3D isosurfaces are
drawn using shaded triangles.  If P3DSURFACETYPE is set to 2 or 3
then Plot3D isosurfaces are drawn using triangle outlines and
points respectively. (default: 1)

\hitemini{P3DSURFACESMOOTH}When drawing Plot3D isosurfaces using
shaded triangles, this option specifies whether the vertex normals
should be averaged (P3DSURFACESMOOTH set to 1) resulting in smooth
isosurfaces or not averaged resulting in isosurfaces that have
sharp edges (P3DSURFACESMOOTH set to 0). (default: 1)

\hitemini{RENDERFILETYPE}Specifies whether PNG (RENDERFILETYPE set to 0) or
JPEG (RENDERFILETYPE set to 1)
should be used to render images.  (default: 1)

\hitemini{RENDEROPTION}\ Records the option used to render images.

\hitemini{SENSORRELSIZE}\ Specifies a scaling factor that is applied when
drawing all sensors.  (default: 1.0)

\hitemini{SHOWAXISLABELS}\ Specifies whether axis labels should
be drawn (1) or not drawn (0) drawn. (default: 0)

\hitemini{SHOWBLOCKLABEL}\ Specifies whether a label identifying the
active mesh should be drawn (1) or not drawn (0). (default: 1)

\hitemini{SHOWBLOCKS}Specifies how a blockage should
be drawn.  A value of 0, 1 or 2 indicates that the blockages are
invisible, drawn normally or drawn as outlines respectively.  (default: 1)

\hitemini{SHOWCEILING}Specifies whether
the ceiling (upper bounding surface) should be drawn (1) or not drawn (0).
(default: 0)


\hitemini{SHOWCOLORBARS}Specifies whether
the colorbars should be drawn (1) or not drawn (0).
(default: 1)

\hitemini{SHOWEXTREMEDATA}Specifies whether data exceeding the maximum colorbar
label value or less than the minimum colorbar label value should be colored
with a different color (black).  If SHOWEXTREMEDATA is set to 1 then extreme
data is colored black, if SHOWEXTREMEDATA is set to 0 then extreme data is
colored as indicated by the maximum and minimum region of the colorbar. (default: 0).  .

\hitemini{SHOWFLOOR}Specifies whether
the floor (lower bounding surface) should be drawn (1) or not drawn
(0).
(default: 1)

\hitemini{SHOWFRAME}\ Specifies whether the frame surrounding
the scene should be drawn (1) or not drawn (0). (default:
1)

\hitemini{SHOWFRAMELABEL}\ Specifies whether the frame number
should be drawn (1) or not drawn (0). (default: 1)

\hitemini{SHOWFRAMERATE}Specifies whether
the frame rate label should be drawn (1) or not drawn (0).
(default: 0)

\hitemini{SHOWGRIDLOC}\ Specifies where grid location should be drawn (1) or not
drawn (0). (default: 0)

\hitemini{SHOWHRRCUTOFF}Specifies whether the HRRPUV cutoff label should be (1)
or should not be (0) displayed. (default: 0)

\hitemini{SHOWISO}Specifies how an isosurface should be drawn: hidden (0),
solid (1), outline (2) or with points (3).  (default: 1)

\hitemini{SHOWISONORMALS}\ Specifies whether iso-surface normals are drawn
(1) or not drawn (0).  (default: 0)

\hitemini{SHOWIGNITION}\ When drawing a temperature boundary file, this option
specifies whether ignited materials (regions exceeding the materials
ignition temperature) should be drawn (1) or not drawn (0).
A second parameter specifies whether only the ignited regions
should be drawn (1) or both the ignited regions and other regions should be drawn (0).
(default: 0 0)


\hitemini{SHOWLABELS}Specifies whether labels should be drawn (1) or
not drawn (0).  Labels are specified using the {\em LABEL}\
keyword described in subsection \ref{subsect_features}. (default:
0)

\hitemini{SHOWMEMLOAD}\ Specifies (when run on a PC) whether a label
giving the memory used should be drawn (1) or not drawn (0).
(default: 0)

\hitemini{SHOWOPENVENTS}\ Specifies that open vents should be drawn
(1) or not drawn (0).  (default: 0)

\hitemini{SHOWDUMMYVENTS}\ Specifies that dummy vents (vents created
by FDS) should be drawn (1) or not drawn (0).  (default: 0)

\hitemini{SHOWSENSORS}\ Specifies whether sensors should be drawn (1) or not drawn (0).
A second parameter specifies whether the sensor's orientation or normal vector
should be drawn (1) or not drawn (0). (default: 1 0).

\hitemini{SHOWSLICEINOBST}\ Specifies whether a slice file should be drawn (1) inside
a blockage or not drawn (0) inside a blockage.  Normally a slice file
is not drawn inside a blockage but one would want to draw a
slice file inside a blockage if the blockage disappears over the
duration of a run.  (default: 0).


\hitemini{SHOWSMOKEPART}Specifies whether smoke or trace particles
should be drawn (1) or not drawn (0). (default: 1)

\hitemini{SHOWSPRINKPART}Specifies whether
sprinkler droplet particles (if present in the particle file)
should be drawn (1) or not drawn (0).
(default: 1)

\hitemini{SHOWSTREAK}Specify parameters that define streak properties.  This keyword
has four integer parameters with format:
\begin{lstlisting}
SHOWSTREAK
streak5show,streak5step,showstreakhead,streakindex
\end{lstlisting}
The {\tt streak5show}\ parameter may be 0 or 1 and indicates whether a
streak is not (0) or is (1) shown. The {\tt streak5step}\ parameter
indicates number of streaks skipped or not displayed.
The {\tt showstreakhead}\ parameter may be 0 or 1 and indicates
whether a streak head is not (0) or is (1) shown.
The {\tt streakindex}\ parameter indicates the length of the streak.


\hitemini{SHOWTERRAIN}\ If terrain is present, specifies that terrain
should be visualized as a {\em warped}\ sheet
rather than as a set of FDS blockages.

\hitemini{SHOWTICKS}\ Specifies whether labels should be drawn (1) or
not drawn (0).  Ticks are specified using the {\em TICK}\ keyword
described in subsection \ref{subsect_features}. (default: 0)

\hitemini{SHOWTIMEBAR}Specifies whether the timebar should be drawn
(1) or not drawn (0). (default: 1)

\hitemini{SHOWTIMELABEL}\ Specifies whether the time label should be
drawn (1) or not drawn (0). (default: 1)

\hitemini{SHOWTRANSPARENTVENTS}\ Specifies whether vents specified as being invisible should be
shown (1) or not shown (0). (default: 0)..

\hitemini{SHOWHMSTIMELABEL}Specifies whether the time label should be
drawn (1) or not drawn (0) using the format ``h:m:s'' where ``h''
is hours, ``m'' is minutes and ``s'' is seconds.  (default: 0)

\hitemini{SHOWTITLE}Specifies whether
the title should be drawn (1) or not drawn (0).
(default: 1)

\hitemini{SHOWVENTS}\ Specifies whether vents should be drawn
(1) or not drawn (0).  (default: 1)

\hitemini{SHOWALLTEXTURES}\ If wall textures are defined in the input .smv file, this
option specifies whether to draw (1) or not to draw (0) wall textures.
(default: 0)


\hitemini{SHOWWALLS}Specifies whether
the four walls (four vertical bounding surfaces) should be drawn (1) or not drawn (0).
(default: 1)

\hitemini{SURFINC}Smokeview allows one to display two Plot3D
isosurfaces simultaneously.  The SURFINC parameter specifies the
interval between displayed Plot3D surface values. (default: 0)

\hitemini{TERRAINPARMS}\ Specifies various parameters used to characterize
how terrain appears.  The parameters are the color at the minimum depth,
the color at the maximum depth and scaling factor used to vertically exaggerate
the scene.
\begin{lstlisting}
  TERRAINPARMS
  terrain_rgba_zmin[0],terrain_rgba_zmin[1],terrain_rgba_zmin[2];
  terrain_rgba_zmax[0],terrain_rgba_zmax[1],terrain_rgba_zmax[2];
  vertical_factor);
\end{lstlisting}

\hitemini{TIMEOFFSET}\ Specifies that offset time in seconds added to
the displayed simulation time.  Along with the {\em
SHOWHMSTIMELABEL}\ keyword, the {\em TIMEOFFSET}\ keyword allows
one to display {\em wall clock}\ rather than simulation time.
(default: 0.0)

\hitemini{TITLESAFE}Amount in pixels to offset titles when displaying
scene in {\em title safe}\ mode.   (default: 0)

\hitemini{TRANSPARENT}Specifies whether 2D and 3D contours should be
drawn with solid colors (0) or transparent colors(1). (default: 1)

\hitemini{TWOSIDEDVENTS}\  Specifies whether to draw vents so that they are visible from
both sides (1) or visible from only one side (0).  (default: 0)

\hitemini{USEGPU}\ If the GPU is available, specifies whether it should be used. (default: 1)

\hitemini{VECTORSKIP}Specifies
what vectors to draw.  For example, if this parameter is set to 2 then
every 2nd vector is drawn when displaying vectors.
(default: 1)

\hitemini{VIEWPOINT5}Specifies the internal Smokeview parameters used
to record a scene's viewpoint and orientation.  This parameter is
set automatically by Smokeview when a .ini file is created.
(default: none)

{\small
\begin{lstlisting}
VIEWPOINT5
 eyeview,rotation_index,view_id
 eye\_x,eye\_y,eye\_z,zoom,zoomindex
 view_angle,direction_angle,elevation_angle,projection_type
 xcen,ycen,zcen
 angle_zx[0],angle_zx[1]
 mat[0],mat[1],mat[2],mat[3]
 mat[4],mat[5],mat[6],mat[7]
 mat[8],mat[9],mat[10],mat[11]
 mat[12],mat[13],mat[14],mat[15]
 xyz_clipplane,clip_x,clip_y,clip_z,clip_X,clip_Y,clip_Z
 clip_x_val,clip_y_val,clip_z_val,clip_X_val,clip_Y_valclip_Z_val
 name
\end{lstlisting}
}

\begin{itemize}
\item eyeview - view method type (0 - general rotations,
1 - first person movement, 2 - level rotations
\item eye - coordinates of viewing position
\item xcen, ycen, zcen - coordinates of view direction
\item mat - viewing transformation matrix
\item xyz\_clipplane - global clipping flag (on=1, off=0)
\item clip\_x, clip\_y, clip\_z - min clipping plane flag ((on=1, off=0)
\item clip\_X, clip\_Y, clip\_Z - maxn clipping plane flag ((on=1, off=0)
\item clip\_x\_val, clip\_y\_val, clip\_z\_val - min clipping plane values
\item clip\_X\_val, clip\_Y\_val, clip\_Z\_val - max clipping plane values
\item name - label appearing in Viewpoint menu
\end{itemize}

\hitemini{XYZCLIP}\ Specifies clip planes in physical
coordinates. There are six clipping planes, a minimum and
maximum X, a minimum and maximum Y, a minimum and maximum
Z. Each clipping plane may be used or not. The first
parameter is 1 or 0 and specifies whether clipping is
turned on or off. The next three lines specify clipping
parameters for the X, Y and Z axes.  Each line has the
format

\begin{lstlisting}
minflag min-clipval maxflag max-clipval
\end{lstlisting}

where the two flags, minflag and maxflag are 1 if turned on or 0
if turned off. Clipping is specified with the Clipping
dialog box found under the {\em Options}\ menu item.
 (default:
\begin{lstlisting}
0
0 0.0 0 0.0
0 0.0 0 0.0
0 0.0 0 0.0
\end{lstlisting}

\hitemini{ZOOM}\ Specifies the zoom amount used to display a Smokeview
scene using two parameters, an integer zoom index and a floating
point zoom amount.  If the zoom index is 0->4 then the zoom amount
is 0.25, 0.5, 1.0, 2.0 and 4.0 respectively.  If the zoom index is
negative then the second parameter is used to specify the zoom
amount. (default: 0 1.0)

\elist

%%-----------------------------------------------------------------------
%\subsection{Miscellaneous}

%%-----------------------------------------------------------------------
%\subsection{3D Smoke}

%%-----------------------------------------------------------------------
%\subsection{Zone fire}

%%-----------------------------------------------------------------------
\subsection{Tour}
\blist

\hitemini{SHOWPATHNODES}\ Specifies whether the path nodes should (1) or should not (0)
be drawn.  This is a debugging parameter, not normally used.  (default: 0)

\hitemini{SHOWTOURROUTE}\ Specifies whether the tour route should (1) or should not (0)
be drawn.  (default: 0)

\hitemini{TOURS}Keyword used to specify the tours. The format is
\begin{lstlisting}
TOURS
ntours - number of tours
label for tour
nkeyframes - number of keyframes for first tour
time xpos ypos zpos 1 az elev bias continuity tension zoom localspeedflag
...
time xpos ... for last keyframe
nkeyframes for 2nd tour
...
...
\end{lstlisting}
If a Cartesian view direction is specified then instead of {\tt 1
az elev}\ above use {0 xview yview zview}\ where xview, yview, zview
are the coordinates of the view direction. The {\em Circle}\ tour
is not stored in the .ini file unless it has been changed by the
user.  The tour entry created by using the \frameit{Add}\ button
in the Tour dialog box is given by
\begin{lstlisting}
TOURS
1
Added Tour 1
2
0.0 -1.0 -1.0 -1.0 0 0.0 0.0 0.0 0.0 0.0 1.0  0
100.0 7.4 9.0 7.4 0 0.0 0.0 0.0 0.0 0.0 1.0 0
\end{lstlisting}

\hitemini{TOURCOLORS}Keyword used to specify the tour colors.  The colors as before
consist of a red, green and blue component ranging from 0.0 to 1.0 .  One can override
Smokeview's choice for the path, the path knots for both the selected and un-selected
case.  One may also specify the color of the time labels and the location of the
object or avatar on the tour at the current time. The foreground color is used when a
color component less than 0.0 is specified.
Default:

\begin{lstlisting}
TOURCOLORS
1.000000 0.000000 0.000000   :selected path line
1.000000 0.000000 0.000000   :selected path line knots
0.000000 1.000000 0.000000   :selected knot
-1.000000 -1.000000 -1.000000   :path line
-1.000000 -1.000000 -1.000000   :path knots
-1.000000 -1.000000 -1.000000   :text
1.000000 0.000000 0.000000   :avatar
\end{lstlisting}

\hitemini{VIEWALLTOURS}Specifies whether all (1) tours should be
drawn. (default: 0)

\hitemini{VIEWTIMES}Specifies the tour start time, tour stop time and
number of points to specify a tour. (default: 0.0 100.0 1000)

\hitemini{VIEWTOURFROMPATH}Specifies whether the scene should (1) or
should not (0) be observed from the selected tour.

\elist

%%-----------------------------------------------------------------------
\subsection{Realistic Smoke Parameters}

\blist

\hitemini{FIRECOLOR}\ Specifies the color of the fire in red, green,
blue coordinates.  Each color component is an integer ranging from
0 to 255.  (default: 255, 128, 0)

\hitemini{FIREDEPTH}\ Specifies the depth  at which the fire is 50
percent transparent. (default: 2.0)


\hitemini{SMOKECULL}\ Cull (or do not draw) smoke if it is outside of
the viewing frustum.  (default: 1)

\hitemini{SMOKESKIP}\ To speed up smoke drawing, spatial frames may be
skipped.  Allowable parameters are (0, 1, 2).  (default: 0)

 \elist

%%-----------------------------------------------------------------------
\subsection{Zone Fire Modeling Parameters}
\blist
\hitemini{SHOWHZONE}Specifies whether upper layer temperatures should be (1) drawn horizontally
or not (0).  (default: 0)
\hitemini{SHOWVZONE}Specifies whether upper layer temperatures should be (1) drawn vertically
or not (0).  (default: 1)
\hitemini{SHOWHAZARDCOLORS}\ Specifies whether upper layer temperatures should be (1) drawn
in terms of hazard or drawn in terms of a standard color scale (0).  (default: 0)

\elist

%%-----------------------------------------------------------------------
\subsection{Local Parameters}

\blist
\hitemini{SCRIPTFILE}\ Specifies the name of a script file either created by
hand or created automatically
by Smokeview using the script recorder.
\elist

%---------------------------------------------------------------------------------
%---------------------------------------------------------------------------------
\section{Smokeview Parameter Input File (.smv file)}
\label{sectionsmv}\ The FDS software outputs simulation results
into the Smokeview input file with extension {\tt .smv}\ and
various  output data files whose format is documented in the next
section. A {\tt .smv}\ file is a formatted ascii text file
consisting of a set of KEYWORDs followed by DATA describing the
FDS case's geometry, data file names and contents, sensor
information, etc.

%%-----------------------------------------------------------------------
\subsection{Geometry Keywords}
\blist

\hitemsmv{GRID}This keyword specifies the number of grid cells in the
{\tt X}, {\tt Y}\ and {\tt Z}\ directions.
For example,
\begin{lstlisting}
GRID
10 20 30
\end{lstlisting}
specifies that there are 10, 20 and 30 grid cells in the {\tt X},
{\tt Y}\ and {\tt Z}\ directions respectively.

\hitemsmv{OFFSET}This keyword specifies signals to Smokeview that a new mesh has begun and also
gives values for the front, left bottom corner of the mesh.  For example,
\begin{lstlisting}
OFFSET
xmin, ymin, zmin
\end{lstlisting}

Note that the xmin, ymin and zmin values must be identical to the corresponding values given in
the {\tt PDIM}\ keyword. The {\tt OFFSET}\ keyword cannot be eliminated from the .smv file
(it may seem logical to do this due
to the presence of redundant data) because of its role in signaling new meshes.

\hitemsmv{PDIM}This keyword specifies the region where a mesh is located
using the same convention as is used in an FDS input file to specify a blockage location.
PDIM also specifies a color to use for drawing grids.  For example,
\begin{lstlisting}
PDIM
xmin, xmax, ymin, ymax, zmin, zmax, r, g, b
\end{lstlisting}
where {\tt (xmin,~ymin,~ymax)}\ and {\tt (xmax,~ymax,~zmax)}\
represent opposite corners of a mesh
and r, g and b represent the red, green and blue components (0.0 to 1.0) of grids drawn
in the mesh.

Note that the xmin, ymin and zmin values must be identical to the values given in the
{\tt OFFSET}\ keyword.


\hitemsmv{SHOW\_OBST(HIDE\_OBST)}\ This keyword specifies when a
blockage should be shown(hidden). For example,
\begin{lstlisting}
SHOW_OBST  2
  10 120.1
\end{lstlisting}
specifies that the tenth blockage in mesh 2 should be opened at
120.1 seconds.  This keyword is automatically added to the {\tt
.smv}\ file by FDS.

\hitemsmv{OBST}This keyword specifies internal blockages. A
FORTRAN~2003 code segment describing the format of {\tt OBST}\ data
is given by:
\begin{lstlisting}
   read(5,*)nb
   do i = 1, nb
     read(5,*)x1(i),x2(i),y1(i),y2(i),z1(i),z2(i), id(i),
         ... s1(i),...,s6(i),tx(i),ty(i),tz(i)
   end do
   do i = 1, nb
     read(5,*)ib1(i), ib2(i), jb1(i), jb2(i), kb1(i), kb2(i),
            ... colorindex(i) blocktype(i),
            ...  red(i), green(i), blue(i), alpha(i)
   end do
\end{lstlisting}
where the parameters are defined in Table \ref{tabOBST}.
The arrays {\tt x1, ..., z2}\ and {\tt ib1, ..., kb2}\ are
required. All other arrays are optional and may be omitted.

\begin{table}[bph]
\begin{center}
\caption{Descriptions of parameters used by the Smokeview OBST keyword.}\ \vspace{0.1in}
\begin{tabular}{|l|l|l|}
\hline Variable(s) &  type & Description  \\

\hline\hline
\parbox[c]{1.0in}{nb}\ &integer&
\tabentry{number of blockages or entries for the OBST keyword}\  \\ \hline

\parbox[c]{1.0in}{x1, x2\\y1,y2\\z1,z2}\ &float&
\tabentry{floating point blockage bounds}\  \\ \hline

id & integer &
\tabentry{blockage identifier}\\ \hline

\parbox[c]{1.0in}{s1, s2\\s3, s4\\s5, s6}\ & integer&
\tabentry{index of surface (SURF) used to draw blockage sides}\\ \hline

tx, ty, tz & float &
\tabentry{texture origin}\\  \hline

\parbox[c]{1.0in}{ib1, ib2\\jb1, jb2\\kb1, kb2}\ &integer &
\tabentry{Indices used to define blockage bounds in terms of grid locations.}\\ \hline

colorindex & integer &
\tabentry{Type of coloring used to color blockage.\\
-1 - default color\\ -2 - invisible\\-3 - use red, green, blue and alpha
to follow (values follow)\\n$>$0 - use n'th color table entry}\\  \hline

blocktype & integer &
\tabentry{Defines how the blockage is drawn.\\ -1 - use surface to obtain
blocktype\\0 - regular block\\2 - outline}\\  \hline

\parbox[c]{1.0in}{red, green, blue\\alpha}\ & float &
\tabentry{Each color value ranges from 0.0 to 1.0 .  The alpha {\em color}\
represents transparency, alpha=0.0 is transparent, alpha=1.0 is opaque.}\\  \hline

\end{tabular}
\label{tabOBST}
\end{center}
\end{table}

\hitemsmv{TOFFSET}The {\tt TOFFSET}\ keyword defines a default texture origin, $(x_0, y_0, z_0)$ .
This origin may be overridden with data provided with the {\tt OBST}\ keyword. For example,

\begin{lstlisting}
TOFFSET
0.0 0.0 0.0
\end{lstlisting}

\hitemsmv{TRNX,TRNY,TRNZ}The {\tt TRNX, TRNY, TRNZ}\ keywords specify
grid nodes in the {\tt X, Y, Z}\ coordinate directions.  A
FORTAN~2003 code segment describing the format of {\tt TRNX}\ data
is given by:
\begin{lstlisting}
    read(5,*)nv
    do i = 1, nv
      read(5,*)idummy
    end do
    do i = 1, nv
      read(5,*)xplt(i)
    end do
\end{lstlisting}
{\tt TRNY}\ and {\tt TRNZ}\ data entries are defined similarly. The
first {\tt nx}\ data items are not required by Smokeview.

\hitemsmv{VENT, CVENT}These keywords specify vent coordinates for regular and circular vents.
Note that the parameters {\tt x0, y0, z0}\ and {\tt radius}\ describing the center and radius
of a circular vent
only appear with the CVENT keyword.
A FORTRAN~2003
code segment describing the format of {\tt VENT}\ and {\tt CVENT}\ data is given by:
\begin{lstlisting}
   read(5,*)nv
   do i = 1, nv
     read(5,*)xv1(i), xv2(i), yv1(i), yv2(i), zv1(i), zv2(i), id(i)
      ... s1(i), tx(i), ty(i), tz(i) % x0, y0, z0, radius
   end do
   do i = 1, nv
     read(5,*)iv1(i), iv2(i), jv1(i), jv2(i), kv1(i), kv2(i)
      ... index(i), type(i), red(i), green(i), blue(i), alpha(i)
   end do
\end{lstlisting}
where the parameters are defined in Table \ref{tabVENT}.
The arrays {\tt xv1,  ..., zv2}\ and {\tt iv1, ..., kv2}\ are
required. All other arrays are optional and may be omitted.

\renewcommand{\arraystretch}{1.2}
\begin{table}[bph]
\begin{center}
\caption{Descriptions of parameters used by the Smokeview VENT and
CVENT keywords.}\ \vspace{0.1in}
\begin{tabular}{|l|l|l|}
\hline Variable(s) &  type & Description  \\

\hline\hline
\parbox[c]{1.0in}{nv}\ &integer&
\tabentry{number of vents or entries for the VENT keyword}\  \\ \hline

\parbox[c]{1.0in}{xv1, xv2\\yv1,yv2\\zv1,zv2}\ &float&
\tabentry{floating point bounds}\  \\ \hline

id & integer &
\tabentry{vent identifier}\\ \hline

\parbox[c]{1.0in}{s1}\ & integer&
\tabentry{index of surface (SURF) used to draw vent}\\ \hline

tx, ty, tz & float &
\tabentry{texture origin}\\  \hline

\parbox[c]{1.0in}{iv1, iv2\\jv1, jv2\\kv1, kv2}\ &integer &
\tabentry{Indices used to define vent bounds in terms of grid locations.}\\ \hline

index & integer &\tabentry{Type of coloring used to color vent.\\
-99 or +99 - use default color\\-n or +n - use n'th
palette color\\$<$ 0 - do not draw boundary file over vent \\
$>$ 0 - draw boundary file over vent }\\  \hline

type & integer &
\tabentry{Defines how the vent is drawn.\\
0 - solid surface \\
2 - outline\\
-2 - hidden}\\  \hline

\parbox[c]{1.0in}{red, green, blue\\alpha}\ & float &
\tabentry{Each color value ranges from 0.0 to 1.0 .
The alpha {\em color}\ represents transparency,
alpha=0.0 is transparent, alpha=1.0 is opaque.}\\  \hline

x0, y0, z0 & float &
\tabentry{circular vent origin}\\ \hline
radius & float &
\tabentry{if positive, radius of circular vent}\\ \hline

\end{tabular}
\label{tabVENT}
\end{center}
\end{table}


\hitemsmv{OPEN\_VENT(CLOSE\_VENT)}This keyword specifies when a vent
should be opened(closed). For example,
\begin{lstlisting}
OPEN_VENT 2
3 15.6
\end{lstlisting}
specifies that the third vent in mesh 2 should be opened at
15.6~S.

\elist

%%-----------------------------------------------------------------------
\subsection{File Keywords}
\blist

\hitemsmv{BNDF}The {\tt BNDF}\ keyword defines the {\tt .bf}\ file name along with character
labels used to describe the data contents of the boundary file.

\hitemsmv{HRRPUVCUT}The {\tt HRRPUVCUT}\ keyword defines the heat release
rate per unit volume cutoff value.  When displaying
realistic smoke and fire, fire is displayed above this
cutoff and smoke is displayed below.

\hitemsmv{INPF}The {\tt INPF}\ keyword specifies a file containing a
copy of the FDS input file.


\hitemsmv{ISOF}The {\tt ISOF}\ keyword defines the {\tt .iso}\ file name along with
character labels used to describe the data contents of the isosurface file.

\hitemsmv{PART,PRT5}The {\tt PART}\ and {\tt PRT5}\ keywords define
the {\tt .part}\ file name along with character
labels used to describe the data contents of the particle file.

\hitemsmv{PL3D}The {\tt PL3D}\ keyword defines the {\tt .q}
file name along with character labels used to describe the
data contents for each Plot3D variable.

\hitemsmv{SLCF,SLCT}The {\tt SLCF}\ and {\tt SLCT}\ keywords define the {\tt .sf}
file name along with character labels used to describe the
data contents of the slice file.  The {\tt SLCT}\ keyword is used
for wildland urban interface fire simulations performed over terrains.  Smokeview allows one
to visualize slice files for these types of simulations that conform to the terrain.

\elist


%%-----------------------------------------------------------------------
\subsection{Device (sensor) Keywords}
\blist

\hitemsmv{DEVICE}A {\em device}\ generalizes the notion of a sensor,
sprinkler or heat detector.  The {\tt DEVICE}\ keyword defines the
device location and name.  This name is used to access the set of
instructions for drawing the device.  These instructions are
contained in a file named {\tt objects.svo}. The default location
of this file on the PC is
C:$\backslash$Program Files$\backslash$FDS$\backslash$FDS5$\backslash$bin$\backslash$objects.svo . This
file may be customized by the user by adding instructions for
devices of their own design (i.e., custom devices may be
added to a Smokeview visualization without re-programming
Smokeview). See Chapter \ref{chap:devices}\ for more information on how to do this. The
format for the {\tt DEVICE}\ keyword is

\begin{lstlisting}
DEVICE
device_name
x y z xn yn zn 0 0 % PROP_ID
\end{lstlisting}

\noindent where device\_name is the entry in the {\tt objects.svo}\ file used
to draw the device and $(x,y,z)$ and $(xn,yn,zn)$
are the location and direction vector of the device.
The label
{\tt PROP\_ID}\ (pre-pended by a {\tt \%}) is the {\tt PROP}\ entry used to
list of other properties used
for drawing the device
(see the {\tt PROP}\ entry in this section for more details).
Note, the two {\tt 0 0}\ numbers are used for backwards compatibility.

\hitemsmv{DEVICE\_ACT}The {\tt DEVICE\_ACT}\ keyword defines the
activation time for a particular device. The format for the {\tt
DEVICE\_ACT}\ keyword is
\begin{lstlisting}
DEVICE_ACT
idevice time state
\end{lstlisting}
where {\tt time}\ is the activation time of the {\tt idevice}'th device.
State is the state of the device, 0 for off or in-active and 1 for
on or active.  If the device may be drawn more than two ways then
more than 2 states may be used with this keyword.

\hitemsmv{HEAT}The {\tt HEAT}\ keyword defines heat detector location
data. A FORTAN~2003 code segment describing the format of {\tt
HEAT}\ data is given by:
\begin{lstlisting}
   read(5,*)nheat
   do i = 1, nheat
     read(5,*)xheat(i),yheat(i),zheat(i)
   end do
\end{lstlisting}
where {\tt nheat}\ is the number of heat detectors and
{\tt xheat, yheat, zheat}\ are the {\tt x, y, z}\
coordinates of the heat detectors.

\hitemsmv{HEAT\_ACT}The {\tt HEAT\_ACT}\ keyword defines heat detector
activation data. A FORTAN~2003 code segment describing the format
of {\tt HEAT\_ACT}\ data is given by:
\begin{lstlisting}
   read(5,*)iheat, heat_time
\end{lstlisting}
where {\tt heat\_time}\ is the activation time of the {\tt iheat}'th heat
detector.

\hitemsmv{PROP}The {\tt PROP}\ keyword specifies
a list of general properties used
by Smokeview to customize the drawing of devices defined in
the objects.svo file.
The format of the {\tt PROP}\ keyword is

\begin{lstlisting}
PROP
 prop_id           (character string)
 smokeview_id      (character string)
 number of keyword/value pairs  (integer)
 keyword1=value1   (character string)
 ..
 ..
 keywordn=valuen   (character string)
 number of texture files (integer) (0 or 1 for now)
 texture file 1
 ...
 ...
 texture file n
\end{lstlisting}

\hitemsmv{SPRK}The {\tt SPRK}\ keyword defines sprinkler location
data. A FORTAN~2003 code segment describing the format of {\tt
SPRK}\ data is given by:
\begin{lstlisting}
   read(5,*)nsprink
   do i = 1, nsprink
     read(5,*)xsprink(i),ysprink(i),zsprink(i)
   end do
\end{lstlisting}
where {\tt nsprink}\ is the number of sprinklers and
{\tt xsprink, ysprink, zsprink}\ are the {\tt x, y, z}\
coordinates of the sprinklers.

\hitemsmv{SPRK\_ACT}The {\tt SPRK\_ACT}\ keyword defines sprinkler
activation data. A FORTAN~2003 code segment describing the format
of {\tt SPRK\_ACT}\ data is given by:
\begin{lstlisting}
   read(5,*)isprink, sprink_time
\end{lstlisting}
where {\tt sprink\_time}\ is the activation time of the {\tt isprink}'th sprinkler.

\hitemsmv{THCP}The {\tt THCP}\ keyword defines thermocouple location
data. A FORTAN~2003 code segment describing the format of {\tt
THCP}\ data is given by:
\begin{lstlisting}
   read(5,*)ntherm
   do i = 1, ntherm
     read(5,*)xtherm(i),ytherm(i),ztherm(i)
   end do
\end{lstlisting}
where {\tt ntherm}\ is the number of thermocouples and
{\tt xtherm, ytherm and ztherm}\ are the {\tt x, y and z}\
coordinates of the thermocouples.

\elist

%%-----------------------------------------------------------------------
\subsection{Zone Modeling Keywords}
This section contains documentation for keywords used to
describe features found in a zone fire model, features such as
rooms, vents, fires, etc. Smokeview also supports a number of the keywords described above
to support visualization of slice files, isosurface files, and devices for zone fire models.

\blist

\hitemsmv{FIRE}The {\tt FIRE}\ keyword defines the location and
room number of a zone fire modeling fire.
The format for the {\tt FIRE}\ keyword is
\begin{lstlisting}
FIRE
 i x y z
\end{lstlisting}
where {\tt i}\ is the room number containing the fire and $(x,y,z)$
is the location within the room of the base of the fire.
One {\tt FIRE}\ entry is specified  for each fire.
The order of the {\tt FIRE}\ entries found in the .smv file should correspond
to the ordering of the fires (i.e., The n'th fire is specified with
the n'th .smv {\tt FIRE}\ entry.)

\hitemsmv{VENTGEOM}The {\tt VENTGEOM}\ keyword defines the location,
orientation and size of a zone fire modeling vent (
with horizontal flow).
The format for the {\tt VENTGEOM}\ keyword is
\begin{lstlisting}
VENTGEOM
 from to face width ventoffset bottom top r g b
\end{lstlisting}
where {\tt from}\ and {\tt to}\  are the room indices
(ranging from 1 to the number of rooms) of the ``from'' and ``to'' rooms,
{\tt face}\ is the index of the wall (or face) where the
vent is located (front wall=1,
right wall=2, back wall=3, left wall=4), {\tt width}\ is
the vent width, {\tt bottom}\ is the elevation (relative to the floor)
of the vent sill, {\tt top}\
is the elevation (relative to the floor) of the vent soffit and {\tt r, g, b}\ are the red, green and blue components
(ranging from 0.0 to 1.0) of the vent color.
The order of the {\tt VENTGEOM}\ entries found in the .smv file should correspond
to the ordering of the vents  (i.e., The n'th vent is
specified with the n'th .smv {\tt VENTGEOM}\ entry.) Note that the VENTGEOM keyword is not used in CFAST 7 files and has
been replaced by {\tt HVENTPOS}\, {\tt VVENTPOS}\, and {\tt MVENTPOS}\ keywords described below.

\hitemsmv{HVENTPOS} defines the location, orientation, size, color, and initial opening for a zone fire modeling wall vent.
The format for the {\tt HVENTPOS}\ keyword is
\begin{lstlisting}
HVENTPOS
 from to x1 x2 y2 y2 z1 z2 red green blue opening
\end{lstlisting}
where {\tt from}\ and {\tt to}\  are the room indices
(ranging from 1 to the number of rooms) of the ``from'' and ``to'' rooms, {\tt x1 x2 y1 y2 z1 z2}\ define lower left and upper right
coordinates of the vent relative to the lower left/front corner of the {\tt from}\ compartment, {\tt red green blue}\ (optional) specify the
color of the drawn vent (color values range from 0.0 to 1.0), and {\tt opening}\ (optional) is the initial opening area of the vent.

\hitemsmv{VVENTPOS} defines the location, orientation, size, color, and initial opening for a zone fire modeling ceiling/floor vent.
The format for the {\tt VVENTPOS}\ keyword is
\begin{lstlisting}
VVENTPOS
 from to x1 x2 y2 y2 z1 z2 red green blue opening
\end{lstlisting}
where {\tt from}\ and {\tt to}\  are the room indices
(ranging from 1 to the number of rooms) of the ``from'' and ``to'' rooms, {\tt x1 x2 y1 y2 z1 z2}\ define lower left and upper right
coordinates of the vent relative to the lower left/front corner of the {\tt from}\ compartment, {\tt red green blue}\ (optional) specify the
color of the drawn vent (color values range from 0.0 to 1.0), and {\tt opening}\ (optional) is the initial opening area of the vent.

\hitemsmv{MVENTPOS} defines the location, orientation, size, color, and initial opening for a zone fire modeling wall vent.
The format for the {\tt MVENTPOS}\ keyword is
\begin{lstlisting}
MVENTPOS
 from to x1 x2 y2 y2 z1 z2 red green blue opening
\end{lstlisting}
where {\tt from}\ and {\tt to}\  are the room indices
(ranging from 1 to the number of rooms) of the ``from'' and ``to'' rooms, {\tt x1 x2 y1 y2 z1 z2}\ define lower left and upper right
coordinates of the vent relative to the lower left/front corner of the {\tt from}\ compartment, {\tt red green blue}\ (optional) specify the
color of the drawn vent (color values range from 0.0 to 1.0), and {\tt opening}\ (optional) is the initial opening area of the vent.

\hitemsmv{ROOM}The {\tt ROOM}\ keyword defines the size and
location of a zone fire modeling compartment or room.
The format for the {\tt ROOM}\ keyword is
\begin{lstlisting}
ROOM
 x y z
 x0 y0 z0
\end{lstlisting}\hitemsmv{ROOM}
where $(x,y,z)$ is the width, depth and height of the room
respectively and $(x0,y0,z0)$ is the location of the
left, front, bottom corner of the room.
The order of the {\tt ROOM}\ entries found in the .smv file should correspond
to the ordering of the rooms (i.e., The n'th room is specified
with the n'th .smv {\tt ROOM}\ entry.)

\hitemsmv{ZONE}The {\tt ZONE}\ keyword defines the file used to store
zone fire modeling data and the types of data
found within the file. The format for the {\tt ZONE}\ keyword is
\begin{lstlisting}
ZONE
 file
 long label
 short label
 unit
 long label
 short label
 unit
 long label
 short label
 unit
 long label
 short label
 unit
\end{lstlisting}

where {\tt file}\ is the name of the file containing the zone fire modeling data. {\tt long label}\, {\tt short label}\, and {\tt unit}\ describe
the columns of data in the {\tt ZONE file}\. Note, Smokeview does not use the label or unit data with CFAST 6 or later.
This data is found within the spread sheet data now used to store zone fire modeling data.
\elist


%%-----------------------------------------------------------------------
\subsection{Miscellaneous Keywords}
\blist

\hitemsmv{FDSVERSION}The {\tt FDSVERSION}\ keyword defines the GIT revision or
build number for the version of FDS that ran the case being visualized.
The FDS and Smokeview revisions are displayed in the Help menu.

\hitemsmv{TITLE1/TITLE2}The {\tt TITLE1}\ and {\tt TITLE2}\ keywords
allow one to specify extra information documenting a Smokeview
case.  These keywords and associated labels are added by hand to
the Smokeview (\tt .smv) file using the format:
\begin{lstlisting}
TITLE1
first line of descriptive text
TITLE2
second line of descriptive text
\end{lstlisting}

\elist


%---------------------------------------------------------------------------------
%---------------------------------------------------------------------------------
\section{CAD/GE1 file format}

A program called DXF2FDS, written by David Sheppard of the US Bureau of Alcohol,
Tobacco and Firearms (ATF), creates an FDS
input file and a Smokeview geometric description file (.GE1) given
a CAD description of the building being modeled.  The CAD
description must be in a {\em dxf}\ format and created using {\em 3DFACE}\ commands.
The .GE1 file has a simple text format and is
described below. DXF2FDS specifies that .GE1 filename on the {\tt \&DUMP }\
line in an FDS input file using the {\tt RENDER\_FILE}\
keyword,  as in

\begin{lstlisting}
&DUMP RENDER_FILE='Capecod.GE1' /
\end{lstlisting}

%Figure \ref{fig:cadexample}\ gives two views of a scene, a view used by
FDS to perform computations and a CAD view.
%The `q' may be pressed to switch between an {\em OBST}\ view and a {\em CAD}\ view of the scene.

%\begin{figure}[bph]
%\begin{center}
%\includegraphics[trim=0.0in 4.0in 0.0in 4.0in, clip, width=6.0in]{SCRIPT_FIGURES/capecodfds}
%\includegraphics[trim=0.0in 4.0in 0.0in 4.0in, clip, width=6.0in]{SCRIPT_FIGURES/capecodcad}
%\end{center}
%\caption{Example Smokeview rendering using .fds and .GE1 files generated by DXF2FDS.  Blockage and CAD representation of the scene may be toggled by pressing the `q' key.}
%\label{fig:cadexample}%
%\end{figure}

\begin{lstlisting}
[APPEARANCE]
nappearances
string (material description)
index r g b twidth, theight, alpha, shininess, tx0, ty0, tz0
tfile
:
:   The above entry is repeated nappearances-1 more times
:
[FACES]
nfaces
x1 y1 z1 x2 y2 z2 x3 y3 z3 x4 y4 z4 index
The above line is repeated nfaces-1 more times.
\end{lstlisting}

\blist

\hitem{nappearances}\ Number of appearance entries to follow
Each appearance entry has 3 lines.
\hitem{string}\ A material description is written out by DX2FDS but is ignored by Smokeview.
\hitem{index}\ An index number starting at 0.
\hitem{r, g, b}\ Red green and blue components of the CAD face used when a texture is not drawn.
The values of r, g and b range from 0 to 255.  If a color is not used
then use -1 for each color component.  In this case, the CAD face is opaque regardless of the alpha value specified.
\hitem{twidth, theight}\ Textures are tiled or repeated.
The characteristic width and length of the
texture file is twidth and theight respectively.
\hitem{alpha}Opaqueness value of the cad element being drawn.
Values may range from 0.0 (completely transparent)
to 1.0 (completely opaque.  (default: 1.0)
\hitem{shininess}\ Shininess value of the cad element being drawn.
Values may be larger than 0.0 .  (default: 800.0)
\hitem{tx0, ty0, tz0}\ x, y and z values in physical coordinates of the offset used to apply a
texture to a cad element. (default: 0.0, 0.0, 0.0)
\hitem{tfile}\ The name of the texture file.  If one is not used or
available then leave this line blank.
\hitem{nfaces}\ Number of face entries to follow.  Each face entry has one line.
\hitem{x1/y1/z1/.../x4/y4/z4}x,y,z coordinates of a quadrilateral.  T
he four corners of the quad
must lie in a plane or weird effects may result when Smokeview draws it.
(This is a requirement
of OpenGL).  The four points should be in counter-clockwise order.
\hitem{index}\ Points to a material in the [APPEARANCE] section.
\elist

%---------------------------------------------------------------------------------
%---------------------------------------------------------------------------------
\section{Objects.svo}
\label{section:objects}
\lstinputlisting{../../../bot/Bundlebot/smv/for_bundle/objects.svo}

\end{document}
